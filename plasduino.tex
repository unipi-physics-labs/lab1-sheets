\newcommand{\plasduinodoctext}[1]%
{
  Una volta acceso il calcolatore, selezionare dal men\`u principale
  (in alto a sinistra) \menuitem{Application}~$\rightarrow$~%
  \menuitem{Education}~$\rightarrow$~\menuitem{plasduino}. Questo dovrebbe
  mostrare la finestra principale del programma di acquisizione. Per questa
  esperienza, tra la lista dei moduli, lanciate~\menuitem{#1} (doppio
  click sulla linea corrispondente, oppure selezionate la linea stessa e
  premete \menuitem{Open}).
}


\newcommand{\plasduinodoc}[1]%
{
  \labsection{Note sul programma di acquisizione}

  \plasduinodoctext{#1}
}

\newcommand{\plasduinosave}%
{
  Di norma al termine di ogni sessione di presa dati il programma vi chiede se
  volete salvare una copia del \emph{file} dei dati in una cartella a vostra
  scelta (il che pu\`o essere comodo per l'analisi successiva). Se questa
  funzionalit\`a dovesse essere disabilitata potete ri-abilitarla 
  attraverso il men\`u di plasduino \menuitem{Configuration}~%
  $\rightarrow$~\menuitem{Change settings}: nella finestra che si apre
  selezionate il tab~\menuitem{daq} e abilitate l'opzione
  \menuitem{prompt-save-dialog}.
}
