\documentclass{lab1-article}

\title{Moto di un volano}

\newcommand{\plasduinodoctext}[1]%
{
  Una volta acceso il calcolatore, selezionare dal men\`u principale
  (in alto a sinistra) \menuitem{Application}~$\rightarrow$~%
  \menuitem{Education}~$\rightarrow$~\menuitem{plasduino}. Questo dovrebbe
  mostrare la finestra principale del programma di acquisizione. Per questa
  esperienza, tra la lista dei moduli, lanciate~\menuitem{#1} (doppio
  click sulla linea corrispondente, oppure selezionate la linea stessa e
  premete \menuitem{Open}).
}


\newcommand{\plasduinodoc}[1]%
{
  \labsection{Note sul programma di acquisizione}

  \plasduinodoctext{#1}
}

\newcommand{\plasduinosave}%
{
  Di norma al termine di ogni sessione di presa dati il programma vi chiede se
  volete salvare una copia del \emph{file} dei dati in una cartella a vostra
  scelta (il che pu\`o essere comodo per l'analisi successiva). Se questa
  funzionalit\`a dovesse essere disabilitata potete ri-abilitarla 
  attraverso il men\`u di plasduino \menuitem{Configuration}~%
  $\rightarrow$~\menuitem{Change settings}: nella finestra che si apre
  selezionate il tab~\menuitem{daq} e abilitate l'opzione
  \menuitem{prompt-save-dialog}.
}


\begin{document}


\begin{article}
\selectlanguage{italian}

\maketitle

\secsummary

Lo scopo dell'esperienza \`e lo studio del moto di un volano sotto l'azione
di una forza esterna.

\secmaterials

\begin{itemize}
\item Un volano dotato di \emph{encoder};
\item un piattino ed una serie di pesetti;
\item sistema di acquisizione.
\end{itemize}

\secmeasurements

L'equazione fondamentale che regola il moto del nostro sistema \`e
\begin{align}\label{eq:eq_moto}
  I \frac{d\overrightarrow\omega}{dt} = \overrightarrow\tau,
\end{align}
dove $I$ \`e il momento di inerzia del volano (i.e., di \emph{tutte} le
componenti del volano), $\overrightarrow\omega$ la sua velocit\`a angolare e
$\overrightarrow\tau$ il momento totale delle forze esterne.


\labsubsection{Misura del momento della forza di attrito}

Nel caso di moto libero l'unica forza esterna che fa momento \`e quella di
attrito. La prima misura richiesta \`e proprio la stima del modulo $\tau_a$
del momento della forza di attrito attraverso la misura
dell'accelerazione (o, meglio, decelerazione) angolare per il moto
libero. Assumendo che $\tau_a$ non dipenda da $\omega$ (si tratta di un'ipotesi
di lavoro da verificare a posteriori), per la~\eqref{eq:eq_moto} si ha
\begin{align}
  \omega(t) = \omega_0 - \frac{\tau_a}{I}t.
\end{align}
Si costruisca un grafico cartesiano della velocit\`a in funzione del tempo e si
esegua un \emph{fit} con una retta. Dal coefficiente angolare, e avendo
preliminarmente stimato $I$, si pu\`o ricavare $\tau_a$.
Per la stima di $I$ si schematizzi la parte mobile dell'apparato come una serie
di dischi coassiali di alluminio ($\varrho_{\rm Al} = 2.70 \pm 0.02$~g/cm$^3$),
ricordando che il momento di inerzia $I_d$ di un disco di raggio $r_d$ e massa
$m_d$ \`e
\begin{align}
  I_d = \frac{1}{2}m_dr_d^2.
\end{align}

L'ipotesi di lavoro iniziale (cio\`e la costanza del momento della forza
di attrito) \`e verificata?


\labsubsection{Moto sotto l'azione di una forza esterna}

Si avvolga un filo di nylon attorno al rocchetto (di raggio $r$) coassiale con
il volano. All'estremit\`a libera del filo, tramite un piattello, si appenda
una massa sufficiente a mettere in moto il volano (nel seguito chiameremo
$m$ la massa totale di pesino pi\`u piattello).

Durante la fase di discesa del pesino, detto $T_d$ il modulo della tensione del
filo e $z$ la posizione verticale del pesino, le equazioni del moto del
sistema sono
\begin{align}
  m\ddot{z_d} &= mg - T_d\\
  I \frac{d\omega_d}{dt} &= T_dr - \tau_a,
\end{align}
(abbiamo assunto l'asse $z$ orientato verso il basso). A queste due va
aggiunta la condizione di inestensibilit\`a del filo, che si scrive come
\begin{align}
  \ddot{z_d} = r\frac{d\omega_d}{dt},
\end{align}
che permette di scrivere la soluzione in forma chiusa:
\begin{align}
  \frac{d\omega_d}{dt} = \frac{mgr - \tau_a}{I + mr^2} \quad (> 0).
\end{align}
Analogamente per la fase di salita:
\begin{align}
  \frac{d\omega_s}{dt} = \frac{-mgr - \tau_a}{I + mr^2} \quad (<0). 
\end{align}

Si misurino le accelerazioni angolari del sistema durante la salita e la
discesa e si confrontino i valori ottenuti con quelli attesi dalla teoria.


\secconsiderations

Si consiglia di eseguire la misura iniziale di $\tau_a$ per valori di
$\omega$ non troppo diversi da quelli che si registrano durante il moto
forzato.

Per completezza, il momento di inerzia $I$ del sistema dovrebbe essere
dell'ordine $\sim 3 \times 10^5$~g~cm$^2$.


\subsecdataformat

Il programma di acquisizione fornisce un \emph{file} di uscita contenente due
colonne che rappresentano, rispettivamente:
\begin{enumerate}
\item il tempo $t$ (in s), dall'inizio dell'acquisizione;
\item la velocit\`a angolare (in rad/s) mediata sull'ultimo $1/20$ di giro
(l'\emph{encoder} \`e a 20 segmenti).
\end{enumerate}

\plasduinodoc{Wheel}

\plasduinosave

\end{article}
\end{document}
