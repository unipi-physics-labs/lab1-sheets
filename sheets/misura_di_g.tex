\documentclass{lab1-article}

\title{Misura di $g$ utilizzando una molla}

\usepackage{fancyvrb}
\makeatletter
\def\PY@reset{\let\PY@it=\relax \let\PY@bf=\relax%
    \let\PY@ul=\relax \let\PY@tc=\relax%
    \let\PY@bc=\relax \let\PY@ff=\relax}
\def\PY@tok#1{\csname PY@tok@#1\endcsname}
\def\PY@toks#1+{\ifx\relax#1\empty\else%
    \PY@tok{#1}\expandafter\PY@toks\fi}
\def\PY@do#1{\PY@bc{\PY@tc{\PY@ul{%
    \PY@it{\PY@bf{\PY@ff{#1}}}}}}}
\def\PY#1#2{\PY@reset\PY@toks#1+\relax+\PY@do{#2}}

\expandafter\def\csname PY@tok@gd\endcsname{\def\PY@tc##1{\textcolor[rgb]{0.63,0.00,0.00}{##1}}}
\expandafter\def\csname PY@tok@gu\endcsname{\let\PY@bf=\textbf\def\PY@tc##1{\textcolor[rgb]{0.50,0.00,0.50}{##1}}}
\expandafter\def\csname PY@tok@gt\endcsname{\def\PY@tc##1{\textcolor[rgb]{0.00,0.27,0.87}{##1}}}
\expandafter\def\csname PY@tok@gs\endcsname{\let\PY@bf=\textbf}
\expandafter\def\csname PY@tok@gr\endcsname{\def\PY@tc##1{\textcolor[rgb]{1.00,0.00,0.00}{##1}}}
\expandafter\def\csname PY@tok@cm\endcsname{\let\PY@it=\textit\def\PY@tc##1{\textcolor[rgb]{0.25,0.50,0.50}{##1}}}
\expandafter\def\csname PY@tok@vg\endcsname{\def\PY@tc##1{\textcolor[rgb]{0.10,0.09,0.49}{##1}}}
\expandafter\def\csname PY@tok@m\endcsname{\def\PY@tc##1{\textcolor[rgb]{0.40,0.40,0.40}{##1}}}
\expandafter\def\csname PY@tok@mh\endcsname{\def\PY@tc##1{\textcolor[rgb]{0.40,0.40,0.40}{##1}}}
\expandafter\def\csname PY@tok@go\endcsname{\def\PY@tc##1{\textcolor[rgb]{0.53,0.53,0.53}{##1}}}
\expandafter\def\csname PY@tok@ge\endcsname{\let\PY@it=\textit}
\expandafter\def\csname PY@tok@vc\endcsname{\def\PY@tc##1{\textcolor[rgb]{0.10,0.09,0.49}{##1}}}
\expandafter\def\csname PY@tok@il\endcsname{\def\PY@tc##1{\textcolor[rgb]{0.40,0.40,0.40}{##1}}}
\expandafter\def\csname PY@tok@cs\endcsname{\let\PY@it=\textit\def\PY@tc##1{\textcolor[rgb]{0.25,0.50,0.50}{##1}}}
\expandafter\def\csname PY@tok@cp\endcsname{\def\PY@tc##1{\textcolor[rgb]{0.74,0.48,0.00}{##1}}}
\expandafter\def\csname PY@tok@gi\endcsname{\def\PY@tc##1{\textcolor[rgb]{0.00,0.63,0.00}{##1}}}
\expandafter\def\csname PY@tok@gh\endcsname{\let\PY@bf=\textbf\def\PY@tc##1{\textcolor[rgb]{0.00,0.00,0.50}{##1}}}
\expandafter\def\csname PY@tok@ni\endcsname{\let\PY@bf=\textbf\def\PY@tc##1{\textcolor[rgb]{0.60,0.60,0.60}{##1}}}
\expandafter\def\csname PY@tok@nl\endcsname{\def\PY@tc##1{\textcolor[rgb]{0.63,0.63,0.00}{##1}}}
\expandafter\def\csname PY@tok@nn\endcsname{\let\PY@bf=\textbf\def\PY@tc##1{\textcolor[rgb]{0.00,0.00,1.00}{##1}}}
\expandafter\def\csname PY@tok@no\endcsname{\def\PY@tc##1{\textcolor[rgb]{0.53,0.00,0.00}{##1}}}
\expandafter\def\csname PY@tok@na\endcsname{\def\PY@tc##1{\textcolor[rgb]{0.49,0.56,0.16}{##1}}}
\expandafter\def\csname PY@tok@nb\endcsname{\def\PY@tc##1{\textcolor[rgb]{0.00,0.50,0.00}{##1}}}
\expandafter\def\csname PY@tok@nc\endcsname{\let\PY@bf=\textbf\def\PY@tc##1{\textcolor[rgb]{0.00,0.00,1.00}{##1}}}
\expandafter\def\csname PY@tok@nd\endcsname{\def\PY@tc##1{\textcolor[rgb]{0.67,0.13,1.00}{##1}}}
\expandafter\def\csname PY@tok@ne\endcsname{\let\PY@bf=\textbf\def\PY@tc##1{\textcolor[rgb]{0.82,0.25,0.23}{##1}}}
\expandafter\def\csname PY@tok@nf\endcsname{\def\PY@tc##1{\textcolor[rgb]{0.00,0.00,1.00}{##1}}}
\expandafter\def\csname PY@tok@si\endcsname{\let\PY@bf=\textbf\def\PY@tc##1{\textcolor[rgb]{0.73,0.40,0.53}{##1}}}
\expandafter\def\csname PY@tok@s2\endcsname{\def\PY@tc##1{\textcolor[rgb]{0.73,0.13,0.13}{##1}}}
\expandafter\def\csname PY@tok@vi\endcsname{\def\PY@tc##1{\textcolor[rgb]{0.10,0.09,0.49}{##1}}}
\expandafter\def\csname PY@tok@nt\endcsname{\let\PY@bf=\textbf\def\PY@tc##1{\textcolor[rgb]{0.00,0.50,0.00}{##1}}}
\expandafter\def\csname PY@tok@nv\endcsname{\def\PY@tc##1{\textcolor[rgb]{0.10,0.09,0.49}{##1}}}
\expandafter\def\csname PY@tok@s1\endcsname{\def\PY@tc##1{\textcolor[rgb]{0.73,0.13,0.13}{##1}}}
\expandafter\def\csname PY@tok@sh\endcsname{\def\PY@tc##1{\textcolor[rgb]{0.73,0.13,0.13}{##1}}}
\expandafter\def\csname PY@tok@sc\endcsname{\def\PY@tc##1{\textcolor[rgb]{0.73,0.13,0.13}{##1}}}
\expandafter\def\csname PY@tok@sx\endcsname{\def\PY@tc##1{\textcolor[rgb]{0.00,0.50,0.00}{##1}}}
\expandafter\def\csname PY@tok@bp\endcsname{\def\PY@tc##1{\textcolor[rgb]{0.00,0.50,0.00}{##1}}}
\expandafter\def\csname PY@tok@c1\endcsname{\let\PY@it=\textit\def\PY@tc##1{\textcolor[rgb]{0.25,0.50,0.50}{##1}}}
\expandafter\def\csname PY@tok@kc\endcsname{\let\PY@bf=\textbf\def\PY@tc##1{\textcolor[rgb]{0.00,0.50,0.00}{##1}}}
\expandafter\def\csname PY@tok@c\endcsname{\let\PY@it=\textit\def\PY@tc##1{\textcolor[rgb]{0.25,0.50,0.50}{##1}}}
\expandafter\def\csname PY@tok@mf\endcsname{\def\PY@tc##1{\textcolor[rgb]{0.40,0.40,0.40}{##1}}}
\expandafter\def\csname PY@tok@err\endcsname{\def\PY@bc##1{\setlength{\fboxsep}{0pt}\fcolorbox[rgb]{1.00,0.00,0.00}{1,1,1}{\strut ##1}}}
\expandafter\def\csname PY@tok@kd\endcsname{\let\PY@bf=\textbf\def\PY@tc##1{\textcolor[rgb]{0.00,0.50,0.00}{##1}}}
\expandafter\def\csname PY@tok@ss\endcsname{\def\PY@tc##1{\textcolor[rgb]{0.10,0.09,0.49}{##1}}}
\expandafter\def\csname PY@tok@sr\endcsname{\def\PY@tc##1{\textcolor[rgb]{0.73,0.40,0.53}{##1}}}
\expandafter\def\csname PY@tok@mo\endcsname{\def\PY@tc##1{\textcolor[rgb]{0.40,0.40,0.40}{##1}}}
\expandafter\def\csname PY@tok@kn\endcsname{\let\PY@bf=\textbf\def\PY@tc##1{\textcolor[rgb]{0.00,0.50,0.00}{##1}}}
\expandafter\def\csname PY@tok@mi\endcsname{\def\PY@tc##1{\textcolor[rgb]{0.40,0.40,0.40}{##1}}}
\expandafter\def\csname PY@tok@gp\endcsname{\let\PY@bf=\textbf\def\PY@tc##1{\textcolor[rgb]{0.00,0.00,0.50}{##1}}}
\expandafter\def\csname PY@tok@o\endcsname{\def\PY@tc##1{\textcolor[rgb]{0.40,0.40,0.40}{##1}}}
\expandafter\def\csname PY@tok@kr\endcsname{\let\PY@bf=\textbf\def\PY@tc##1{\textcolor[rgb]{0.00,0.50,0.00}{##1}}}
\expandafter\def\csname PY@tok@s\endcsname{\def\PY@tc##1{\textcolor[rgb]{0.73,0.13,0.13}{##1}}}
\expandafter\def\csname PY@tok@kp\endcsname{\def\PY@tc##1{\textcolor[rgb]{0.00,0.50,0.00}{##1}}}
\expandafter\def\csname PY@tok@w\endcsname{\def\PY@tc##1{\textcolor[rgb]{0.73,0.73,0.73}{##1}}}
\expandafter\def\csname PY@tok@kt\endcsname{\def\PY@tc##1{\textcolor[rgb]{0.69,0.00,0.25}{##1}}}
\expandafter\def\csname PY@tok@ow\endcsname{\let\PY@bf=\textbf\def\PY@tc##1{\textcolor[rgb]{0.67,0.13,1.00}{##1}}}
\expandafter\def\csname PY@tok@sb\endcsname{\def\PY@tc##1{\textcolor[rgb]{0.73,0.13,0.13}{##1}}}
\expandafter\def\csname PY@tok@k\endcsname{\let\PY@bf=\textbf\def\PY@tc##1{\textcolor[rgb]{0.00,0.50,0.00}{##1}}}
\expandafter\def\csname PY@tok@se\endcsname{\let\PY@bf=\textbf\def\PY@tc##1{\textcolor[rgb]{0.73,0.40,0.13}{##1}}}
\expandafter\def\csname PY@tok@sd\endcsname{\let\PY@it=\textit\def\PY@tc##1{\textcolor[rgb]{0.73,0.13,0.13}{##1}}}

\def\PYZbs{\char`\\}
\def\PYZus{\char`\_}
\def\PYZob{\char`\{}
\def\PYZcb{\char`\}}
\def\PYZca{\char`\^}
\def\PYZam{\char`\&}
\def\PYZlt{\char`\<}
\def\PYZgt{\char`\>}
\def\PYZsh{\char`\#}
\def\PYZpc{\char`\%}
\def\PYZdl{\char`\$}
\def\PYZhy{\char`\-}
\def\PYZsq{\char`\'}
\def\PYZdq{\char`\"}
\def\PYZti{\char`\~}
% for compatibility with earlier versions
\def\PYZat{@}
\def\PYZlb{[}
\def\PYZrb{]}
\makeatother


\begin{document}


\begin{article}
\selectlanguage{italian}

\maketitle

\secsummary
Lo scopo dell'esperienza \`e la misura dell'accelerazione di gravit\`a al
livello del suolo ($g$) a partire dagli allungamenti di una molla.


\secmaterials

\begin{itemize}
\item Una molla, un piattino ed una serie di pesetti.
\item Cronometro (risoluzione 0.01~s).
\item Bilancia di precisione (risoluzione 1~mg).
\item Metro a nastro (risoluzione 1~mm).
\end{itemize}


\secmeasurements

Se gli allungamenti della molla non sono troppo grandi, vale la legge di Hooke:
l'allungamento \`e proporzionale all'intensit\`a della forza che lo ha
causato. In condizioni di equilibrio, risulta allora
\begin{align}\label{eq:equilibrio}
(m_p + m_i)g = k(l_i - l_0),
\end{align}
dove $m_p$ \`e la massa del piattello, $m_i$ \`e la massa posta sul piattello
nella $i$-esima misura, $l_i$ \`e la lunghezza della molla sotto carico, $l_0$
\`e la lunghezza della molla a riposo e $k$ la costante elastica della molla
stessa.


\labsubsection{Stima di $k$}

Se mettiamo in oscillazione la molla (di massa $m$ non trascurabile) il periodo
del suo moto armonico vale
\begin{align}
T_i = 2\pi\sqrt{\frac{(m_p + m_i + m/3)}{k}}.
\end{align}
Si metta in oscillazione il sistema per diversi valori di $m_i$ e si faccia
un grafico del quadrato del periodo di oscillazione in funzione del carico:
\begin{align}
T^2_i = \frac{4\pi^2}{k}m_i + \frac{4\pi^2}{k}(m_p + m/3).
\end{align}
Dalla stima, tramite \emph{fit} al calcolatore, del coefficiente angolare \`e
possibile stimare $k$ (e, propagando opportunamente l'errore, l'incertezza
associata).


\labsubsection{Stima di $g$}

Dalla \eqref{eq:equilibrio}, con semplici calcoli, si ricava
\begin{align}
l_i = \frac{g}{k}m_i + \frac{g}{k}m_p + l_0
\end{align}
per cui si pu\`o ricavare $g/k$ dalla stima, tramite \emph{fit} al calcolatore,
del coefficiente angolare della retta ottenuta riportando le coppie di dati
$(m_i,~l_i)$ su carta millimetrata.

Utilizzando la stima di $k$ ottenuta sopra si ricavi $g$ e l'incertezza
associata $\Delta g$. Fare il confronto tra il valore trovato e il valore noto
per Pisa 9.807~m~s$^{-2}$.


\secconsiderations

\labsubsection{Composizione del carico}

Si consiglia di porre sul piattello una o pi\`u masse campione
(2 o 3 ci stanno senza problemi) in modo da misurare una decina di allungamenti
al variare delle masse (per esempio da 5 a 50~g).


\labsubsection{Incertezze sulle lunghezze della molla}

Il millimetro di incertezza dovuto alla risoluzione del metro a nastro potrebbe
non essere sufficiente come stima dell'errore massimo, dato che spesso la
misura viene letta mentre il piattello non \`e perfettamente fermo o non \`e
perfettamente orizzontale: una stima \emph{a priori} di 2~mm per l'errore
massimo assoluto sembra pi\`u ragionevole.

In ogni caso si consiglia un controllo \emph{a posteriori} dell'errore.
Si osservi attentamente il grafico $(m_i,~l_i)$: se le misure sono state prese
con cura, gli scarti dei punti sperimentali dalla retta tracciata potrebbero
risultare tutti inferiori ai 2~mm supposti \emph{a priori}. Se \`e cos\`i,
riducete pure l'errore sulle lunghezze ad un solo mm, altrimenti confermate la
stima di 2~mm.

\labsubsection{Incertezze sulle misure di tempo}

Anche se la risoluzione del cronometro usato vale 0.01~s, \`e illusorio
pensare che questo sia l'errore di misura da attribuire alle misurazioni
manuali della durata temporale di un qualunque fenomeno fisico.

Nella fattispecie, per ridurre l'impatto del tempo di reazione, si consiglia
di misurare il tempo $\tau$ che il sistema impiega a compiere $10$
oscillazioni complete. Per stimare l'errore associato a $\tau$ si ripeta la
misure $n$ volte (con $n \geq 5$); il valor medio delle varie misure effettuate
\begin{align}
  m_{\tau} = \frac{1}{n}\sum_{i = 1}^n \tau_i
\end{align}
sar\`a assunto come miglior stima della durata temporale del fenomeno e la
deviazione standard della media
\begin{align}
  s_\tau = \sqrt{\frac{1}{n(n-1)}\sum_{i = 1}^n(\tau_i - m_\tau)^2}
\end{align}
come errore associato (nel corso dell'anno avremo occasione di discutere il
significato quantitativo di questa espressione nel caso di piccoli campioni).

Va da s\'e che si passa da $\tau$ a $T$ dividendo per $10$ (se abbiamo
misurato $10$ oscillazioni complete) sia la misura che l'errore.

\onecolumn

\begin{Verbatim}[label=\makebox{\href{https://github.com/unipi-physics-labs/lab1-sheets/tree/main/snippy/misura_di_g.py}{https://github.com/.../misura\_di\_g.py}},commandchars=\\\{\}]
\PY{k+kn}{import}\PY{+w}{ }\PY{n+nn}{numpy}\PY{+w}{ }\PY{k}{as}\PY{+w}{ }\PY{n+nn}{np}
\PY{k+kn}{from}\PY{+w}{ }\PY{n+nn}{matplotlib}\PY{+w}{ }\PY{k+kn}{import} \PY{n}{pyplot} \PY{k}{as} \PY{n}{plt}
\PY{k+kn}{from}\PY{+w}{ }\PY{n+nn}{scipy}\PY{n+nn}{.}\PY{n+nn}{optimize}\PY{+w}{ }\PY{k+kn}{import} \PY{n}{curve\PYZus{}fit}

\PY{c+c1}{\PYZsh{} Misura di allungamento e periodo di oscillazione in funzione della massa appesa.}
\PY{n}{m} \PY{o}{=} \PY{n}{np}\PY{o}{.}\PY{n}{array}\PY{p}{(}\PY{p}{[}\PY{l+m+mf}{5.005}\PY{p}{,} \PY{l+m+mf}{10.006}\PY{p}{,} \PY{l+m+mf}{20.011}\PY{p}{,} \PY{l+m+mf}{50.032}\PY{p}{]}\PY{p}{)}
\PY{n}{sigma\PYZus{}m} \PY{o}{=} \PY{n}{np}\PY{o}{.}\PY{n}{array}\PY{p}{(}\PY{p}{[}\PY{l+m+mf}{0.001}\PY{p}{]} \PY{o}{*} \PY{n+nb}{len}\PY{p}{(}\PY{n}{m}\PY{p}{)}\PY{p}{)}
\PY{n}{l} \PY{o}{=} \PY{n}{np}\PY{o}{.}\PY{n}{array}\PY{p}{(}\PY{p}{[}\PY{l+m+mf}{152.}\PY{p}{,} \PY{l+m+mf}{171.}\PY{p}{,} \PY{l+m+mf}{220.}\PY{p}{,} \PY{l+m+mf}{351.}\PY{p}{]}\PY{p}{)}
\PY{n}{sigma\PYZus{}l} \PY{o}{=} \PY{n}{np}\PY{o}{.}\PY{n}{array}\PY{p}{(}\PY{p}{[}\PY{l+m+mf}{1.}\PY{p}{]} \PY{o}{*} \PY{n+nb}{len}\PY{p}{(}\PY{n}{l}\PY{p}{)}\PY{p}{)}
\PY{n}{T} \PY{o}{=} \PY{n}{np}\PY{o}{.}\PY{n}{array}\PY{p}{(}\PY{p}{[}\PY{l+m+mf}{0.530}\PY{p}{,} \PY{l+m+mf}{0.611}\PY{p}{,} \PY{l+m+mf}{0.739}\PY{p}{,} \PY{l+m+mf}{1.044}\PY{p}{]}\PY{p}{)}
\PY{n}{sigma\PYZus{}T} \PY{o}{=} \PY{n}{np}\PY{o}{.}\PY{n}{array}\PY{p}{(}\PY{p}{[}\PY{l+m+mf}{0.005}\PY{p}{,} \PY{l+m+mf}{0.005}\PY{p}{,} \PY{l+m+mf}{0.006}\PY{p}{,} \PY{l+m+mf}{0.004}\PY{p}{]}\PY{p}{)}
\PY{c+c1}{\PYZsh{} Calculate the square of the period and propagate the errors.}
\PY{n}{T2} \PY{o}{=} \PY{n}{T}\PY{o}{*}\PY{o}{*}\PY{l+m+mf}{2.}
\PY{n}{sigma\PYZus{}T2} \PY{o}{=} \PY{l+m+mi}{2} \PY{o}{*} \PY{n}{T} \PY{o}{*} \PY{n}{sigma\PYZus{}T}

\PY{k}{def}\PY{+w}{ }\PY{n+nf}{line}\PY{p}{(}\PY{n}{x}\PY{p}{,} \PY{n}{m}\PY{p}{,} \PY{n}{q}\PY{p}{)}\PY{p}{:}
\PY{+w}{    }\PY{l+s+sd}{\PYZdq{}\PYZdq{}\PYZdq{}Funzione di fit (una semplice retta).}
\PY{l+s+sd}{    \PYZdq{}\PYZdq{}\PYZdq{}}
    \PY{k}{return} \PY{n}{m} \PY{o}{*} \PY{n}{x} \PY{o}{+} \PY{n}{q}

\PY{n}{plt}\PY{o}{.}\PY{n}{figure}\PY{p}{(}\PY{l+s+s1}{\PYZsq{}}\PY{l+s+s1}{Grafico allungamento\PYZhy{}massa}\PY{l+s+s1}{\PYZsq{}}\PY{p}{)}
\PY{n}{plt}\PY{o}{.}\PY{n}{errorbar}\PY{p}{(}\PY{n}{m}\PY{p}{,} \PY{n}{l}\PY{p}{,} \PY{n}{sigma\PYZus{}l}\PY{p}{,} \PY{n}{sigma\PYZus{}m}\PY{p}{,} \PY{n}{fmt}\PY{o}{=}\PY{l+s+s1}{\PYZsq{}}\PY{l+s+s1}{o}\PY{l+s+s1}{\PYZsq{}}\PY{p}{)}
\PY{c+c1}{\PYZsh{} Il fit in dettaglio: questa e` la funzione che esegue il fit e restituisce i parametri di best\PYZhy{}fit e}
\PY{c+c1}{\PYZsh{} tutto quello che serve (la cosiddetta matrice di covarianza) per stimare gli errori associati. Per il}
\PY{c+c1}{\PYZsh{} momento non passiamo le incertezze di misura al fit (per un motivo che vedremo piu` avanti) ma}
\PY{c+c1}{\PYZsh{} ricordate che in generale questo non e` corretto.}
\PY{n}{popt}\PY{p}{,} \PY{n}{pcov} \PY{o}{=} \PY{n}{curve\PYZus{}fit}\PY{p}{(}\PY{n}{line}\PY{p}{,} \PY{n}{m}\PY{p}{,} \PY{n}{l}\PY{p}{)}
\PY{c+c1}{\PYZsh{} Spacchettiamo l\PYZsq{}array dei parametri per averli disponibili separatamente}
\PY{n}{m\PYZus{}fit}\PY{p}{,} \PY{n}{q\PYZus{}fit} \PY{o}{=} \PY{n}{popt}
\PY{c+c1}{\PYZsh{} Calcoliamo le incertezze di misura (a questo livello l\PYZsq{}unica cosa che dovete sapere e` che sono la}
\PY{c+c1}{\PYZsh{} radice quadrata degli elementi diagonali della matrice di covarianza).}
\PY{n}{sigma\PYZus{}m\PYZus{}fit}\PY{p}{,} \PY{n}{sigma\PYZus{}q\PYZus{}fit} \PY{o}{=} \PY{n}{np}\PY{o}{.}\PY{n}{sqrt}\PY{p}{(}\PY{n}{pcov}\PY{o}{.}\PY{n}{diagonal}\PY{p}{(}\PY{p}{)}\PY{p}{)}
\PY{c+c1}{\PYZsh{} Facciamo stampare i parametri (per la relazione non dimenticate di convertire nelle unita` di misura}
\PY{c+c1}{\PYZsh{} opportune e di scrivere il numero corretto di cifre significative)}
\PY{n+nb}{print}\PY{p}{(}\PY{n}{m\PYZus{}fit}\PY{p}{,} \PY{n}{sigma\PYZus{}m\PYZus{}fit}\PY{p}{,} \PY{n}{q\PYZus{}fit}\PY{p}{,} \PY{n}{sigma\PYZus{}q\PYZus{}fit}\PY{p}{)}
\PY{c+c1}{\PYZsh{} Infine: facciamo il grafico del modello di best fit.}
\PY{n}{x} \PY{o}{=} \PY{n}{np}\PY{o}{.}\PY{n}{linspace}\PY{p}{(}\PY{l+m+mf}{0.}\PY{p}{,} \PY{l+m+mf}{60.}\PY{p}{,} \PY{l+m+mi}{100}\PY{p}{)}
\PY{n}{plt}\PY{o}{.}\PY{n}{plot}\PY{p}{(}\PY{n}{x}\PY{p}{,} \PY{n}{line}\PY{p}{(}\PY{n}{x}\PY{p}{,} \PY{n}{m\PYZus{}fit}\PY{p}{,} \PY{n}{q\PYZus{}fit}\PY{p}{)}\PY{p}{)}
\PY{n}{plt}\PY{o}{.}\PY{n}{xlabel}\PY{p}{(}\PY{l+s+s1}{\PYZsq{}}\PY{l+s+s1}{Massa [g]}\PY{l+s+s1}{\PYZsq{}}\PY{p}{)}
\PY{n}{plt}\PY{o}{.}\PY{n}{ylabel}\PY{p}{(}\PY{l+s+s1}{\PYZsq{}}\PY{l+s+s1}{Allungamento [mm]}\PY{l+s+s1}{\PYZsq{}}\PY{p}{)}
\PY{n}{plt}\PY{o}{.}\PY{n}{grid}\PY{p}{(}\PY{n}{ls}\PY{o}{=}\PY{l+s+s1}{\PYZsq{}}\PY{l+s+s1}{dashed}\PY{l+s+s1}{\PYZsq{}}\PY{p}{)}

\PY{n}{plt}\PY{o}{.}\PY{n}{figure}\PY{p}{(}\PY{l+s+s1}{\PYZsq{}}\PY{l+s+s1}{Grafico periodo quadro\PYZhy{}massa}\PY{l+s+s1}{\PYZsq{}}\PY{p}{)}
\PY{n}{plt}\PY{o}{.}\PY{n}{errorbar}\PY{p}{(}\PY{n}{m}\PY{p}{,} \PY{n}{T2}\PY{p}{,} \PY{n}{sigma\PYZus{}T2}\PY{p}{,} \PY{n}{sigma\PYZus{}m}\PY{p}{,} \PY{n}{fmt}\PY{o}{=}\PY{l+s+s1}{\PYZsq{}}\PY{l+s+s1}{o}\PY{l+s+s1}{\PYZsq{}}\PY{p}{)}
\PY{n}{popt}\PY{p}{,} \PY{n}{pcov} \PY{o}{=} \PY{n}{curve\PYZus{}fit}\PY{p}{(}\PY{n}{line}\PY{p}{,} \PY{n}{m}\PY{p}{,} \PY{n}{T2}\PY{p}{)}
\PY{n}{m\PYZus{}fit}\PY{p}{,} \PY{n}{q\PYZus{}fit} \PY{o}{=} \PY{n}{popt}
\PY{n}{sigma\PYZus{}m\PYZus{}fit}\PY{p}{,} \PY{n}{sigma\PYZus{}q\PYZus{}fit} \PY{o}{=} \PY{n}{np}\PY{o}{.}\PY{n}{sqrt}\PY{p}{(}\PY{n}{pcov}\PY{o}{.}\PY{n}{diagonal}\PY{p}{(}\PY{p}{)}\PY{p}{)}
\PY{n+nb}{print}\PY{p}{(}\PY{n}{m\PYZus{}fit}\PY{p}{,} \PY{n}{sigma\PYZus{}m\PYZus{}fit}\PY{p}{,} \PY{n}{q\PYZus{}fit}\PY{p}{,} \PY{n}{sigma\PYZus{}q\PYZus{}fit}\PY{p}{)}
\PY{n}{x} \PY{o}{=} \PY{n}{np}\PY{o}{.}\PY{n}{linspace}\PY{p}{(}\PY{l+m+mf}{0.}\PY{p}{,} \PY{l+m+mf}{60.}\PY{p}{,} \PY{l+m+mi}{100}\PY{p}{)}
\PY{n}{plt}\PY{o}{.}\PY{n}{plot}\PY{p}{(}\PY{n}{x}\PY{p}{,} \PY{n}{line}\PY{p}{(}\PY{n}{x}\PY{p}{,} \PY{n}{m\PYZus{}fit}\PY{p}{,} \PY{n}{q\PYZus{}fit}\PY{p}{)}\PY{p}{)}
\PY{n}{plt}\PY{o}{.}\PY{n}{xlabel}\PY{p}{(}\PY{l+s+s1}{\PYZsq{}}\PY{l+s+s1}{Massa [g]}\PY{l+s+s1}{\PYZsq{}}\PY{p}{)}
\PY{n}{plt}\PY{o}{.}\PY{n}{ylabel}\PY{p}{(}\PY{l+s+s1}{\PYZsq{}}\PY{l+s+s1}{Periodo al quadrato [s\PYZdl{}\PYZca{}2\PYZdl{}]}\PY{l+s+s1}{\PYZsq{}}\PY{p}{)}
\PY{n}{plt}\PY{o}{.}\PY{n}{grid}\PY{p}{(}\PY{n}{ls}\PY{o}{=}\PY{l+s+s1}{\PYZsq{}}\PY{l+s+s1}{dashed}\PY{l+s+s1}{\PYZsq{}}\PY{p}{)}

\PY{n}{plt}\PY{o}{.}\PY{n}{show}\PY{p}{(}\PY{p}{)}
\end{Verbatim}


\end{article}
\end{document}
