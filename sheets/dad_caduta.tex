\documentclass{lab1-article}

\title{Caduta di un grave}

\usepackage{fancyvrb}
\usepackage{hyperref}
\input{code/python}

\begin{document}


\begin{article}
\selectlanguage{italian}

\maketitle

\secintro


Lo scopo dell'esperienza \`e lo studio della legge oraria di un corpo in caduta
libera analizzando un filmato registrato con lo \emph{smartphone}.

\secmaterialsdad

\begin{itemize}
\item Un grave\footnote{La cosa \`e probabilmente ovvia, ma scegliete un oggetto
  che: (i) non metta a repentaglio in alcun modo la vostra incolumit\`a;
  (ii) non rischi di danneggiare il pavimento o l'arredo;
  (iii) non rischi di rompersi cadendo.}.
\item Metro a nastro.
\item Smartphone o macchina fotografica digitale.
\end{itemize}


\secmeasurements

Misureremo, a tempi fissati, le posizioni dell'oggetto durante la caduta
analizzando i singoli fotogrammi del filmato realizzato con il nostro
\emph{smartphone}.

\begin{figure}[!htb]
  \includegraphics[width=\linewidth]{figures/caduta}
\end{figure}


Per fissare le idee, il tempo di caduta di un oggetto da un'altezza iniziale
$h_0 = 2$~m \`e
\begin{align*}
  t_0 = \sqrt{\frac{2h_0}{g}} \sim 0.64~\text{s},
\end{align*}
che, alla frequenza tipica di $30$~fotogrammi al secondo, ci permette di
campionare la caduta in almeno una ventina di punti.


\labsubsection{Analisi del file video}

Per analizzare il file video avrete bisogno di un \emph{editor} che vi permetta
di navigare i fotogrammi. Se non avere particolari preferenze,
\href{https://shotcut.org/}{Shotcut} \`e un \emph{editor open-source} e
multi-piattaforma che fa al caso vostro. (Ma la cosa \`e talmente semplice che
qualsiasi programa va bene.)

Il modo pi\`u semplice per misurare l'altezza dell'oggetto fotogramma
per fotogramma \`e quello di avere una guida centimetrata sullo sfondo,
come mostrato in figura.


\labsubsection{Fit della legge oraria}

Realizzate un grafico dell'altezza $h(t_i)$ in funzione dei tempi $t_i$ dei
fotogrammi interessanti (i.e., quelli in cui effettivamente l'oggetto sta
cadendo). Fittate il grafico di dispersione con una parabola
\begin{align}
  h(t) = \frac{1}{2}at^2 + v_0 t + h_0
\end{align}
confrontate il valore di \emph{best-fit} per l'accelerazione $a$ con
l'accelerazione di gravit\`a $g$.

Fate un grafico dei residui per valutare qualitativamente la bont\`a del modello.


\secconsiderations

Come vedete dalla figura, un fotogramma non \`e un immagine \emph{istantanea},
ma un integrale su un intervallo di tempo piccolo ma finito, per cui \`e
perfettamente normale che gli oggetti in movimento possano apparire sfuocati.
Nel nostro caso particolare, la cosa pu\`o avere impatto sulla stima dell'incertezza
di misura sulla posizione---che potenzialmente \`e maggiore quanto pi\`u velocemente
si muove l'oggetto.

Fate bene attenzione a come sono indicati i tempi dei frame nel programma
di \emph{editing} video che utilizzate per l'analisi: secondi con parte decimale,
indice del fotogramma oppure un misto? (Nella figura, ad esempio,
\texttt{00:00:01:29} significa: $0$ ore, $0$ minuti, $1$ secondo e $29$ fotogrammi.)
Va da s\'e che per passare da fotogrammi a frazione di secondo dovete dividere
per il \emph{frame rate} (tipicamente 30 fps), che \`e quindi importante sapere.

\onecolumn

\documentclass{lab1-article}

\title{Caduta di un grave}

\usepackage{fancyvrb}
\usepackage{hyperref}
\input{code/python}

\begin{document}


\begin{article}
\selectlanguage{italian}

\maketitle

\secintro


Lo scopo dell'esperienza \`e lo studio della legge oraria di un corpo in caduta
libera analizzando un filmato registrato con lo \emph{smartphone}.

\secmaterialsdad

\begin{itemize}
\item Un grave\footnote{La cosa \`e probabilmente ovvia, ma scegliete un oggetto
  che: (i) non metta a repentaglio in alcun modo la vostra incolumit\`a;
  (ii) non rischi di danneggiare il pavimento o l'arredo;
  (iii) non rischi di rompersi cadendo.}.
\item Metro a nastro.
\item Smartphone o macchina fotografica digitale.
\end{itemize}


\secmeasurements

Misureremo, a tempi fissati, le posizioni dell'oggetto durante la caduta
analizzando i singoli fotogrammi del filmato realizzato con il nostro
\emph{smartphone}.

\begin{figure}[!htb]
  \includegraphics[width=\linewidth]{figures/caduta}
\end{figure}


Per fissare le idee, il tempo di caduta di un oggetto da un'altezza iniziale
$h_0 = 2$~m \`e
\begin{align*}
  t_0 = \sqrt{\frac{2h_0}{g}} \sim 0.64~\text{s},
\end{align*}
che, alla frequenza tipica di $30$~fotogrammi al secondo, ci permette di
campionare la caduta in almeno una ventina di punti.


\labsubsection{Analisi del file video}

Per analizzare il file video avrete bisogno di un \emph{editor} che vi permetta
di navigare i fotogrammi. Se non avere particolari preferenze,
\href{https://shotcut.org/}{Shotcut} \`e un \emph{editor open-source} e
multi-piattaforma che fa al caso vostro. (Ma la cosa \`e talmente semplice che
qualsiasi programa va bene.)

Il modo pi\`u semplice per misurare l'altezza dell'oggetto fotogramma
per fotogramma \`e quello di avere una guida centimetrata sullo sfondo,
come mostrato in figura.


\labsubsection{Fit della legge oraria}

Realizzate un grafico dell'altezza $h(t_i)$ in funzione dei tempi $t_i$ dei
fotogrammi interessanti (i.e., quelli in cui effettivamente l'oggetto sta
cadendo). Fittate il grafico di dispersione con una parabola
\begin{align}
  h(t) = \frac{1}{2}at^2 + v_0 t + h_0
\end{align}
confrontate il valore di \emph{best-fit} per l'accelerazione $a$ con
l'accelerazione di gravit\`a $g$.

Fate un grafico dei residui per valutare qualitativamente la bont\`a del modello.


\secconsiderations

Come vedete dalla figura, un fotogramma non \`e un immagine \emph{istantanea},
ma un integrale su un intervallo di tempo piccolo ma finito, per cui \`e
perfettamente normale che gli oggetti in movimento possano apparire sfuocati.
Nel nostro caso particolare, la cosa pu\`o avere impatto sulla stima dell'incertezza
di misura sulla posizione---che potenzialmente \`e maggiore quanto pi\`u velocemente
si muove l'oggetto.

Fate bene attenzione a come sono indicati i tempi dei frame nel programma
di \emph{editing} video che utilizzate per l'analisi: secondi con parte decimale,
indice del fotogramma oppure un misto? (Nella figura, ad esempio,
\texttt{00:00:01:29} significa: $0$ ore, $0$ minuti, $1$ secondo e $29$ fotogrammi.)
Va da s\'e che per passare da fotogrammi a frazione di secondo dovete dividere
per il \emph{frame rate} (tipicamente 30 fps), che \`e quindi importante sapere.

\onecolumn

\documentclass{lab1-article}

\title{Caduta di un grave}

\usepackage{fancyvrb}
\usepackage{hyperref}
\input{code/python}

\begin{document}


\begin{article}
\selectlanguage{italian}

\maketitle

\secintro


Lo scopo dell'esperienza \`e lo studio della legge oraria di un corpo in caduta
libera analizzando un filmato registrato con lo \emph{smartphone}.

\secmaterialsdad

\begin{itemize}
\item Un grave\footnote{La cosa \`e probabilmente ovvia, ma scegliete un oggetto
  che: (i) non metta a repentaglio in alcun modo la vostra incolumit\`a;
  (ii) non rischi di danneggiare il pavimento o l'arredo;
  (iii) non rischi di rompersi cadendo.}.
\item Metro a nastro.
\item Smartphone o macchina fotografica digitale.
\end{itemize}


\secmeasurements

Misureremo, a tempi fissati, le posizioni dell'oggetto durante la caduta
analizzando i singoli fotogrammi del filmato realizzato con il nostro
\emph{smartphone}.

\begin{figure}[!htb]
  \includegraphics[width=\linewidth]{figures/caduta}
\end{figure}


Per fissare le idee, il tempo di caduta di un oggetto da un'altezza iniziale
$h_0 = 2$~m \`e
\begin{align*}
  t_0 = \sqrt{\frac{2h_0}{g}} \sim 0.64~\text{s},
\end{align*}
che, alla frequenza tipica di $30$~fotogrammi al secondo, ci permette di
campionare la caduta in almeno una ventina di punti.


\labsubsection{Analisi del file video}

Per analizzare il file video avrete bisogno di un \emph{editor} che vi permetta
di navigare i fotogrammi. Se non avere particolari preferenze,
\href{https://shotcut.org/}{Shotcut} \`e un \emph{editor open-source} e
multi-piattaforma che fa al caso vostro. (Ma la cosa \`e talmente semplice che
qualsiasi programa va bene.)

Il modo pi\`u semplice per misurare l'altezza dell'oggetto fotogramma
per fotogramma \`e quello di avere una guida centimetrata sullo sfondo,
come mostrato in figura.


\labsubsection{Fit della legge oraria}

Realizzate un grafico dell'altezza $h(t_i)$ in funzione dei tempi $t_i$ dei
fotogrammi interessanti (i.e., quelli in cui effettivamente l'oggetto sta
cadendo). Fittate il grafico di dispersione con una parabola
\begin{align}
  h(t) = \frac{1}{2}at^2 + v_0 t + h_0
\end{align}
confrontate il valore di \emph{best-fit} per l'accelerazione $a$ con
l'accelerazione di gravit\`a $g$.

Fate un grafico dei residui per valutare qualitativamente la bont\`a del modello.


\secconsiderations

Come vedete dalla figura, un fotogramma non \`e un immagine \emph{istantanea},
ma un integrale su un intervallo di tempo piccolo ma finito, per cui \`e
perfettamente normale che gli oggetti in movimento possano apparire sfuocati.
Nel nostro caso particolare, la cosa pu\`o avere impatto sulla stima dell'incertezza
di misura sulla posizione---che potenzialmente \`e maggiore quanto pi\`u velocemente
si muove l'oggetto.

Fate bene attenzione a come sono indicati i tempi dei frame nel programma
di \emph{editing} video che utilizzate per l'analisi: secondi con parte decimale,
indice del fotogramma oppure un misto? (Nella figura, ad esempio,
\texttt{00:00:01:29} significa: $0$ ore, $0$ minuti, $1$ secondo e $29$ fotogrammi.)
Va da s\'e che per passare da fotogrammi a frazione di secondo dovete dividere
per il \emph{frame rate} (tipicamente 30 fps), che \`e quindi importante sapere.

\onecolumn

\documentclass{lab1-article}

\title{Caduta di un grave}

\usepackage{fancyvrb}
\usepackage{hyperref}
\input{code/python}

\begin{document}


\begin{article}
\selectlanguage{italian}

\maketitle

\secintro


Lo scopo dell'esperienza \`e lo studio della legge oraria di un corpo in caduta
libera analizzando un filmato registrato con lo \emph{smartphone}.

\secmaterialsdad

\begin{itemize}
\item Un grave\footnote{La cosa \`e probabilmente ovvia, ma scegliete un oggetto
  che: (i) non metta a repentaglio in alcun modo la vostra incolumit\`a;
  (ii) non rischi di danneggiare il pavimento o l'arredo;
  (iii) non rischi di rompersi cadendo.}.
\item Metro a nastro.
\item Smartphone o macchina fotografica digitale.
\end{itemize}


\secmeasurements

Misureremo, a tempi fissati, le posizioni dell'oggetto durante la caduta
analizzando i singoli fotogrammi del filmato realizzato con il nostro
\emph{smartphone}.

\begin{figure}[!htb]
  \includegraphics[width=\linewidth]{figures/caduta}
\end{figure}


Per fissare le idee, il tempo di caduta di un oggetto da un'altezza iniziale
$h_0 = 2$~m \`e
\begin{align*}
  t_0 = \sqrt{\frac{2h_0}{g}} \sim 0.64~\text{s},
\end{align*}
che, alla frequenza tipica di $30$~fotogrammi al secondo, ci permette di
campionare la caduta in almeno una ventina di punti.


\labsubsection{Analisi del file video}

Per analizzare il file video avrete bisogno di un \emph{editor} che vi permetta
di navigare i fotogrammi. Se non avere particolari preferenze,
\href{https://shotcut.org/}{Shotcut} \`e un \emph{editor open-source} e
multi-piattaforma che fa al caso vostro. (Ma la cosa \`e talmente semplice che
qualsiasi programa va bene.)

Il modo pi\`u semplice per misurare l'altezza dell'oggetto fotogramma
per fotogramma \`e quello di avere una guida centimetrata sullo sfondo,
come mostrato in figura.


\labsubsection{Fit della legge oraria}

Realizzate un grafico dell'altezza $h(t_i)$ in funzione dei tempi $t_i$ dei
fotogrammi interessanti (i.e., quelli in cui effettivamente l'oggetto sta
cadendo). Fittate il grafico di dispersione con una parabola
\begin{align}
  h(t) = \frac{1}{2}at^2 + v_0 t + h_0
\end{align}
confrontate il valore di \emph{best-fit} per l'accelerazione $a$ con
l'accelerazione di gravit\`a $g$.

Fate un grafico dei residui per valutare qualitativamente la bont\`a del modello.


\secconsiderations

Come vedete dalla figura, un fotogramma non \`e un immagine \emph{istantanea},
ma un integrale su un intervallo di tempo piccolo ma finito, per cui \`e
perfettamente normale che gli oggetti in movimento possano apparire sfuocati.
Nel nostro caso particolare, la cosa pu\`o avere impatto sulla stima dell'incertezza
di misura sulla posizione---che potenzialmente \`e maggiore quanto pi\`u velocemente
si muove l'oggetto.

Fate bene attenzione a come sono indicati i tempi dei frame nel programma
di \emph{editing} video che utilizzate per l'analisi: secondi con parte decimale,
indice del fotogramma oppure un misto? (Nella figura, ad esempio,
\texttt{00:00:01:29} significa: $0$ ore, $0$ minuti, $1$ secondo e $29$ fotogrammi.)
Va da s\'e che per passare da fotogrammi a frazione di secondo dovete dividere
per il \emph{frame rate} (tipicamente 30 fps), che \`e quindi importante sapere.

\onecolumn

\input{code/dad_caduta}


\end{article}
\end{document}



\end{article}
\end{document}



\end{article}
\end{document}



\end{article}
\end{document}
