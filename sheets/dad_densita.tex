\documentclass{lab1-article}

\title{Misure di densit\`a}

\usepackage{fancyvrb}
\usepackage{hyperref}
\input{code/python}

\begin{document}


\begin{article}
\selectlanguage{italian}

\maketitle

\secintro
Una quantit\`a fissata di qualunque sostanza o materiale occupa un volume che
varia soltanto se variano le condizioni in cui tale sostanza o materiale si
trova (ad esempio se dovesse passare dallo stato solido allo stato liquido, o se
cambiano la temperatura o la pressione). La massa per unit\`a di volume \`e nota
come densit\`a:
\begin{align}
  \rho = \frac{m}{V}, \quad [\rho]= \mathrm{kg/m}^3.
\end{align}

Lo scopo dell'esperienza \`e la stima della densit\`a dell'acqua (come misura di
controllo) e quella di almeno un altro materiale (solido, liquido o, perch\'e
no? gassoso) a scelta.

\secmaterialsdad

\begin{itemize}
\item Righello e/o metro a nastro e/o calibro.
\item Bilancia da cucina.
\item Un recipiente cilindrico trasparente.
\end{itemize}


\secmeasurements

\labsubsection{Misura della densit\`a dell'acqua}

Eseguiremo la misura della densit\`a dell'acqua utilizzando un cilindro
graduato, che si pu\`o realizzare, banalmente, con una bottiglia di plastica
trasparente ed un pennarello---magari apponendo una striscia verticale di nastro
di carta su un lato per scrivere pi\`u facilmente. Se misuriamo il diametro
\emph{interno} $d$ del cilindro e l'altezza $h$ della colonna d'acqua
(relativamente ad una altezza di riferimento arbitraria), possiamo stimare
il volume come
\begin{align}
  \hat{V} = \frac{\pi}{4} d^2h \quad \text{e} \quad
  \frac{\sigma^2_V}{V^2} = 4\frac{\sigma^2_d}{\hat{d}^2} + \frac{\sigma^2_h}{\hat{h}^2}.
\end{align}
La massa si misura direttamente con la bilancia, avendo cura di sottrarre la
\emph{tara}, corrispondente al livello d'acqua di riferimento. Se ripetiamo il
procedimento per un certo numero (e.g., 10) di valori diversi di $h$, possiamo
stimare la densit\`a dell'acqua come il coefficiente angolare della retta di
\emph{best-fit} del grafico massa vs. volume.


\labsubsection{Confronto con il valore tabulato}

Confrontate la vostra misura con il valore tabulato che pi\`u si adatta al
vostro \emph{setup} di misura. All'ordine zero la densit\`a dell'acqua \`e
$\sim 1$~g~cm$^{-3}$, ma sar\`a sufficiente una ricerca superficiale per
convincersi che essa dipende dalla temperatura e dalla composizione chimica
delle sostanze disciolte. Riflettete: tutto questo \`e rilevante oppure no ai
fini del confronto in questione? Se provate a misurare la densit\`a dell'acqua
a $\sim 4^\circ$~C (e.g., dopo averla tenuta in frigo per qualche ora) riuscite
ad apprezzare la differenza?


\labsubsection{Misura della densit\`a di un altro materiale}

Una volta che avete sotto controllo la misura della densit\`a dell'acqua,
misurate la densit\`a di un altro materiale (o di altri materiali, se siete
particolarmente motivati) a vostra scelta, tra le cose che avete a disposizione
in casa.

Il procedimento che avete utilizzato per la prima misura si generalizza
banalmente al caso di altri liquidi (e.g., olio, vino, alcool) o di solidi in
polvere (e.g., farina, zucchero, sale). Ma sentitevi liberi di essere creativi:
se avete a disposizione solidi omogeneei dello stesso materiale ma di forme e
dimensioni varie potete stimarne la densit\`a misurando direttamente le
dimensioni rilevanti con il righello e/o il metro a nastro (se la forma \`e
regolare) oppure immergendo i campioni nel vostro cilindro graduato.

Prima di cominciare cercate di fare una breve ricerca in letteratura per avere
un riferimento con cui confrontare l'esito della vostra misure. Scoprirete che
il web \`e pieno di spunti interessanti, come ad esempio
\url{https://www.sciencedirect.com/science/article/abs/pii/0260877494P1608Z}.


\secconsiderations

\emph{Non utilizzate strumenti pericolosi, o con i quali non vi sentite
  perfettamente a vostro agio, nella realizzazione dell'esperienza. Non \`e
  necessario per la buona riuscita dell'eseperienza stessa!}

Assicuratevi che il vostro cilindro graduato sia effettivamente \emph{cilindrico},
perch\'e altrimenti la stima del volume diventa significativamente pi\`u
complicata. (Non sar\`a difficile trovare in commercio una bottiglia, di plastica
o di vetro, che faccia al caso vostro.)

\`E estremamente probabile che il fondo del vostro cilindro non sia perfettamente
piatto. Potete ovviare facilmente all'inconveniente versando qualche cm di
acqua e prendendo da l\`i il riferimento---sia per la misura di $h$ che per
quella di $m$ (ovverosia includendo la massa di questa quantit\`a iniziale di
acqua nella tara da sottrarre).

Quando leggete l'altezza $h$ dalla vostra scala graduata fate attenzione al
fatto che in generale, a causa della tensione superficiale, la superficie
dell'acqua assume la forma di un menisco. Prestate perci\`o attenzione alla stima
dell'incertezza $\sigma_h$ da associare ad $h$---come del resto dovete fare con
tutte le altre grandezze in gioco.

Nella stesura della relazione siete caldamente incoraggiati ad essere prodighi
di dettagli sul \emph{setup} sperimentale che avete usato (e.g., foto) e a
mettere in evidenza le difficolt\`a che avete incontrato nella misura
(e gli accorgimenti che avete utilizzato per superarle).

\onecolumn

\documentclass{lab1-article}

\title{Misure di densit\`a}

\usepackage{fancyvrb}
\usepackage{hyperref}
\input{code/python}

\begin{document}


\begin{article}
\selectlanguage{italian}

\maketitle

\secintro
Una quantit\`a fissata di qualunque sostanza o materiale occupa un volume che
varia soltanto se variano le condizioni in cui tale sostanza o materiale si
trova (ad esempio se dovesse passare dallo stato solido allo stato liquido, o se
cambiano la temperatura o la pressione). La massa per unit\`a di volume \`e nota
come densit\`a:
\begin{align}
  \rho = \frac{m}{V}, \quad [\rho]= \mathrm{kg/m}^3.
\end{align}

Lo scopo dell'esperienza \`e la stima della densit\`a dell'acqua (come misura di
controllo) e quella di almeno un altro materiale (solido, liquido o, perch\'e
no? gassoso) a scelta.

\secmaterialsdad

\begin{itemize}
\item Righello e/o metro a nastro e/o calibro.
\item Bilancia da cucina.
\item Un recipiente cilindrico trasparente.
\end{itemize}


\secmeasurements

\labsubsection{Misura della densit\`a dell'acqua}

Eseguiremo la misura della densit\`a dell'acqua utilizzando un cilindro
graduato, che si pu\`o realizzare, banalmente, con una bottiglia di plastica
trasparente ed un pennarello---magari apponendo una striscia verticale di nastro
di carta su un lato per scrivere pi\`u facilmente. Se misuriamo il diametro
\emph{interno} $d$ del cilindro e l'altezza $h$ della colonna d'acqua
(relativamente ad una altezza di riferimento arbitraria), possiamo stimare
il volume come
\begin{align}
  \hat{V} = \frac{\pi}{4} d^2h \quad \text{e} \quad
  \frac{\sigma^2_V}{V^2} = 4\frac{\sigma^2_d}{\hat{d}^2} + \frac{\sigma^2_h}{\hat{h}^2}.
\end{align}
La massa si misura direttamente con la bilancia, avendo cura di sottrarre la
\emph{tara}, corrispondente al livello d'acqua di riferimento. Se ripetiamo il
procedimento per un certo numero (e.g., 10) di valori diversi di $h$, possiamo
stimare la densit\`a dell'acqua come il coefficiente angolare della retta di
\emph{best-fit} del grafico massa vs. volume.


\labsubsection{Confronto con il valore tabulato}

Confrontate la vostra misura con il valore tabulato che pi\`u si adatta al
vostro \emph{setup} di misura. All'ordine zero la densit\`a dell'acqua \`e
$\sim 1$~g~cm$^{-3}$, ma sar\`a sufficiente una ricerca superficiale per
convincersi che essa dipende dalla temperatura e dalla composizione chimica
delle sostanze disciolte. Riflettete: tutto questo \`e rilevante oppure no ai
fini del confronto in questione? Se provate a misurare la densit\`a dell'acqua
a $\sim 4^\circ$~C (e.g., dopo averla tenuta in frigo per qualche ora) riuscite
ad apprezzare la differenza?


\labsubsection{Misura della densit\`a di un altro materiale}

Una volta che avete sotto controllo la misura della densit\`a dell'acqua,
misurate la densit\`a di un altro materiale (o di altri materiali, se siete
particolarmente motivati) a vostra scelta, tra le cose che avete a disposizione
in casa.

Il procedimento che avete utilizzato per la prima misura si generalizza
banalmente al caso di altri liquidi (e.g., olio, vino, alcool) o di solidi in
polvere (e.g., farina, zucchero, sale). Ma sentitevi liberi di essere creativi:
se avete a disposizione solidi omogeneei dello stesso materiale ma di forme e
dimensioni varie potete stimarne la densit\`a misurando direttamente le
dimensioni rilevanti con il righello e/o il metro a nastro (se la forma \`e
regolare) oppure immergendo i campioni nel vostro cilindro graduato.

Prima di cominciare cercate di fare una breve ricerca in letteratura per avere
un riferimento con cui confrontare l'esito della vostra misure. Scoprirete che
il web \`e pieno di spunti interessanti, come ad esempio
\url{https://www.sciencedirect.com/science/article/abs/pii/0260877494P1608Z}.


\secconsiderations

\emph{Non utilizzate strumenti pericolosi, o con i quali non vi sentite
  perfettamente a vostro agio, nella realizzazione dell'esperienza. Non \`e
  necessario per la buona riuscita dell'eseperienza stessa!}

Assicuratevi che il vostro cilindro graduato sia effettivamente \emph{cilindrico},
perch\'e altrimenti la stima del volume diventa significativamente pi\`u
complicata. (Non sar\`a difficile trovare in commercio una bottiglia, di plastica
o di vetro, che faccia al caso vostro.)

\`E estremamente probabile che il fondo del vostro cilindro non sia perfettamente
piatto. Potete ovviare facilmente all'inconveniente versando qualche cm di
acqua e prendendo da l\`i il riferimento---sia per la misura di $h$ che per
quella di $m$ (ovverosia includendo la massa di questa quantit\`a iniziale di
acqua nella tara da sottrarre).

Quando leggete l'altezza $h$ dalla vostra scala graduata fate attenzione al
fatto che in generale, a causa della tensione superficiale, la superficie
dell'acqua assume la forma di un menisco. Prestate perci\`o attenzione alla stima
dell'incertezza $\sigma_h$ da associare ad $h$---come del resto dovete fare con
tutte le altre grandezze in gioco.

Nella stesura della relazione siete caldamente incoraggiati ad essere prodighi
di dettagli sul \emph{setup} sperimentale che avete usato (e.g., foto) e a
mettere in evidenza le difficolt\`a che avete incontrato nella misura
(e gli accorgimenti che avete utilizzato per superarle).

\onecolumn

\documentclass{lab1-article}

\title{Misure di densit\`a}

\usepackage{fancyvrb}
\usepackage{hyperref}
\input{code/python}

\begin{document}


\begin{article}
\selectlanguage{italian}

\maketitle

\secintro
Una quantit\`a fissata di qualunque sostanza o materiale occupa un volume che
varia soltanto se variano le condizioni in cui tale sostanza o materiale si
trova (ad esempio se dovesse passare dallo stato solido allo stato liquido, o se
cambiano la temperatura o la pressione). La massa per unit\`a di volume \`e nota
come densit\`a:
\begin{align}
  \rho = \frac{m}{V}, \quad [\rho]= \mathrm{kg/m}^3.
\end{align}

Lo scopo dell'esperienza \`e la stima della densit\`a dell'acqua (come misura di
controllo) e quella di almeno un altro materiale (solido, liquido o, perch\'e
no? gassoso) a scelta.

\secmaterialsdad

\begin{itemize}
\item Righello e/o metro a nastro e/o calibro.
\item Bilancia da cucina.
\item Un recipiente cilindrico trasparente.
\end{itemize}


\secmeasurements

\labsubsection{Misura della densit\`a dell'acqua}

Eseguiremo la misura della densit\`a dell'acqua utilizzando un cilindro
graduato, che si pu\`o realizzare, banalmente, con una bottiglia di plastica
trasparente ed un pennarello---magari apponendo una striscia verticale di nastro
di carta su un lato per scrivere pi\`u facilmente. Se misuriamo il diametro
\emph{interno} $d$ del cilindro e l'altezza $h$ della colonna d'acqua
(relativamente ad una altezza di riferimento arbitraria), possiamo stimare
il volume come
\begin{align}
  \hat{V} = \frac{\pi}{4} d^2h \quad \text{e} \quad
  \frac{\sigma^2_V}{V^2} = 4\frac{\sigma^2_d}{\hat{d}^2} + \frac{\sigma^2_h}{\hat{h}^2}.
\end{align}
La massa si misura direttamente con la bilancia, avendo cura di sottrarre la
\emph{tara}, corrispondente al livello d'acqua di riferimento. Se ripetiamo il
procedimento per un certo numero (e.g., 10) di valori diversi di $h$, possiamo
stimare la densit\`a dell'acqua come il coefficiente angolare della retta di
\emph{best-fit} del grafico massa vs. volume.


\labsubsection{Confronto con il valore tabulato}

Confrontate la vostra misura con il valore tabulato che pi\`u si adatta al
vostro \emph{setup} di misura. All'ordine zero la densit\`a dell'acqua \`e
$\sim 1$~g~cm$^{-3}$, ma sar\`a sufficiente una ricerca superficiale per
convincersi che essa dipende dalla temperatura e dalla composizione chimica
delle sostanze disciolte. Riflettete: tutto questo \`e rilevante oppure no ai
fini del confronto in questione? Se provate a misurare la densit\`a dell'acqua
a $\sim 4^\circ$~C (e.g., dopo averla tenuta in frigo per qualche ora) riuscite
ad apprezzare la differenza?


\labsubsection{Misura della densit\`a di un altro materiale}

Una volta che avete sotto controllo la misura della densit\`a dell'acqua,
misurate la densit\`a di un altro materiale (o di altri materiali, se siete
particolarmente motivati) a vostra scelta, tra le cose che avete a disposizione
in casa.

Il procedimento che avete utilizzato per la prima misura si generalizza
banalmente al caso di altri liquidi (e.g., olio, vino, alcool) o di solidi in
polvere (e.g., farina, zucchero, sale). Ma sentitevi liberi di essere creativi:
se avete a disposizione solidi omogeneei dello stesso materiale ma di forme e
dimensioni varie potete stimarne la densit\`a misurando direttamente le
dimensioni rilevanti con il righello e/o il metro a nastro (se la forma \`e
regolare) oppure immergendo i campioni nel vostro cilindro graduato.

Prima di cominciare cercate di fare una breve ricerca in letteratura per avere
un riferimento con cui confrontare l'esito della vostra misure. Scoprirete che
il web \`e pieno di spunti interessanti, come ad esempio
\url{https://www.sciencedirect.com/science/article/abs/pii/0260877494P1608Z}.


\secconsiderations

\emph{Non utilizzate strumenti pericolosi, o con i quali non vi sentite
  perfettamente a vostro agio, nella realizzazione dell'esperienza. Non \`e
  necessario per la buona riuscita dell'eseperienza stessa!}

Assicuratevi che il vostro cilindro graduato sia effettivamente \emph{cilindrico},
perch\'e altrimenti la stima del volume diventa significativamente pi\`u
complicata. (Non sar\`a difficile trovare in commercio una bottiglia, di plastica
o di vetro, che faccia al caso vostro.)

\`E estremamente probabile che il fondo del vostro cilindro non sia perfettamente
piatto. Potete ovviare facilmente all'inconveniente versando qualche cm di
acqua e prendendo da l\`i il riferimento---sia per la misura di $h$ che per
quella di $m$ (ovverosia includendo la massa di questa quantit\`a iniziale di
acqua nella tara da sottrarre).

Quando leggete l'altezza $h$ dalla vostra scala graduata fate attenzione al
fatto che in generale, a causa della tensione superficiale, la superficie
dell'acqua assume la forma di un menisco. Prestate perci\`o attenzione alla stima
dell'incertezza $\sigma_h$ da associare ad $h$---come del resto dovete fare con
tutte le altre grandezze in gioco.

Nella stesura della relazione siete caldamente incoraggiati ad essere prodighi
di dettagli sul \emph{setup} sperimentale che avete usato (e.g., foto) e a
mettere in evidenza le difficolt\`a che avete incontrato nella misura
(e gli accorgimenti che avete utilizzato per superarle).

\onecolumn

\documentclass{lab1-article}

\title{Misure di densit\`a}

\usepackage{fancyvrb}
\usepackage{hyperref}
\input{code/python}

\begin{document}


\begin{article}
\selectlanguage{italian}

\maketitle

\secintro
Una quantit\`a fissata di qualunque sostanza o materiale occupa un volume che
varia soltanto se variano le condizioni in cui tale sostanza o materiale si
trova (ad esempio se dovesse passare dallo stato solido allo stato liquido, o se
cambiano la temperatura o la pressione). La massa per unit\`a di volume \`e nota
come densit\`a:
\begin{align}
  \rho = \frac{m}{V}, \quad [\rho]= \mathrm{kg/m}^3.
\end{align}

Lo scopo dell'esperienza \`e la stima della densit\`a dell'acqua (come misura di
controllo) e quella di almeno un altro materiale (solido, liquido o, perch\'e
no? gassoso) a scelta.

\secmaterialsdad

\begin{itemize}
\item Righello e/o metro a nastro e/o calibro.
\item Bilancia da cucina.
\item Un recipiente cilindrico trasparente.
\end{itemize}


\secmeasurements

\labsubsection{Misura della densit\`a dell'acqua}

Eseguiremo la misura della densit\`a dell'acqua utilizzando un cilindro
graduato, che si pu\`o realizzare, banalmente, con una bottiglia di plastica
trasparente ed un pennarello---magari apponendo una striscia verticale di nastro
di carta su un lato per scrivere pi\`u facilmente. Se misuriamo il diametro
\emph{interno} $d$ del cilindro e l'altezza $h$ della colonna d'acqua
(relativamente ad una altezza di riferimento arbitraria), possiamo stimare
il volume come
\begin{align}
  \hat{V} = \frac{\pi}{4} d^2h \quad \text{e} \quad
  \frac{\sigma^2_V}{V^2} = 4\frac{\sigma^2_d}{\hat{d}^2} + \frac{\sigma^2_h}{\hat{h}^2}.
\end{align}
La massa si misura direttamente con la bilancia, avendo cura di sottrarre la
\emph{tara}, corrispondente al livello d'acqua di riferimento. Se ripetiamo il
procedimento per un certo numero (e.g., 10) di valori diversi di $h$, possiamo
stimare la densit\`a dell'acqua come il coefficiente angolare della retta di
\emph{best-fit} del grafico massa vs. volume.


\labsubsection{Confronto con il valore tabulato}

Confrontate la vostra misura con il valore tabulato che pi\`u si adatta al
vostro \emph{setup} di misura. All'ordine zero la densit\`a dell'acqua \`e
$\sim 1$~g~cm$^{-3}$, ma sar\`a sufficiente una ricerca superficiale per
convincersi che essa dipende dalla temperatura e dalla composizione chimica
delle sostanze disciolte. Riflettete: tutto questo \`e rilevante oppure no ai
fini del confronto in questione? Se provate a misurare la densit\`a dell'acqua
a $\sim 4^\circ$~C (e.g., dopo averla tenuta in frigo per qualche ora) riuscite
ad apprezzare la differenza?


\labsubsection{Misura della densit\`a di un altro materiale}

Una volta che avete sotto controllo la misura della densit\`a dell'acqua,
misurate la densit\`a di un altro materiale (o di altri materiali, se siete
particolarmente motivati) a vostra scelta, tra le cose che avete a disposizione
in casa.

Il procedimento che avete utilizzato per la prima misura si generalizza
banalmente al caso di altri liquidi (e.g., olio, vino, alcool) o di solidi in
polvere (e.g., farina, zucchero, sale). Ma sentitevi liberi di essere creativi:
se avete a disposizione solidi omogeneei dello stesso materiale ma di forme e
dimensioni varie potete stimarne la densit\`a misurando direttamente le
dimensioni rilevanti con il righello e/o il metro a nastro (se la forma \`e
regolare) oppure immergendo i campioni nel vostro cilindro graduato.

Prima di cominciare cercate di fare una breve ricerca in letteratura per avere
un riferimento con cui confrontare l'esito della vostra misure. Scoprirete che
il web \`e pieno di spunti interessanti, come ad esempio
\url{https://www.sciencedirect.com/science/article/abs/pii/0260877494P1608Z}.


\secconsiderations

\emph{Non utilizzate strumenti pericolosi, o con i quali non vi sentite
  perfettamente a vostro agio, nella realizzazione dell'esperienza. Non \`e
  necessario per la buona riuscita dell'eseperienza stessa!}

Assicuratevi che il vostro cilindro graduato sia effettivamente \emph{cilindrico},
perch\'e altrimenti la stima del volume diventa significativamente pi\`u
complicata. (Non sar\`a difficile trovare in commercio una bottiglia, di plastica
o di vetro, che faccia al caso vostro.)

\`E estremamente probabile che il fondo del vostro cilindro non sia perfettamente
piatto. Potete ovviare facilmente all'inconveniente versando qualche cm di
acqua e prendendo da l\`i il riferimento---sia per la misura di $h$ che per
quella di $m$ (ovverosia includendo la massa di questa quantit\`a iniziale di
acqua nella tara da sottrarre).

Quando leggete l'altezza $h$ dalla vostra scala graduata fate attenzione al
fatto che in generale, a causa della tensione superficiale, la superficie
dell'acqua assume la forma di un menisco. Prestate perci\`o attenzione alla stima
dell'incertezza $\sigma_h$ da associare ad $h$---come del resto dovete fare con
tutte le altre grandezze in gioco.

Nella stesura della relazione siete caldamente incoraggiati ad essere prodighi
di dettagli sul \emph{setup} sperimentale che avete usato (e.g., foto) e a
mettere in evidenza le difficolt\`a che avete incontrato nella misura
(e gli accorgimenti che avete utilizzato per superarle).

\onecolumn

\input{code/dad_densita}


\end{article}
\end{document}



\end{article}
\end{document}



\end{article}
\end{document}



\end{article}
\end{document}
