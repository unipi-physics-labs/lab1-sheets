\documentclass{lab1-article}

\title{Tavolo ad aria}

\newcommand{\plasduinodoctext}[1]%
{
  Una volta acceso il calcolatore, selezionare dal men\`u principale
  (in alto a sinistra) \menuitem{Application}~$\rightarrow$~%
  \menuitem{Education}~$\rightarrow$~\menuitem{plasduino}. Questo dovrebbe
  mostrare la finestra principale del programma di acquisizione. Per questa
  esperienza, tra la lista dei moduli, lanciate~\menuitem{#1} (doppio
  click sulla linea corrispondente, oppure selezionate la linea stessa e
  premete \menuitem{Open}).
}


\newcommand{\plasduinodoc}[1]%
{
  \labsection{Note sul programma di acquisizione}

  \plasduinodoctext{#1}
}

\newcommand{\plasduinosave}%
{
  Di norma al termine di ogni sessione di presa dati il programma vi chiede se
  volete salvare una copia del \emph{file} dei dati in una cartella a vostra
  scelta (il che pu\`o essere comodo per l'analisi successiva). Se questa
  funzionalit\`a dovesse essere disabilitata potete ri-abilitarla 
  attraverso il men\`u di plasduino \menuitem{Configuration}~%
  $\rightarrow$~\menuitem{Change settings}: nella finestra che si apre
  selezionate il tab~\menuitem{daq} e abilitate l'opzione
  \menuitem{prompt-save-dialog}.
}


\begin{document}


\begin{article}
\selectlanguage{italian}

\maketitle

\secsummary

Lo scopo dell'esperienza \`e di verificare la conservazione della quantit\`a
di moto nell'urto di due pedine su un tavolo ad aria.


\secmeasurements

Se la risultante delle forze esterne agenti sul sistema (costituito dalla due
pedine coinvolte nell'urto) \`e nulla, la quantit\`a di moto del sistema si
conserva. 
L'equazione vettoriale che esprime la legge di conservazione si pu\`o scomporre
in due equazioni scalari per le componenti sui due assi $x$ ed $y$
\begin{align}
  m_1 v^{\rm i}_{1x} + m_2 v^{\rm i}_{2x} &=
  m_1 v^{\rm f}_{1x} + m_2 v^{\rm f}_{2x} \nonumber\\
  m_1 v^{\rm i}_{1y} + m_2 v^{\rm i}_{2y} &= m_1 v^{\rm f}_{1y} + m_2 v^{\rm f}_{2y}
\end{align}
in cui gli indici numerici indicano le due pedine (non dimenticate di misurare
le masse) e le lettere ``i'' ed ``f'' indicano le quantit\`a iniziali (prima
dell'urto) e finali (dopo l'urto).
\emph{Il segno delle componenti della velocit\`a \`e fondamentale!}

Si possono studiare urti con una delle due pedine inizialmente ferma o con
entrambe le pedine in moto.


\labsubsection{Stima della velocit\`a}

Le componenti delle velocit\`a (medie) di ciascuna delle pedine si possono
stimare mediante \emph{fit} con una retta dell'andamento in funzione
del tempo della posizione (misurata) corrispondente, e.g. $v_{1x}$ si ricava dal
\emph{fit} di $x_1$ in funzione di $t$. Per stimare le velocit\`a prima e dopo
l'urto si deve avere cura di eseguire il \emph{fit} in un intervallo temporale
appropriato.

Il programma di analisi permette di \emph{calibrare} la misura della posizione
(cio\`e di ricavare il fattore di conversione tra \emph{pixel} ed unit\`a
fisiche) attraverso una lunghezza nota---anche se, in linea di principio, la
conservazione della quantit\`a di moto pu\`o essere verificata misurando le
velocit\`a in pixel/s.


\secconsiderations

\labsubsection{Strategia di misura}

L'urto tra le due pedine viene registrato da una videocamera controllata da un
programma di acquisizione che produce un file video della durata di
$\sim 5$~s. Si consiglia di cercare di registrare almeno 1--2 secondi di moto
\emph{libero} (senza urti) per ciascuna delle due pedine sia prima che dopo
l'urto, in modo da avere abbastanza punti su cui fare il \emph{fit} in fase di
analisi. Si consiglia anche di cercare di non coprire
le pedine con la mano durante la registrazione. 

\subsecdataformat

Il programma di analisi dei filmati fornisce (nel caso di 2 pedine) un
\emph{file} di uscita contenente 6 colonne che rappresentano, per ogni
fotogramma analizzato:
\begin{enumerate}
\item il numero del fotogramma;
\item il tempo $t$, dall'inizio del filmato;
\item le coordinate $x_1$ ed $y_1$ della prima pedina.
\item le coordinate $x_2$ ed $y_2$ della seconda pedina.
\end{enumerate}

Il sistema di riferimento ha origine nell'angolo in alto a sinistra
dell'immagine, con l'asse $x$ orientato verso destra e l'asse $y$ orientato
verso il basso.



\labsection{Note sul programma di acquisizione}

Il programma di acquisizione dei filmati si lancia attraverso il collegamento
al programma ``TAVOLO'' che si trova sul desktop. Quando non si trova in 
modalit\`a di registrazione il programma fornisce un video in tempo reale
dalla telecamera posta sopra il tavolo. In questa fase si possono impostare la
luminosit\`a, il numero di fotogrammi da acquisire e la velocit\`a (in
fotogrammi al secondo) di acquisizione---ma i parametri preimpostati dovrebbero
essere ragionevoli.

A questo punto si pu\`o avviare la registrazione del filmato premendo il tasto
``START''. Una volta terminata la registrazione, il programma chiede il nome
con cui salvare il file del filmato (in formato avi). \`E importante prendere
nota della cartella in cui il file viene salvato per ritrovarlo in seguito.
(Potete ignorare la finestrella che chiede se si vuole tagliare il filmato
producendo una copia ridotta: andate oltre.) Una volta che il filmato \`e stato
salvato (e si \`e premuto ``OK''), il sistema \`e pronto per una nuova
acquisizione.

I filmati vanno trasferiti, tramite \emph{memory stick} USB, su un altro pc
dove \`e installato il programma di analisi.


\labsection{Note sul programma di analisi}

\plasduinodoctext{Video Click}

Potete aprire i filmati che avete registrato dal men\`u
\menuitem{File}~$\rightarrow$~\menuitem{Open clip}.
Il filmato \`e disponibile per l'analisi sotto forma di fotogrammi su
cui le posizioni si misurano con un click sinistro del mouse.
Dopo un numero di click pari al numero di oggetti impostati 
-- 2 di default -- si porta automaticamente al fotogramma successivo, ma
potete sempre tornare indietro a correggere. Le misure si salvano da
\menuitem{File}~$\rightarrow$~\menuitem{Save data to file}.

Notate che non \`e necessario cliccare su \emph{tutti} i fotogrammi.
Nel file di uscita compariranno solo i dati relativi ai fotogrammi per cui
sono state identificate le posizioni di tutti gli oggetti.
\end{article}
\end{document}
