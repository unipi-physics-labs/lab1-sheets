\documentclass{lab1-article}

\title{Oscillazioni forzate}

\newcommand{\plasduinodoctext}[1]%
{
  Una volta acceso il calcolatore, selezionare dal men\`u principale
  (in alto a sinistra) \menuitem{Application}~$\rightarrow$~%
  \menuitem{Education}~$\rightarrow$~\menuitem{plasduino}. Questo dovrebbe
  mostrare la finestra principale del programma di acquisizione. Per questa
  esperienza, tra la lista dei moduli, lanciate~\menuitem{#1} (doppio
  click sulla linea corrispondente, oppure selezionate la linea stessa e
  premete \menuitem{Open}).
}


\newcommand{\plasduinodoc}[1]%
{
  \labsection{Note sul programma di acquisizione}

  \plasduinodoctext{#1}
}

\newcommand{\plasduinosave}%
{
  Di norma al termine di ogni sessione di presa dati il programma vi chiede se
  volete salvare una copia del \emph{file} dei dati in una cartella a vostra
  scelta (il che pu\`o essere comodo per l'analisi successiva). Se questa
  funzionalit\`a dovesse essere disabilitata potete ri-abilitarla 
  attraverso il men\`u di plasduino \menuitem{Configuration}~%
  $\rightarrow$~\menuitem{Change settings}: nella finestra che si apre
  selezionate il tab~\menuitem{daq} e abilitate l'opzione
  \menuitem{prompt-save-dialog}.
}


\begin{document}


\begin{article}
\selectlanguage{italian}

\maketitle

\secsummary

Lo scopo dell'esperienza \`e lo studio del fenomeno della risonanza.

\secmaterials

\begin{itemize}
\item Un pendolo dotato di smorzatore;
\item un motorino collegato al pendolo tramite due molle;
\item sistema di acquisizione per controllare il motorino e registrare la
  posizione del pendolo.
\end{itemize}


\secmeasurements

L'equazione differenziale che descrive il moto di un pendolo fisico
(in approssimazione di piccole oscillazioni) sollecitato
da una forza esterna variabile in modo sinusoidale nel tempo \`e
\begin{align}\label{eq:eq_moto}
  \frac{d^2\theta}{dt^2} + 2\gamma\frac{d\theta}{dt} + \omega_0^2\theta = 
  F_0 \cos(\omega t).
\end{align}
Nella~\eqref{eq:eq_moto} $\omega_0$ \`e la pulsazione angolare propria
(fissata) del pendolo, da non confondere con la pulsazione $\omega$ (variabile)
della forzante (cio\`e del motorino); $\gamma$ \`e proporzionale alla
forza di attrito che smorza il moto e $F_0$ \`e proporzionale alla massima
intensit\`a della forza esterna.

Detta $\omega_s$ \`e la pulsazione del moto armonico smorzato
\begin{align}
  \omega_s = \sqrt{\omega_0^2 - \gamma^2},
\end{align}
la soluzione della~\eqref{eq:eq_moto} \`e data dalla somma
di due termini---il primo smorzato esponenzialmente ed il secondo, alla
pulsazione della forzante, non smorzato:
\begin{align}
  \theta(t) = C_1 e^{-\gamma t} \cos( \omega_s t + \phi_1) + 
  C_2 \cos(\omega t + \phi_2).
\end{align}

Se aspettiamo sufficientemente a lungo il primo di questi due moti si smorza
completamente e si ha:
\begin{align}\label{eq:legge_oraria_asintotica}
  \theta(t) = C_2 \cos(\omega t + \phi_2), \quad t \gg \frac{1}{\gamma}.
\end{align}
Sostituendo la soluzione~\eqref{eq:legge_oraria_asintotica} nella
\eqref{eq:eq_moto} si pu\`o calcolare la costante $C_2$ come funzione di
$\omega$:
\begin{align}\label{eq:curva_risonanza}
  C_2(\omega) = \frac{F_0}{\sqrt{(\omega_0^2 - \omega^2)^2 + 4\gamma^2\omega^2}}.
\end{align}


\labsubsection{Misura dello smorzamento}

Si faccia oscillare il pendolo a motorino spento e si stimino (graficamente
o mediante un \emph{fit} analitico) i valori di $\omega_0$ e $\gamma$.
Per il \emph{fit} analitico si pu\`o utilizzare il modello
\begin{align}
  \theta(t) = C_0 + C_1 e^{-\gamma t} \cos( \omega_s t + \phi_1),
\end{align}
in cui il termine $C_0$ \`e necessario perch\'e il riferimento per il sistema
di acquisizione non coincide con la posizione di riposo dell'oscillatore.


\labsubsection{Curva di risonanza}

Si metta il motorino in rotazione e si misuri l'ampiezza delle oscillazioni per
varie pulsazioni $\omega$ del motorino stesso.
Si costruisca il grafico dell'ampiezza in funzione di $\omega$ e verifichi
quantitativamente l'accordo con la~\eqref{eq:curva_risonanza}.

\begin{figure}[htbp]
  \centering\includegraphics[width=0.92\linewidth]%
                            {figures/pendulumdrive_resonance}
  
  \caption{Esempio di curva di risonanza ottenuta con l'apparato sperimentale
    in uso per questa esperienza.}
\end{figure}

\secconsiderations

Fissata la velocit\`a del motorino, la pulsazione angolare e l'ampiezza di
oscillazione si possono stimare entrambe dal grafico (con il \emph{mouse} o
facendo un \emph{fit} analitico).

Per la curva di risonanza si consiglia di eseguire un buon numero di misure
(20-­30) allo scopo di campionare con sufficiente dettaglio la curva stessa,
in particolar modo in prossimit\`a del picco.

Per alcuni valori della velocit\`a del motorino il sistema di fatto non va
mai a regime. In questi casi non preoccupatevi troppo e cercate di selezionare
un periodo di $\sim 20$~s in cui l'ampiezza sia approssimativamente costante.


\plasduinodoc{Pendolum Drive}

\plasduinosave

\end{article}
\end{document}
