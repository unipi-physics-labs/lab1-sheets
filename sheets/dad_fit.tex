\documentclass{lab1-article}

\title{Fitting}

\usepackage{fancyvrb}
\usepackage{hyperref}
\input{code/python}

\begin{document}


\begin{article}
\selectlanguage{italian}

\maketitle

\secintro


Lo scopo dell'esperienza \`e la rianalisi di alcuni dei dati acquisiti durante
il primo semestre per mettere in pratica le nozioni di base sul \emph{fit} dei
minimi quadrati introdotte nel corso.

\secmaterialsdad

\begin{itemize}
\item Dati relativi alle esperienze del primo semestre;
\item calcolatore con ecosistema scientifico Python.
\end{itemize}


\secmeasurements

Recuperate i dati relativi alle esperienze del primo semestre, ed in particolare
quelle su:
\begin{enumerate}
  \item misure di densit\`a;
  \item equazione della catenaria;
  \item caduta di un grave;
  \item rimbalzi di una pallina.
\end{enumerate}
Per ciascuna di queste avete gi\`a eseguito un fit---al tempo senza prestare
attenzione alle incertezze di misura. Questa volta, viceversa, il \emph{focus}
dell'esperienza \`e proprio una corretta trattazione di queste ultime.

\labsubsection{Fit}

Per ciascuno dei data set a disposizione, detto $f(x; \theta_1 \ldots \theta_m)$
il modello utilizzato, verificate preliminarmente se gli errori sulla variabile
indipendente possono essere trascurati, ovverosia se la condizione
\begin{align}\label{eq:errorex}
  \left| \frac{df(x_i; \theta_1 \ldots \theta_m)}{dx} \right| \sigma_{x_i} \ll
  \sigma_{y_i} \quad \forall i = 1 \ldots n
\end{align}
\`e verificata o meno. Nel primo caso farete un fit dei minimi quadrati
propriamente detto, mentre in caso contrario utilizzerete gli
\emph{errori efficaci}:
\begin{align}\label{eq:errori_efficaci}
  \sigma_i^2 = \sigma_{y_i}^2 +
  \left( \frac{df(x_i; \theta_1 \ldots \theta_m)}{dx} \right)^2 \sigma_{x_i}^2
\end{align}
iterando il procedimento fino alla convergenza.

(Notate che, per verificare la condizione~\eqref{eq:errorex} avete bisogno di
una stima iniziale dei parametri del modello; a questo scopo potete utilizzare
i risultati di un fit dei minimi quadrati trascurando le incertezze sulla
variabile dipendente.)


\labsubsection{Test del $\chi^2$}

Per ciascuna delle esperienze eseguite un test del $\chi^2$ con i parametri di
\emph{best-fit} ottenuti. Confrontate il valore ottenuto con il numero di gradi
di libert\`a del problema e commentate sulla bont\`a del fit.

(Spunto di riflessione: le incertezze possono essere considerate distribuite
gaussianamente? E, se questo non fosse il caso, cambia qualcosa nella nostra
interpretazione del test del $\chi^2$?


\labsubsection{Grafico dei residui}

Per ciascuna delle esperienze costruite un grafico dei residui e
verificate la presenza o meno di indicazione di strutture che potrebbero
indicare effetti sistematici.

(Notate che le incertezze da utilizzare sul grafico dei residui sono quelle
efficaci~\eqref{eq:errori_efficaci}, di modo che il grafico stesso sia
visualmente in accordo con il risultato del test del $chi^2$.)


\secconsiderations

Nella stesura della relazione indicate chiaramente se ci sono differenze
significative, in termini dei valori di \emph{best-fit} dei parametri, con i
risultati ottenuti nel primo semestre. In caso affermativo cercate di spiegare
il motivo.


\end{article}
\end{document}
