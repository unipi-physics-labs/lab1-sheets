\documentclass{lab1-article}

\title{Periodo di un pendolo}

\usepackage{fancyvrb}
\usepackage{hyperref}
\input{code/python}

\begin{document}


\begin{article}
\selectlanguage{italian}

\maketitle

\secintro
Lo scopo dell'esperienza \`e la misura della dipendenza del periodo di un
pendolo dalla sua lunghezza e dall'ampiezza iniziale di oscillazione.


\secmaterialsdad

\begin{itemize}
\item Filo e pesetto
\item Righello o metro a nastro
\item Cronometro digitale o \emph{smartphone}
\end{itemize}


\secdefinitions

Un pendolo semplice \`e tipicamente schematizzato come una massa puntiforme
appesa ad un punto fisso attraverso un filo inestensibile di massa nulla.

\begin{figure}[htb!]
  \begin{tikzpicture}[scale=1]
    \pgfmathsetmacro{\xc}{2.25}
    \pgfmathsetmacro{\yc}{0}
    \pgfmathsetmacro{\r}{4.5}
    \pgfmathsetmacro{\rangle}{1}
    \pgfmathsetmacro{\thetazero}{40}
    \node at (0, 0) {};
    \draw[style=densely dashed] (\xc, \yc) -- (\xc, \yc - \r);
    \draw (\xc, \yc) -- (\xc + \r*sin{\thetazero}, \yc - \r*cos{\thetazero});
    \draw[style=densely dashed] (\xc, \yc - \r) arc (270:270+\thetazero:\r);
    \draw[style=densely dashed] (\xc, \yc - \r*cos{\thetazero}) --%
    (\xc + \r*sin{\thetazero}, \yc - \r*cos{\thetazero});
    \draw (\xc, \yc - \rangle) arc (270:270+\thetazero:\rangle);
    \fill (\xc + \r*sin{\thetazero}, \yc - \r*cos{\thetazero}) circle%
          [radius=0.15];
    \node at (\xc - 0.25, 0) {$0$};
    \node[anchor=east] at (\xc , \yc - 0.5*\r) {$l\cos\theta_0$};
    \node[anchor=south] at%
    (\xc + 0.5*\r*sin{\thetazero}, \yc - \r*cos{\thetazero}) {$l\sin\theta_0$};
    \node at (\xc + 0.5*\r*sin{\thetazero} + 0.5, \yc - 0.5*\r*cos{\thetazero})%
          {$l$};
    \node at (\xc + \r*sin{\thetazero} + 0.5, \yc - \r*cos{\thetazero}) {$m$};
    \node at (\xc + 0.6, \yc - 1.3) {$\theta_0$};
  \end{tikzpicture}
  \caption{Schematizzazione dell'apparato sperimentale e definizioni di base.}
  \label{fig:pendolo}
\end{figure}

La lunghezza $l$ \`e la distanza tra il punto di sospensione ed
il centro di massa del pendolo stesso.
L'ampiezza di oscillazione $\theta_0$ \`e l'angolo formato dal filo con la
verticale all'inizio dell'oscillazione.
Il periodo di oscillazione $T$ \`e il tempo che il pendolo impiega a
compiere un'oscillazione completa.

L'espressione per il periodo del pendolo si pu\`o sviluppare in serie come
\begin{align}\label{eq:periodo_pendolo}
  T = 2\pi\sqrt{\frac{l}{g}} \left( 1 + \frac{1}{16}\theta_0^2 +
  \frac{11}{3072}\theta_0^4 + \cdots \right).
\end{align}


\secmeasurements

Realizzate un \emph{pendolo semplice} (per quanto praticamente possibile) con il
materiale che avete a disposizione a casa. Fate in modo che la lunghezza sia
modificabile con facilit\`a.

\labsubsection{Misure da effettuare}

Si misuri con il cronometro (o con lo \emph{smartphone} utilizzato come cronometro)
il periodo del pendolo per diversi valori:
\begin{itemize}
    \item della lunghezza $l$ (ad ampiezza di oscillazione fissata);
    \item dell'ampiezza iniziale di oscillazione $\theta_0$ (a lunghezza fissata),
\end{itemize}
e si confrontino i risultati ottenuti con quanto atteso dalla teoria.

Va da s\'e che, nell'esecuzione della misura, si deve far tesoro di
quanto imparato nelle esercitazioni precedenti.


\secconsiderations

\emph{Non utilizzate strumenti pericolosi, o con i quali non vi sentite
  perfettamente a vostro agio, nella realizzazione dell'esperienza. Non \`e
  necessario per la buona riuscita dell'eseperienza stessa!}

Parte integrante dell'esperienza \`e la scelta di una griglia ragionevole di
valori di lunghezza e di ampiezza iniziale di oscillazione. Per la seconda, in
particolare, bisogna assicurarsi di arrivare ad un'ampiezza tale che la
deviazione dal regime di piccole oscillazioni sia chiaramente misurabile.


\end{article}
\end{document}
