\documentclass{lab1-article}

\usepackage{hyperref}


\title{Breve guida al programma di acquisizione}


\begin{document}


\begin{article}
\selectlanguage{italian}

\maketitle

\secsummary
Di seguito alcune brevi note sull'utilizzo del programma di acquisizione dati
in uso nei laboratori didattici del primo anno. Si consiglia di leggere
questo breve documento almeno una volta e di utilizzarlo come riferimento.


\labsection{L'interfaccia grafica principale}

Una volta acceso il calcolatore, selezionare dal men\`u principale
(sulla barra superiore del \emph{Desktop}, in alto a sinistra)
\menuitem{Application}~$\rightarrow$~\menuitem{Education}~$\rightarrow$~%
\menuitem{plasduino}. Questo dovrebbe mostrare la finestra principale del
programma di acquisizione, che contiene una lista dei moduli (cio\`e delle
applicazioni specifiche ad ogni singola esperienza) disponibili.


\labsection{Selezionare e lanciare un modulo}

Il modulo specifico per l'esperienza (che sar\`a specificato di volta in volta
nella traccia a vostra disposizione) si seleziona tramite un doppio click
(del tasto sinistro del \emph{mouse}) sul nome del modulo oppure selezionando
il nome stesso (click sinistro singolo) e premendo il tasto \menuitem{Open}.


\labsection{Iniziare e terminare l'acquisizione}

Di norma l'acquisizione dati si avvia e si interrompe premendo i tasti opportuni
(rispettivamente quelli che assomigliano ai tasti di \emph{Play} e \emph{Stop}
di un lettore multimediale) sulla barra di trasporto nella parte inferiore
dell'interfaccia grafica di ogni modulo. Se siete incerti, posizionate il
\emph{mouse} sul bottone in questione: una piccola finestrella grafica
contenente una breve spiegazione dovrebbe venirvi in aiuto.

Ogni sessione di acquisizione dati, dal momento in cui si avvia al momento in
cui si termina l'acquisizione stessa, \`e univocamente identificata da un
numero, il \emph{Run number}, visibile sull'interfaccia grafica di ciascun
modulo. Il \emph{Run number}, che viene incrementato di una unit\`a all'inizio
di ciascuna sessione, \`e utilizzato internamente dal programma per archiviare
i dati in modo coerente.


\labsection{Salvare i dati}

Di norma il programma di acquisizione%
\footnote{Fanno eccezione i moduli \emph{Plane}, per cui i dati non vengono
  scritti su disco e devono essere annotati manualmente, ed il modulo
  \emph{Video Click}, per cui il salvataggio dei dati deve essere fatto
  manualmente dall'apposito men\`u a tendina.}
salva automaticamente tutti i dati raccolti in due \emph{file} distinti: uno
\emph{grezzo}, in formato binario, ed uno, opportunamente elaborato, in formato
testo. Quest'ultimo \`e quello da utilizzare, quando necessario, in fase di
analisi. Nel terminale che viene aperto al lancio dell'interfaccia grafica
trovate stampato il percorso completo di ogni \emph{file} salvato in uscita.

Se l'apposita opzione \`e abilitata, al termine di ogni sessione di presa dati
il programma vi chiede se volete salvare una copia del \emph{file} di testo
con i dati da analizzare in una cartella a vostra scelta. Nel caso in cui
questa funzionalit\`a dovesse essere disabilitata potete ri-abilitarla
attraverso il men\`u di plasduino
\menuitem{Configuration}~$\rightarrow$~\menuitem{Change settings}:
nella finestra che si apre selezionate il \emph{tab}~\menuitem{daq} e abilitate
l'opzione \menuitem{prompt-save-dialog} (cliccando sul piccolo quadratino a
destra della finestra di configurazione e selezionando \emph{True}),
quindi riavviate il programma.


\labsection{I moduli con grafici interattivi}

Alcuni moduli, tra cui \emph{Temperature Monitor}, \emph{Pendulum View} e
\emph{Pendulum Drive}, sono caratterizzati da grafici, integrati
nell'interfaccia di acquisizione, che si aggiornano automaticamente al
procedere della sessione. Alcuni suggerimenti su come utilizzarli al meglio:
\begin{itemize}
\item Posizionando il \emph{mouse} all'interno dell'area del grafico il
  sistema mostra le coordinate corrispondenti alla posizione del
  puntatore.
\item Dall'istante in cui si termina la sessione di acquisizione il grafico
  diviene interattivo ed \`e possibile ingrandirne porzioni specifiche.
  Per fare uno \emph{zoom} \`e sufficiente selezionare il rettangolo
  desiderato tenendo premuto (e poi rilasciando) il tasto sinistro del
  \emph{mouse}. Un click singolo del tasto destro riporta alla visualizzazione
  di partenza.
\end{itemize}


\labsection{Commenti?}

Il programma di acquisizione \`e in fase di sviluppo. Se pensate di aver
identificato un \emph{bug} oppure avete idee per migliorarlo potete parlare con
il vostro esercitatore o, meglio ancora, esporre le vostre idee su
\url{https://bitbucket.org/lbaldini/plasduino/issues}
(utilizzando il tasto \emph{Create issue}).


\end{article}
\end{document}

