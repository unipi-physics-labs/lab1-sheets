\documentclass{lab1-article}

\title{Misure di indice di rifrazione}

\input{code/python}

\begin{document}


\begin{article}
\selectlanguage{italian}

\maketitle

\secsummary

L'obiettivo dell'esperienza consiste nella misura degli indici di rifrazione
del plexiglass e dell'acqua.


\secmaterials

\begin{itemize}
\item Banco ottico con sorgente luminosa;
\item un semicilindro di plexiglass;
\item un diottro sferico riempito di acqua;
\item un metro a nastro (risoluzione $1$~mm).
\end{itemize}


\secmeasurements


\labsubsection{Indice di rifrazione del plexiglass}

\begin{figure}[htb!]
  \begin{tikzpicture}[scale=1]
    \pgfmathsetmacro{\xc}{4.5}
    \pgfmathsetmacro{\yc}{0}
    \pgfmathsetmacro{\r}{2.25}
    \pgfmathsetmacro{\rthetaa}{0.7}
    \pgfmathsetmacro{\rthetab}{0.75}
    \pgfmathsetmacro{\l}{4}
    \pgfmathsetmacro{\thetain}{30}
    \pgfmathsetmacro{\thetarif}{20}
    \node at (0, 0) {};
    \draw (\xc, \yc) arc (90:270:\r);
    \draw (\xc, \yc)--(\xc, \yc-2*\r);
    \draw[style=densely dashed] (0, \yc - \r)--(8, \yc - \r);
    \draw (\xc, \yc-\r)--(\xc+\l*cos{\thetain}, \yc-\r+\l*sin{\thetain});
    \draw (\xc, \yc-\r)--(\xc-\l*cos{\thetarif}, \yc-\r-\l*sin{\thetarif});
    \draw (\xc+\rthetaa, \yc-\r) arc (0:\thetain:\rthetaa);
    \draw (\xc+\rthetab, \yc-\r) arc (0:\thetain:\rthetab);
    \draw (\xc-\rthetab, \yc-\r) arc (180:180+\thetarif:\rthetab);
    \node at (\xc+2*\rthetab*cos{\thetain},%
    \yc-\r+2*\rthetab*sin{\thetain/2}) {$\theta_{\rm i}$};
    \node at (\xc-2*\rthetab*cos{\thetarif},%
    \yc-\r-2*\rthetab*sin{\thetarif/2}) {$\theta_{\rm r}$};
    \node at (\xc+2*\rthetaa, \yc-1.5*\r) {$n_1 \sim 1$};
    \node at (\xc-1.25*\rthetaa, \yc-1.5*\r) {$n_2$};
  \end{tikzpicture}
  \caption{Schematizzazione dell'apparato per la misura dell'indice di
    rifrazione del plexiglass.
    L'angolo di incidenza $\theta_{\rm i}$ (di rifrazione $\theta_{\rm r}$)
    \`e l'angolo formato dal raggio luminoso incidente (rifratto) con la
    normale alla superficie di separazione tra i due mezzi.}
  \label{fig:plexiglass}
\end{figure}


Se un raggio di luce passa da un mezzo con indice di rifrazione $n_1$ ad
uno con indice di rifrazione $n_2$, gli angoli di incidenza e di rifrazione
sono legati tra di loro dalla legge di Snell
\begin{align}
  n_1 \sin\theta_{\rm i} = n_2\sin\theta_{\rm r}.
\end{align}

Si posizioni il semicilindro in modo che il raggio incida al centro della
superficie piana rifrangente (per evitare una seconda rifrazione in uscita) e
si misuri una serie di coppie $(\sin\theta_{\rm i},~\sin\theta_{\rm r})$ per un
certo numero (diciamo $10$) di valori di $\theta_{\rm i}$.
Si ricavi l'indice di rifrazione cercato da un fit lineare alle misure,
ricordando che l'indice di rifrazione dell'aria \`e con buona approssimazione
$n_1 \sim 1$.


\labsubsection{Indice di rifrazione dell'acqua}

\begin{figure}[htb!]
  \begin{tikzpicture}[scale=1]
    \pgfmathsetmacro{\xc}{2}
    \pgfmathsetmacro{\yc}{-1}
    \pgfmathsetmacro{\r}{2.25}
    \node at (0, 0) {};
    \draw (\xc, \yc) arc (180:360:\r);
    \fill (\xc+\r, 0) circle [radius=0.075];
    \node[anchor=west] at (\xc+\r, 0) {Sorgente};
    \fill (\xc+\r, \yc-1.75*\r) circle [radius=0.075];
    \node[anchor=west] at (\xc+\r, \yc-1.75*\r) {Immagine};
    \draw[style=densely dashed] (\xc+\r, 0)--(\xc+\r, \yc-1.75*\r);    
    \node at (\xc+\r+0.2, 0.5*\yc-0.5*\r) {$p$};
    \node at (\xc+\r+0.2, \yc-1.375*\r) {$q$};
    \node at (\xc, \yc-1.375*\r) {$n_1 \sim 1$};
    \node at (\xc+0.75, \yc-0.375*\r) {$n_2$};
  \end{tikzpicture}
  \caption{Schematizzazione dell'apparato per la misura dell'indice di
    rifrazione dell'acqua. Le grandezze $q$ e $p$ sono definite,
    rispettivamente come la distanza dal vertice del diottro della
    sorgente e dell'immagine (a fuoco sullo schermo).}
  \label{fig:acqua}
\end{figure}

Con riferimento alla figura~\ref{fig:acqua}, e detto $r$ il raggio del diottro,
$p$ e $q$ sono legati dalla relazione
\begin{align}\label{eq:diottro}
  \frac{n_2}{p} + \frac{n_1}{q} = \frac{(n_2 - n_1)}{r}.
\end{align}
%che, nel caso il diottro stesso sia immerso in aria ($n_1 \sim 1$) diviene
%\begin{align}\label{eq:diottro}
%  \frac{n_2}{p} + \frac{1}{q} = \frac{(n_2 - 1)}{r}.
%\end{align}

Operativamente, fissata una posizione per la sorgente, si muova lo schermo fino
a che l'immagine non \`e a fuoco, e si misurino $p$ e $q$. Si ripeta
l'operazione per diverse posizioni della sorgente e si costruisca il grafico
cartesiano di $1/q$ in funzione di $1/p$.
Ricordando che $n_1 \sim 1$, per la~\eqref{eq:diottro} le due grandezze saranno
legate da
\begin{align}
  \frac{1}{q} = -\frac{n_2}{p} + \frac{(n_2 - 1)}{r}.
\end{align}
Tramite fit lineare si stimi l'indice di rifrazione cercato come il
coefficiente angolare della retta di \emph{best fit}.


\secconsiderations

\labsubsection{Indice di rifrazione del plexiglass}

Troverete gi\`a montati sul banco ottico, accanto alla sorgente di luce, una
lente convergente ed un diaframma a fenditura per creare un fascio di luce
sottile. Non dovrebbe essere necessario modificare il montaggio---in caso di
bisogno chiedete aiuto all'esercitatore.


\labsubsection{Indice di rifrazione dell'acqua}

Come vedrete, la sorgente luminosa \`e immersa in acqua, per cui si raccomanda
di fare attenzione, durante gli spostamenti, onde evitare spiacevoli
fuoriuscite.

Si noti che l'oggetto da mettere a fuoco \`e un piccolo rombo incollato sulla
sorgente.


\secappendix{valori tabulati}

Si riportano di seguito i valori \emph{indicativi} degli indici di rifrazione
da misurare.

\medskip

\begin{center}
\begin{tabular}{p{0.65\linewidth}@{}p{0.3\linewidth}}
\hline
Materiale & $n$\\
\hline
\hline
Plexiglass & $1.48$\\
Acqua & $1.33$\\
\hline
\end{tabular}
\end{center}

\vfill

\noindent{\scriptsize Si veda il retro per un programma di esempio per
l'analisi dei dati con il calcolatore.\par}


\onecolumn

\begin{Verbatim}[label=\makebox{\href{https://github.com/unipi-physics-labs/lab1-sheets/tree/main/snippy/refraction_index.py}{https://github.com/.../refraction\_index.py}},commandchars=\\\{\}]
\PY{c+c1}{\PYZsh{} Programma di esempio per l\PYZsq{}analisi delle misure sull\PYZsq{}indice di rifrazione di acqua e plexiglass}
\PY{k+kn}{import}\PY{+w}{ }\PY{n+nn}{pylab}
\PY{k+kn}{from}\PY{+w}{ }\PY{n+nn}{scipy}\PY{n+nn}{.}\PY{n+nn}{optimize}\PY{+w}{ }\PY{k+kn}{import} \PY{n}{curve\PYZus{}fit}

\PY{c+c1}{\PYZsh{} Dati in ingresso per il diottro (da modificare con le vostre misure).}
\PY{n}{p} \PY{o}{=} \PY{n}{pylab}\PY{o}{.}\PY{n}{array}\PY{p}{(}\PY{p}{[}\PY{l+m+mf}{45.6}\PY{p}{,} \PY{l+m+mf}{42.9}\PY{p}{,} \PY{l+m+mf}{41.0}\PY{p}{,} \PY{l+m+mf}{38.7}\PY{p}{,} \PY{l+m+mf}{35.8}\PY{p}{,} \PY{l+m+mf}{34.1}\PY{p}{]}\PY{p}{,} \PY{l+s+s1}{\PYZsq{}}\PY{l+s+s1}{d}\PY{l+s+s1}{\PYZsq{}}\PY{p}{)}
\PY{n}{q} \PY{o}{=} \PY{n}{pylab}\PY{o}{.}\PY{n}{array}\PY{p}{(}\PY{p}{[}\PY{l+m+mf}{43.5}\PY{p}{,} \PY{l+m+mf}{47.6}\PY{p}{,} \PY{l+m+mf}{51.2}\PY{p}{,} \PY{l+m+mf}{56.4}\PY{p}{,} \PY{l+m+mf}{68.1}\PY{p}{,} \PY{l+m+mf}{75.4}\PY{p}{]}\PY{p}{,} \PY{l+s+s1}{\PYZsq{}}\PY{l+s+s1}{d}\PY{l+s+s1}{\PYZsq{}}\PY{p}{)}
\PY{n}{Dp} \PY{o}{=} \PY{n}{pylab}\PY{o}{.}\PY{n}{array}\PY{p}{(}\PY{n+nb}{len}\PY{p}{(}\PY{n}{p}\PY{p}{)}\PY{o}{*}\PY{p}{[}\PY{l+m+mf}{0.5}\PY{p}{]}\PY{p}{,}\PY{l+s+s1}{\PYZsq{}}\PY{l+s+s1}{d}\PY{l+s+s1}{\PYZsq{}}\PY{p}{)}
\PY{n}{Dq} \PY{o}{=} \PY{n}{pylab}\PY{o}{.}\PY{n}{array}\PY{p}{(}\PY{n+nb}{len}\PY{p}{(}\PY{n}{q}\PY{p}{)}\PY{o}{*}\PY{p}{[}\PY{l+m+mi}{1}\PY{p}{]}\PY{p}{,}\PY{l+s+s1}{\PYZsq{}}\PY{l+s+s1}{d}\PY{l+s+s1}{\PYZsq{}}\PY{p}{)}

\PY{c+c1}{\PYZsh{} Plot di 1/q vs 1/p}
\PY{n}{pylab}\PY{o}{.}\PY{n}{figure}\PY{p}{(}\PY{l+m+mi}{1}\PY{p}{)}
\PY{n}{pylab}\PY{o}{.}\PY{n}{title}\PY{p}{(}\PY{l+s+s1}{\PYZsq{}}\PY{l+s+s1}{Indice di rifrazione dell}\PY{l+s+se}{\PYZbs{}\PYZsq{}}\PY{l+s+s1}{acqua}\PY{l+s+s1}{\PYZsq{}}\PY{p}{)}
\PY{n}{pylab}\PY{o}{.}\PY{n}{xlabel}\PY{p}{(}\PY{l+s+s1}{\PYZsq{}}\PY{l+s+s1}{1/p [1/cm]}\PY{l+s+s1}{\PYZsq{}}\PY{p}{)} 
\PY{n}{pylab}\PY{o}{.}\PY{n}{ylabel}\PY{p}{(}\PY{l+s+s1}{\PYZsq{}}\PY{l+s+s1}{1/q [1/cm]}\PY{l+s+s1}{\PYZsq{}}\PY{p}{)} 
\PY{n}{pylab}\PY{o}{.}\PY{n}{grid}\PY{p}{(}\PY{n}{color} \PY{o}{=} \PY{l+s+s1}{\PYZsq{}}\PY{l+s+s1}{gray}\PY{l+s+s1}{\PYZsq{}}\PY{p}{)}
\PY{n}{pylab}\PY{o}{.}\PY{n}{errorbar}\PY{p}{(}\PY{l+m+mf}{1.}\PY{o}{/}\PY{n}{p}\PY{p}{,} \PY{l+m+mi}{1}\PY{o}{/}\PY{n}{q}\PY{p}{,} \PY{n}{Dp}\PY{o}{/}\PY{p}{(}\PY{n}{p}\PY{o}{*}\PY{n}{p}\PY{p}{)}\PY{p}{,} \PY{n}{Dq}\PY{o}{/}\PY{p}{(}\PY{n}{q}\PY{o}{*}\PY{n}{q}\PY{p}{)}\PY{p}{,} \PY{l+s+s1}{\PYZsq{}}\PY{l+s+s1}{o}\PY{l+s+s1}{\PYZsq{}}\PY{p}{,} \PY{n}{color}\PY{o}{=}\PY{l+s+s1}{\PYZsq{}}\PY{l+s+s1}{black}\PY{l+s+s1}{\PYZsq{}} \PY{p}{)}

\PY{c+c1}{\PYZsh{} Fit con una retta \PYZhy{} nota che le incertezze sono ignorate!}
\PY{k}{def}\PY{+w}{ }\PY{n+nf}{f}\PY{p}{(}\PY{n}{x}\PY{p}{,} \PY{n}{a}\PY{p}{,} \PY{n}{b}\PY{p}{)}\PY{p}{:}
    \PY{k}{return} \PY{n}{a}\PY{o}{*}\PY{n}{x} \PY{o}{+} \PY{n}{b}

\PY{n}{popt}\PY{p}{,} \PY{n}{pcov} \PY{o}{=} \PY{n}{curve\PYZus{}fit}\PY{p}{(}\PY{n}{f}\PY{p}{,} \PY{l+m+mf}{1.}\PY{o}{/}\PY{n}{p}\PY{p}{,} \PY{l+m+mi}{1}\PY{o}{/}\PY{n}{q}\PY{p}{,} \PY{n}{pylab}\PY{o}{.}\PY{n}{array}\PY{p}{(}\PY{p}{[}\PY{o}{\PYZhy{}}\PY{l+m+mf}{1.}\PY{p}{,}\PY{l+m+mf}{1.}\PY{p}{]}\PY{p}{)}\PY{p}{)}
\PY{n}{a}\PY{p}{,} \PY{n}{b}       \PY{o}{=} \PY{n}{popt}
\PY{n}{da}\PY{p}{,} \PY{n}{db}     \PY{o}{=} \PY{n}{pylab}\PY{o}{.}\PY{n}{sqrt}\PY{p}{(}\PY{n}{pcov}\PY{o}{.}\PY{n}{diagonal}\PY{p}{(}\PY{p}{)}\PY{p}{)}
\PY{n+nb}{print}\PY{p}{(}\PY{l+s+s1}{\PYZsq{}}\PY{l+s+s1}{Acqua: n = }\PY{l+s+si}{\PYZpc{}f}\PY{l+s+s1}{ +\PYZhy{} }\PY{l+s+si}{\PYZpc{}f}\PY{l+s+s1}{\PYZsq{}} \PY{o}{\PYZpc{}} \PY{p}{(}\PY{n}{a}\PY{p}{,} \PY{n}{da}\PY{p}{)}\PY{p}{)}
\PY{n}{pylab}\PY{o}{.}\PY{n}{plot}\PY{p}{(}\PY{l+m+mf}{1.}\PY{o}{/}\PY{n}{p}\PY{p}{,} \PY{n}{f}\PY{p}{(}\PY{l+m+mf}{1.}\PY{o}{/}\PY{n}{p}\PY{p}{,} \PY{n}{a}\PY{p}{,} \PY{n}{b}\PY{p}{)}\PY{p}{,} \PY{n}{color}\PY{o}{=}\PY{l+s+s1}{\PYZsq{}}\PY{l+s+s1}{black}\PY{l+s+s1}{\PYZsq{}} \PY{p}{)}
\PY{n}{pylab}\PY{o}{.}\PY{n}{savefig}\PY{p}{(}\PY{l+s+s1}{\PYZsq{}}\PY{l+s+s1}{rifrazione\PYZus{}acqua.png}\PY{l+s+s1}{\PYZsq{}}\PY{p}{)}

\PY{c+c1}{\PYZsh{} Dati in ingresso per il plexiglass (da modificare con le vostre misure).}
\PY{c+c1}{\PYZsh{} In questo esempio x = R*sin(theta\PYZus{}r), y = R*sin(theta\PYZus{}i) [cm]}
\PY{n}{x} \PY{o}{=} \PY{n}{pylab}\PY{o}{.}\PY{n}{array}\PY{p}{(}\PY{p}{[}\PY{l+m+mf}{1.25}\PY{p}{,} \PY{l+m+mf}{1.85}\PY{p}{,} \PY{l+m+mf}{2.9}\PY{p}{,} \PY{l+m+mf}{4.35}\PY{p}{,} \PY{l+m+mf}{0.5}\PY{p}{,}  \PY{l+m+mf}{0.25}\PY{p}{,} \PY{l+m+mf}{0.8}\PY{p}{,} \PY{l+m+mf}{1.5}\PY{p}{,} \PY{l+m+mf}{4.8}\PY{p}{]}\PY{p}{,} \PY{l+s+s1}{\PYZsq{}}\PY{l+s+s1}{d}\PY{l+s+s1}{\PYZsq{}}\PY{p}{)}
\PY{n}{y} \PY{o}{=} \PY{n}{pylab}\PY{o}{.}\PY{n}{array}\PY{p}{(}\PY{p}{[}\PY{l+m+mf}{1.9}\PY{p}{,}  \PY{l+m+mf}{2.5}\PY{p}{,}  \PY{l+m+mf}{4.2}\PY{p}{,} \PY{l+m+mf}{6.65}\PY{p}{,} \PY{l+m+mf}{0.75}\PY{p}{,} \PY{l+m+mf}{0.45}\PY{p}{,} \PY{l+m+mf}{1.1}\PY{p}{,} \PY{l+m+mf}{2.4}\PY{p}{,} \PY{l+m+mf}{7.} \PY{p}{]}\PY{p}{,} \PY{l+s+s1}{\PYZsq{}}\PY{l+s+s1}{d}\PY{l+s+s1}{\PYZsq{}}\PY{p}{)}
\PY{n}{Dx} \PY{o}{=} \PY{n}{pylab}\PY{o}{.}\PY{n}{array}\PY{p}{(}\PY{n+nb}{len}\PY{p}{(}\PY{n}{x}\PY{p}{)}\PY{o}{*}\PY{p}{[}\PY{l+m+mf}{0.1}\PY{p}{]}\PY{p}{,}\PY{l+s+s1}{\PYZsq{}}\PY{l+s+s1}{d}\PY{l+s+s1}{\PYZsq{}}\PY{p}{)}
\PY{n}{Dy} \PY{o}{=} \PY{n}{pylab}\PY{o}{.}\PY{n}{array}\PY{p}{(}\PY{n+nb}{len}\PY{p}{(}\PY{n}{y}\PY{p}{)}\PY{o}{*}\PY{p}{[}\PY{l+m+mf}{0.1}\PY{p}{]}\PY{p}{,}\PY{l+s+s1}{\PYZsq{}}\PY{l+s+s1}{d}\PY{l+s+s1}{\PYZsq{}}\PY{p}{)}

\PY{c+c1}{\PYZsh{} Plot di x vs y}
\PY{n}{pylab}\PY{o}{.}\PY{n}{figure}\PY{p}{(}\PY{l+m+mi}{2}\PY{p}{)}
\PY{n}{pylab}\PY{o}{.}\PY{n}{title}\PY{p}{(}\PY{l+s+s1}{\PYZsq{}}\PY{l+s+s1}{Indice di rifrazione del plexiglass}\PY{l+s+s1}{\PYZsq{}}\PY{p}{)}
\PY{n}{pylab}\PY{o}{.}\PY{n}{xlabel}\PY{p}{(}\PY{l+s+s1}{\PYZsq{}}\PY{l+s+s1}{R sin(theta\PYZus{}r) [cm]}\PY{l+s+s1}{\PYZsq{}}\PY{p}{)} 
\PY{n}{pylab}\PY{o}{.}\PY{n}{ylabel}\PY{p}{(}\PY{l+s+s1}{\PYZsq{}}\PY{l+s+s1}{R sin(theta\PYZus{}i) [cm]}\PY{l+s+s1}{\PYZsq{}}\PY{p}{)} 
\PY{n}{pylab}\PY{o}{.}\PY{n}{grid}\PY{p}{(}\PY{n}{color} \PY{o}{=} \PY{l+s+s1}{\PYZsq{}}\PY{l+s+s1}{gray}\PY{l+s+s1}{\PYZsq{}}\PY{p}{)}
\PY{n}{pylab}\PY{o}{.}\PY{n}{errorbar}\PY{p}{(}\PY{n}{x}\PY{p}{,} \PY{n}{y}\PY{p}{,} \PY{n}{Dx}\PY{p}{,} \PY{n}{Dy}\PY{p}{,} \PY{l+s+s1}{\PYZsq{}}\PY{l+s+s1}{o}\PY{l+s+s1}{\PYZsq{}}\PY{p}{,} \PY{n}{color}\PY{o}{=}\PY{l+s+s1}{\PYZsq{}}\PY{l+s+s1}{black}\PY{l+s+s1}{\PYZsq{}} \PY{p}{)}

\PY{c+c1}{\PYZsh{} Fit con una retta per essere sicuri che il termine noto sia compatibile con zero}
\PY{n}{popt}\PY{p}{,} \PY{n}{pcov} \PY{o}{=} \PY{n}{curve\PYZus{}fit}\PY{p}{(}\PY{n}{f}\PY{p}{,} \PY{n}{x}\PY{p}{,} \PY{n}{y}\PY{p}{,} \PY{n}{pylab}\PY{o}{.}\PY{n}{array}\PY{p}{(}\PY{p}{[}\PY{l+m+mf}{1.}\PY{p}{,}\PY{l+m+mf}{0.}\PY{p}{]}\PY{p}{)}\PY{p}{)}
\PY{n}{a}\PY{p}{,} \PY{n}{b}       \PY{o}{=} \PY{n}{popt}
\PY{n}{da}\PY{p}{,} \PY{n}{db}     \PY{o}{=} \PY{n}{pylab}\PY{o}{.}\PY{n}{sqrt}\PY{p}{(}\PY{n}{pcov}\PY{o}{.}\PY{n}{diagonal}\PY{p}{(}\PY{p}{)}\PY{p}{)}
\PY{n+nb}{print}\PY{p}{(}\PY{l+s+s1}{\PYZsq{}}\PY{l+s+s1}{Plexiglass: b = }\PY{l+s+si}{\PYZpc{}f}\PY{l+s+s1}{ +\PYZhy{} }\PY{l+s+si}{\PYZpc{}f}\PY{l+s+s1}{ compatibile con 0?}\PY{l+s+s1}{\PYZsq{}} \PY{o}{\PYZpc{}} \PY{p}{(}\PY{n}{b}\PY{p}{,} \PY{n}{db}\PY{p}{)}\PY{p}{)}

\PY{c+c1}{\PYZsh{} Fit con la legge di Snell}
\PY{k}{def}\PY{+w}{ }\PY{n+nf}{f1}\PY{p}{(}\PY{n}{x}\PY{p}{,} \PY{n}{a}\PY{p}{)}\PY{p}{:}
    \PY{k}{return} \PY{n}{a}\PY{o}{*}\PY{n}{x}

\PY{n}{popt}\PY{p}{,} \PY{n}{pcov} \PY{o}{=} \PY{n}{curve\PYZus{}fit}\PY{p}{(}\PY{n}{f1}\PY{p}{,} \PY{n}{x}\PY{p}{,} \PY{n}{y}\PY{p}{,} \PY{n}{pylab}\PY{o}{.}\PY{n}{array}\PY{p}{(}\PY{p}{[}\PY{l+m+mf}{1.}\PY{p}{]}\PY{p}{)}\PY{p}{)}
\PY{n+nb}{print}\PY{p}{(}\PY{l+s+s1}{\PYZsq{}}\PY{l+s+s1}{Plexiglass: n = }\PY{l+s+si}{\PYZpc{}f}\PY{l+s+s1}{ +\PYZhy{} }\PY{l+s+si}{\PYZpc{}f}\PY{l+s+s1}{\PYZsq{}} \PY{o}{\PYZpc{}} \PY{p}{(}\PY{n}{popt}\PY{p}{,} \PY{n}{pylab}\PY{o}{.}\PY{n}{sqrt}\PY{p}{(}\PY{n}{pcov}\PY{o}{.}\PY{n}{diagonal}\PY{p}{(}\PY{p}{)}\PY{p}{)}\PY{p}{)}\PY{p}{)}
\PY{n}{pylab}\PY{o}{.}\PY{n}{plot}\PY{p}{(}\PY{n}{x}\PY{p}{,} \PY{n}{f1}\PY{p}{(}\PY{n}{x}\PY{p}{,} \PY{n}{a}\PY{p}{)}\PY{p}{,} \PY{n}{color}\PY{o}{=}\PY{l+s+s1}{\PYZsq{}}\PY{l+s+s1}{black}\PY{l+s+s1}{\PYZsq{}} \PY{p}{)}
\PY{n}{pylab}\PY{o}{.}\PY{n}{savefig}\PY{p}{(}\PY{l+s+s1}{\PYZsq{}}\PY{l+s+s1}{rifrazione\PYZus{}plexiglass.png}\PY{l+s+s1}{\PYZsq{}}\PY{p}{)}

\PY{n}{pylab}\PY{o}{.}\PY{n}{show}\PY{p}{(}\PY{p}{)}
\end{Verbatim}


\end{article}
\end{document}
