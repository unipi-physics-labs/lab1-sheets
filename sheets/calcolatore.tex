\documentclass{lab1-article}

\title{Esercitazione al calcolatore}

\usepackage{hyperref}

\begin{document}


\begin{article}
\selectlanguage{italian}

\maketitle

\secsummary

In questa esperienza cercheremo di familiarizzare con la visualizzazione e
l'analisi dei dati al calcolatore.


\secmaterials

\begin{itemize}
\item Computer, carta, penna e righello.
\item Dati delle esperienze della conducibilit\`a termica e del pendolo
  fisico.
\item Buona volont\`a.
\end{itemize}


\labsection{Visualizzazione dei dati}


\labsubsection{Conducibilit\`a termica}

Trascrivete i dati relativi all'esperienza della conducibilit\`a termica
(per una qualsiasi delle due sbarre) in un \emph{file} di testo, scegliendo
oppurtunamente il formato. (Tipicamente scriverete quattro colonne,
che identificheranno rispettivamente la posizione, la temperatura e le
relative incertezze di misura.)

Scrivete un breve programma in Python per leggere i dati nel \emph{file}
di testo che avete preparato e realizzate un grafico di dispersione dei dati
stessi. (Prestate attenzione alla rappresentazione degli errori ed indicate
contenuto ed unit\`a di misura sugli assi.)

Realizzate un fit analitico con una retta utilizzando il calcolatore per
ricavare il gradiente di temperatura, e visualizzate la retta corrispondente
ai vostri parametri di \emph{best fit} sul grafico.


\labsubsection{Pendolo fisico}

Trascrivete i dati relativi all'esperienza del pendolo fisico in un \emph{file}
di testo e, analogamente a quanto fatto nel punto precedente, realizzate
un grafico di dispersione dei valori misurati $T_i$ del periodo in funzione
della distanza $d_i$ tra punto di sospensione e centro di massa.

Aggiungete al grafico di dispersione quello corrispondente al nostro modello
per il pendolo fisico
\begin{align}\label{eq:period}
  T(d) = 2\pi\sqrt{\frac{(l^2/12 + d^2)}{gd}},
\end{align}
utilizzando il valore misurato per la lunghezza $l$ della sbarra. Commentare sul
livello di accordo (o disaccordo) tra dati e modello.

Realizzate un grafico dei residui, cio\`e un grafico di dispersione in cui,
in funzione dei valori misurati della distanza $d_i$ tra punto di sospensione
e centro di massa si mostra la differenza
\begin{align}
  r_i = T_i - T(d_i)
\end{align}
tra periodo misurato e periodo previsto dalla \eqref{eq:period}. (Si faccia
attenzione a propagare correttamente le incertezze di misura.) Il grafico
dei residui aiuta a valutare l'accordo tra dati e modello?


\labsection{Stesura della relazione}

Se possibile relalizzate la relazione in \LaTeX\ ed inserite nel testo
i grafici che avete realizzato.


\secconsiderations

Durante l'esercitazione sar\`a utile avere sotto mano le dispense sull'uso di
Python disponibili a \url{https://bitbucket.org/lbaldini/computing}.

Per leggere il contentuto di \emph{file} di testo potete utilizzare la
funzione \verb|loadtxt| di \verb|numpy|.

Per realizzare un grafico di dispersione potete utilizzare la funzione
\verb|errorbar| di \verb|matplotlib.pyplot|.

Per realizzare fit analitici potete utilizzare la funzione
\verb|curve_fit| di \verb|scipy.optimize|.

Per realizzare il grafico di una funzione dovete per prima cosa definire una
serie di valori della variabile indipendente per il suo campionamento
(ad esempio con la funzione \verb|linspace| di \verb|numpy|), poi valutare
la funzione sui valori stessi, e infine creare il grafico, ad esempio con
la funzione \verb|plot| di \verb|matplotlib.pyplot|.


\end{article}
\end{document}
