\documentclass{lab1-article}

\title{Random walk}

\usepackage{fancyvrb}
\input{python}


\begin{document}


\begin{article}
\selectlanguage{italian}

\maketitle

\secsummary
Lo scopo dell'esperienza \`e quello di simulare un random walk in due dimensioni
e studiare alcune delle propriet\`a statistiche del fenomeno.

\secmaterials

\begin{itemize}
    \item Calcolatore dotato di Python e dell'ecosistema scientifico associato.
\end{itemize}

\secmeasurements

Nella sua versione pi\`u semplice, un random walk in due dimensioni consiste in
una sequenza di passi di lunghezza fissata (che per comodit\`a possiamo assumere
uguale ad $1$ in qualche unit\`a di misura), in direzioni casuali.
Formalmente, dato un sistema di riferimento cartesiano, e assumendo di partire
dalla posizione $(x,~y) = (0,~0)$ ad ogni passo dobbiamo:
\begin{itemize}
    \item estrarre una direzione casuale $\phi \in [0,~2\pi]$;
    \item aggiornare la posizione corrente secondo la relazione
        $(x,~y) \rightarrow (x + \cos\phi,~y + \sin\phi)$.
\end{itemize}


\labsubsection{Rappresentazione grafica}

Per cominciare rappresentate graficamente una particolare realizzazione del processo
con un numero di passi a vostra scelta, ad esempio $1000$. (Sperimentate liberamente
e scegliete la configurazione che vi sembra pi\`u interessante).

Operativamente potete realizzare un ciclo \texttt{for} in cui riempite una lista
con le posizioni ad ogni passo, e poi utilizzare la funzione \texttt{plot} del modulo
\texttt{matplotlib}, esattamente come se si trattasse di un grafico di dispersione.
Cercate di curare la presentazione grafica, ed inserite l'immagine ottenuta nella
relazione.


\labsubsection{Studio delle propriet\`a statistiche}

Per studiare le propriet\`a statistiche del random walk, dovrete creare un certo
numero di realizzazioni del processo, mantenendo costante il numero di passi.
Le cose che ci interessano sono essenzialmente la posizione finale $(x,~y)$ del
random walk, e la distanza corrispondente $d = \sqrt{x^2 + y^2}$ dall'origine, per
cui in questo caso, all'interno del nostro ciclo \texttt{for}, non \`e necessario
tener traccia di tutte le posizioni---\`e sufficiente aggiornare una sola coppia
ordinata $(x,~y)$. Operativamente potete creare una funzione che accetti come unico
parametro il numero di passi, e restituisca la posizione finale nella forma di una
tupla.

Cominciate con lo scegliere un numero di passi $n$ (ad esempio $100$), e generate un
gran numero $N$ di realizzazioni (ad esempio $10000$) del processo. Create gli istogrammi
di $x$ (oppure $y$ che dovrebbe essere qualitativamente simile) e $d$ e verificate
che seguano le distribuzioni attese attraverso un test del $\chi^2$. Notate in particolare
che:
\begin{itemize}
    \item $x$ e $y$ dovrebbero essere distributi gaussianamente con media $0$ e
        $\sigma = \sqrt{n / 2}$;
    \item $d^2$ \`e la somma di due gaussiane al quadrato, per cui la sua funzione
    di distribuzione \`e legata a quella di un $\chi^2$ a due gradi di libert\`a,
    ovverosia una distribuzione esponenziale con media $n$:
    \begin{align*}
        p(d^2; n) = \frac{1}{n} e^{-\frac{d^2}{n}};
    \end{align*}
    la funzione di distribuzione per $d$ si pu\`o ottenere da questa con un cambio
    di variabile
    \begin{align}\label{eq:pdf_d}
        p(d;~n) = \frac{2d}{n} e^{-\frac{d^2}{n}}.
    \end{align}
    (In pratica l'unica differenza \`e lo Jacobiano $2d$ della trasformazione
    $d^2 \rightarrow d$.)
\end{itemize}

Notate che il valor medio di $d$ si pu\`o calcolare a partire dalla funzione di
distribuzione~\eqref{eq:pdf_d}
\begin{align}
    E[d] = \int_0^\infty d p(d; n) dd = \frac{\sqrt{\pi}}{2} \sqrt{n}.
\end{align}
e scala come $\sqrt{n}$, come potete verificare direttamente cambiando $n$, se
siete particolarmente motivati.

% \labsubsection{Legge di scala per il valor medio di $d$}

% Si studi come varia il valor medio di $d$ al variare del numero di passi $N$.
% A questo scopo, fissata una griglia di valori di $N$ (ad esempio 10, 25, 50, 100,
% 250, 500, 1000) si generi un certo numero di realizzazioni del nostro random walk e
% si faccia un grafico di $\left<d\right>$ in funzione di N. Verificare con un fit
% ed un test del $\chi^2$ che
% \begin{align}
%     \left<d\right> = \frac{\sqrt{\pi}}{2} \sqrt{N}.
% \end{align}



\secconsiderations

Per lo svolgimento dell'esperienza utilizzerete principalmente il generatore di
numeri pseudo-casuali disponibile nel modulo \texttt{random} della libreria \texttt{numpy},
come mostrato negli esempi nella prossima pagina. (Non c'\`e niente di profondo, ma
\`e computazionalmente pi\`u efficiente generare \emph{array} di numeri con
\texttt{numpy}, piuttosto che fare un loop esplicito in Python.)

Quando fate il test del chi quadro per gli istogrammi userete la definizione
\begin{align}
    \chi^2 = \sum_i \frac{(o_i - e_i)^2}{e_i},
\end{align}
in cui $o_i$ sono le occorrenze osservate (ovvero il contenuto dei canali, o \emph{bin},
dell'istogramma), $e_i$ sono le occorrenze attese in base al modello, e l'indice
$i$ scorre su tutti i \emph{bin}. Quando calcolate i valori attesi $e_i$ ricordate
due cose fondamentali:
\begin{itemize}
  \item il modello deve essere normalizzato alla somma $N$ dei contenuti
    dei bin, e non a $1$ (i.e., dovete moltiplicare per $N$ le funzioni di distribuzione);
  \item  formalmente dovreste integrare il modello tra gli estremi dei bin, ma se
    questi ultimi non sono troppo grandi potete prendere il valore della funzione
    al centro del bin, moltiplicando per la larghezza del bin stesso.
\end{itemize}


\onecolumn
\begin{Verbatim}[label=\makebox{\href{https://github.com/unipi-physics-labs/lab1-sheets/tree/main/snippy/random_walk.py}{https://github.com/.../random\_walk.py}},commandchars=\\\{\}]
\PY{k+kn}{import}\PY{+w}{ }\PY{n+nn}{numpy}\PY{+w}{ }\PY{k}{as}\PY{+w}{ }\PY{n+nn}{np}
\PY{k+kn}{import}\PY{+w}{ }\PY{n+nn}{matplotlib}\PY{n+nn}{.}\PY{n+nn}{pyplot}\PY{+w}{ }\PY{k}{as}\PY{+w}{ }\PY{n+nn}{plt}



\PY{k}{def}\PY{+w}{ }\PY{n+nf}{random\PYZus{}walk\PYZus{}2d}\PY{p}{(}\PY{n}{num\PYZus{}steps}\PY{p}{:} \PY{n+nb}{int}\PY{p}{)} \PY{o}{\PYZhy{}}\PY{o}{\PYZgt{}} \PY{n+nb}{float}\PY{p}{:}
\PY{+w}{    }\PY{l+s+sd}{\PYZdq{}\PYZdq{}\PYZdq{}2\PYZhy{}dimensional random walk.}
\PY{l+s+sd}{    \PYZdq{}\PYZdq{}\PYZdq{}}
    \PY{n}{pos} \PY{o}{=} \PY{p}{(}\PY{l+m+mf}{0.}\PY{p}{,} \PY{l+m+mf}{0.}\PY{p}{)}
    \PY{k}{for} \PY{n}{\PYZus{}} \PY{o+ow}{in} \PY{n+nb}{range}\PY{p}{(}\PY{n}{num\PYZus{}steps}\PY{p}{)}\PY{p}{:}
        \PY{n}{phi} \PY{o}{=} \PY{n}{np}\PY{o}{.}\PY{n}{random}\PY{o}{.}\PY{n}{uniform}\PY{p}{(}\PY{l+m+mf}{0.}\PY{p}{,} \PY{l+m+mf}{2.} \PY{o}{*} \PY{n}{np}\PY{o}{.}\PY{n}{pi}\PY{p}{)}
        \PY{n}{pos} \PY{o}{+}\PY{o}{=} \PY{p}{(}\PY{n}{np}\PY{o}{.}\PY{n}{cos}\PY{p}{(}\PY{n}{phi}\PY{p}{)}\PY{p}{,} \PY{n}{np}\PY{o}{.}\PY{n}{sin}\PY{p}{(}\PY{n}{phi}\PY{p}{)}\PY{p}{)}
    \PY{k}{return} \PY{n}{pos}


\PY{k}{def}\PY{+w}{ }\PY{n+nf}{random\PYZus{}walk\PYZus{}2d\PYZus{}fast}\PY{p}{(}\PY{n}{num\PYZus{}steps}\PY{p}{:} \PY{n+nb}{int}\PY{p}{)} \PY{o}{\PYZhy{}}\PY{o}{\PYZgt{}} \PY{n+nb}{float}\PY{p}{:}
\PY{+w}{    }\PY{l+s+sd}{\PYZdq{}\PYZdq{}\PYZdq{}Vectorized 2\PYZhy{}dimensional random walk.}
\PY{l+s+sd}{    \PYZdq{}\PYZdq{}\PYZdq{}}
    \PY{n}{phi} \PY{o}{=} \PY{n}{np}\PY{o}{.}\PY{n}{random}\PY{o}{.}\PY{n}{uniform}\PY{p}{(}\PY{l+m+mf}{0.}\PY{p}{,} \PY{l+m+mf}{2.} \PY{o}{*} \PY{n}{np}\PY{o}{.}\PY{n}{pi}\PY{p}{,} \PY{n}{num\PYZus{}steps}\PY{p}{)}
    \PY{k}{return} \PY{p}{(}\PY{n}{np}\PY{o}{.}\PY{n}{cos}\PY{p}{(}\PY{n}{phi}\PY{p}{)}\PY{o}{.}\PY{n}{sum}\PY{p}{(}\PY{p}{)}\PY{p}{,} \PY{n}{np}\PY{o}{.}\PY{n}{sin}\PY{p}{(}\PY{n}{phi}\PY{p}{)}\PY{o}{.}\PY{n}{sum}\PY{p}{(}\PY{p}{)}\PY{p}{)}


\PY{k}{def}\PY{+w}{ }\PY{n+nf}{plot\PYZus{}random\PYZus{}walk\PYZus{}2d}\PY{p}{(}\PY{n}{num\PYZus{}steps}\PY{p}{:} \PY{n+nb}{int}\PY{p}{,} \PY{n}{step\PYZus{}size}\PY{p}{:} \PY{n+nb}{float} \PY{o}{=} \PY{l+m+mf}{1.}\PY{p}{)} \PY{o}{\PYZhy{}}\PY{o}{\PYZgt{}} \PY{n+nb}{float}\PY{p}{:}
\PY{+w}{    }\PY{l+s+sd}{\PYZdq{}\PYZdq{}\PYZdq{}2\PYZhy{}dimensional random walk.}
\PY{l+s+sd}{    \PYZdq{}\PYZdq{}\PYZdq{}}
    \PY{n}{x} \PY{o}{=} \PY{p}{[}\PY{l+m+mf}{0.}\PY{p}{]}
    \PY{n}{y} \PY{o}{=} \PY{p}{[}\PY{l+m+mf}{0.}\PY{p}{]}
    \PY{k}{for} \PY{n}{\PYZus{}} \PY{o+ow}{in} \PY{n+nb}{range}\PY{p}{(}\PY{n}{num\PYZus{}steps}\PY{p}{)}\PY{p}{:}
        \PY{n}{phi} \PY{o}{=} \PY{n}{np}\PY{o}{.}\PY{n}{random}\PY{o}{.}\PY{n}{uniform}\PY{p}{(}\PY{l+m+mf}{0.}\PY{p}{,} \PY{l+m+mf}{2.} \PY{o}{*} \PY{n}{np}\PY{o}{.}\PY{n}{pi}\PY{p}{)}
        \PY{n}{dx} \PY{o}{=} \PY{n}{step\PYZus{}size} \PY{o}{*} \PY{n}{np}\PY{o}{.}\PY{n}{cos}\PY{p}{(}\PY{n}{phi}\PY{p}{)}
        \PY{n}{dy} \PY{o}{=} \PY{n}{step\PYZus{}size} \PY{o}{*} \PY{n}{np}\PY{o}{.}\PY{n}{sin}\PY{p}{(}\PY{n}{phi}\PY{p}{)}
        \PY{n}{x}\PY{o}{.}\PY{n}{append}\PY{p}{(}\PY{n}{x}\PY{p}{[}\PY{o}{\PYZhy{}}\PY{l+m+mi}{1}\PY{p}{]} \PY{o}{+} \PY{n}{dx}\PY{p}{)}
        \PY{n}{y}\PY{o}{.}\PY{n}{append}\PY{p}{(}\PY{n}{y}\PY{p}{[}\PY{o}{\PYZhy{}}\PY{l+m+mi}{1}\PY{p}{]} \PY{o}{+} \PY{n}{dy}\PY{p}{)}

    \PY{n}{plt}\PY{o}{.}\PY{n}{figure}\PY{p}{(}\PY{l+s+s1}{\PYZsq{}}\PY{l+s+s1}{Random Walk}\PY{l+s+s1}{\PYZsq{}}\PY{p}{)}
    \PY{n}{plt}\PY{o}{.}\PY{n}{plot}\PY{p}{(}\PY{n}{x}\PY{p}{,} \PY{n}{y}\PY{p}{)}
    \PY{n}{plt}\PY{o}{.}\PY{n}{gca}\PY{p}{(}\PY{p}{)}\PY{o}{.}\PY{n}{set\PYZus{}aspect}\PY{p}{(}\PY{l+s+s1}{\PYZsq{}}\PY{l+s+s1}{equal}\PY{l+s+s1}{\PYZsq{}}\PY{p}{)}
    \PY{n}{plt}\PY{o}{.}\PY{n}{xlabel}\PY{p}{(}\PY{l+s+s1}{\PYZsq{}}\PY{l+s+s1}{x [a.u.]}\PY{l+s+s1}{\PYZsq{}}\PY{p}{)}
    \PY{n}{plt}\PY{o}{.}\PY{n}{ylabel}\PY{p}{(}\PY{l+s+s1}{\PYZsq{}}\PY{l+s+s1}{x [a.u.]}\PY{l+s+s1}{\PYZsq{}}\PY{p}{)}


\PY{k}{def}\PY{+w}{ }\PY{n+nf}{gauss}\PY{p}{(}\PY{n}{x}\PY{p}{,} \PY{n}{norm}\PY{p}{,} \PY{n}{mu}\PY{p}{,} \PY{n}{sigma}\PY{p}{)}\PY{p}{:}
\PY{+w}{    }\PY{l+s+sd}{\PYZdq{}\PYZdq{}\PYZdq{}Gaussian model.}
\PY{l+s+sd}{    \PYZdq{}\PYZdq{}\PYZdq{}}
    \PY{k}{return} \PY{n}{norm} \PY{o}{/}  \PY{n}{np}\PY{o}{.}\PY{n}{sqrt}\PY{p}{(}\PY{l+m+mi}{2} \PY{o}{*} \PY{n}{np}\PY{o}{.}\PY{n}{pi}\PY{p}{)} \PY{o}{/} \PY{n}{sigma} \PY{o}{*} \PY{n}{np}\PY{o}{.}\PY{n}{exp}\PY{p}{(}\PY{o}{\PYZhy{}}\PY{p}{(}\PY{n}{x} \PY{o}{\PYZhy{}} \PY{n}{mu}\PY{p}{)}\PY{o}{*}\PY{o}{*}\PY{l+m+mf}{2.} \PY{o}{/} \PY{p}{(}\PY{l+m+mf}{2.} \PY{o}{*} \PY{n}{sigma}\PY{o}{*}\PY{o}{*}\PY{l+m+mf}{2.}\PY{p}{)}\PY{p}{)}


\PY{k}{def}\PY{+w}{ }\PY{n+nf}{expo}\PY{p}{(}\PY{n}{x}\PY{p}{,} \PY{n}{norm}\PY{p}{,} \PY{n}{mean}\PY{p}{)}\PY{p}{:}
\PY{+w}{    }\PY{l+s+sd}{\PYZdq{}\PYZdq{}\PYZdq{}Exponential model.}
\PY{l+s+sd}{    \PYZdq{}\PYZdq{}\PYZdq{}}
    \PY{k}{return} \PY{n}{norm} \PY{o}{/} \PY{n}{mean} \PY{o}{*} \PY{n}{np}\PY{o}{.}\PY{n}{exp}\PY{p}{(}\PY{o}{\PYZhy{}}\PY{n}{x} \PY{o}{/} \PY{n}{mean}\PY{p}{)}


\PY{k}{def}\PY{+w}{ }\PY{n+nf}{dmodel}\PY{p}{(}\PY{n}{x}\PY{p}{,} \PY{n}{norm}\PY{p}{,} \PY{n}{loc}\PY{p}{)}\PY{p}{:}
\PY{+w}{    }\PY{l+s+sd}{\PYZdq{}\PYZdq{}\PYZdq{}}
\PY{l+s+sd}{    \PYZdq{}\PYZdq{}\PYZdq{}}
    \PY{k}{return} \PY{l+m+mf}{2.} \PY{o}{*} \PY{n}{norm} \PY{o}{/} \PY{n}{loc} \PY{o}{*} \PY{n}{x} \PY{o}{*} \PY{n}{np}\PY{o}{.}\PY{n}{exp}\PY{p}{(}\PY{o}{\PYZhy{}}\PY{n}{x}\PY{o}{*}\PY{o}{*}\PY{l+m+mf}{2.} \PY{o}{/} \PY{n}{loc}\PY{p}{)}





\PY{k}{if} \PY{n+nv+vm}{\PYZus{}\PYZus{}name\PYZus{}\PYZus{}} \PY{o}{==} \PY{l+s+s1}{\PYZsq{}}\PY{l+s+s1}{\PYZus{}\PYZus{}main\PYZus{}\PYZus{}}\PY{l+s+s1}{\PYZsq{}}\PY{p}{:}

    \PY{n}{num\PYZus{}realizations} \PY{o}{=} \PY{l+m+mi}{10000}
    \PY{n}{num\PYZus{}steps} \PY{o}{=} \PY{l+m+mi}{100}
    \PY{n}{x} \PY{o}{=} \PY{p}{[}\PY{p}{]}
    \PY{n}{y} \PY{o}{=} \PY{p}{[}\PY{p}{]}
    \PY{n}{d} \PY{o}{=} \PY{p}{[}\PY{p}{]}
    \PY{n}{d2} \PY{o}{=} \PY{p}{[}\PY{p}{]}
    \PY{k}{for} \PY{n}{\PYZus{}} \PY{o+ow}{in} \PY{n+nb}{range}\PY{p}{(}\PY{n}{num\PYZus{}realizations}\PY{p}{)}\PY{p}{:}
        \PY{n}{\PYZus{}x}\PY{p}{,} \PY{n}{\PYZus{}y} \PY{o}{=} \PY{n}{random\PYZus{}walk\PYZus{}2d\PYZus{}fast}\PY{p}{(}\PY{n}{num\PYZus{}steps}\PY{p}{)}
        \PY{n}{x}\PY{o}{.}\PY{n}{append}\PY{p}{(}\PY{n}{\PYZus{}x}\PY{p}{)}
        \PY{n}{y}\PY{o}{.}\PY{n}{append}\PY{p}{(}\PY{n}{\PYZus{}y}\PY{p}{)}
        \PY{n}{d}\PY{o}{.}\PY{n}{append}\PY{p}{(}\PY{n}{np}\PY{o}{.}\PY{n}{sqrt}\PY{p}{(}\PY{n}{\PYZus{}x}\PY{o}{*}\PY{o}{*}\PY{l+m+mi}{2} \PY{o}{+} \PY{n}{\PYZus{}y}\PY{o}{*}\PY{o}{*}\PY{l+m+mi}{2}\PY{p}{)}\PY{p}{)}
        \PY{n}{d2}\PY{o}{.}\PY{n}{append}\PY{p}{(}\PY{n}{\PYZus{}x}\PY{o}{*}\PY{o}{*}\PY{l+m+mi}{2} \PY{o}{+} \PY{n}{\PYZus{}y}\PY{o}{*}\PY{o}{*}\PY{l+m+mi}{2}\PY{p}{)}

    \PY{n}{binning} \PY{o}{=} \PY{n}{np}\PY{o}{.}\PY{n}{linspace}\PY{p}{(}\PY{o}{\PYZhy{}}\PY{l+m+mi}{25}\PY{p}{,} \PY{l+m+mi}{25}\PY{p}{,} \PY{l+m+mi}{51}\PY{p}{)}
    \PY{n}{bin\PYZus{}width} \PY{o}{=} \PY{n}{binning}\PY{p}{[}\PY{l+m+mi}{1}\PY{p}{]} \PY{o}{\PYZhy{}} \PY{n}{binning}\PY{p}{[}\PY{l+m+mi}{0}\PY{p}{]}
    \PY{n}{bin\PYZus{}centers} \PY{o}{=} \PY{n}{binning}\PY{p}{[}\PY{p}{:}\PY{o}{\PYZhy{}}\PY{l+m+mi}{1}\PY{p}{]} \PY{o}{+} \PY{n}{bin\PYZus{}width} \PY{o}{/} \PY{l+m+mf}{2.}
    \PY{n}{model} \PY{o}{=} \PY{k}{lambda} \PY{n}{x}\PY{p}{:} \PY{n}{gauss}\PY{p}{(}\PY{n}{x}\PY{p}{,} \PY{n}{num\PYZus{}realizations} \PY{o}{*} \PY{n}{bin\PYZus{}width}\PY{p}{,} \PY{l+m+mf}{0.}\PY{p}{,} \PY{n}{np}\PY{o}{.}\PY{n}{sqrt}\PY{p}{(}\PY{n}{num\PYZus{}steps} \PY{o}{/} \PY{l+m+mf}{2.}\PY{p}{)}\PY{p}{)}

    \PY{n}{plt}\PY{o}{.}\PY{n}{figure}\PY{p}{(}\PY{l+s+s1}{\PYZsq{}}\PY{l+s+s1}{x histogram}\PY{l+s+s1}{\PYZsq{}}\PY{p}{)}
    \PY{n}{obs}\PY{p}{,} \PY{n}{\PYZus{}}\PY{p}{,} \PY{n}{\PYZus{}} \PY{o}{=} \PY{n}{plt}\PY{o}{.}\PY{n}{hist}\PY{p}{(}\PY{n}{x}\PY{p}{,} \PY{n}{bins}\PY{o}{=}\PY{n}{binning}\PY{p}{)}
    \PY{n}{exp} \PY{o}{=} \PY{n}{model}\PY{p}{(}\PY{n}{bin\PYZus{}centers}\PY{p}{)}
    \PY{n}{mask} \PY{o}{=} \PY{n}{obs} \PY{o}{\PYZgt{}} \PY{l+m+mi}{0}
    \PY{n}{chisq} \PY{o}{=} \PY{p}{(}\PY{p}{(}\PY{n}{obs}\PY{p}{[}\PY{n}{mask}\PY{p}{]} \PY{o}{\PYZhy{}} \PY{n}{exp}\PY{p}{[}\PY{n}{mask}\PY{p}{]}\PY{p}{)}\PY{o}{*}\PY{o}{*}\PY{l+m+mi}{2} \PY{o}{/} \PY{n}{exp}\PY{p}{[}\PY{n}{mask}\PY{p}{]}\PY{p}{)}\PY{o}{.}\PY{n}{sum}\PY{p}{(}\PY{p}{)}
    \PY{n+nb}{print}\PY{p}{(}\PY{l+s+sa}{f}\PY{l+s+s1}{\PYZsq{}}\PY{l+s+s1}{x distribution, chisquare = }\PY{l+s+si}{\PYZob{}}\PY{n}{chisq}\PY{l+s+si}{\PYZcb{}}\PY{l+s+s1}{ / }\PY{l+s+si}{\PYZob{}}\PY{n}{mask}\PY{o}{.}\PY{n}{sum}\PY{p}{(}\PY{p}{)}\PY{l+s+si}{\PYZcb{}}\PY{l+s+s1}{ dof}\PY{l+s+s1}{\PYZsq{}}\PY{p}{)}
    \PY{n}{plt}\PY{o}{.}\PY{n}{plot}\PY{p}{(}\PY{n}{bin\PYZus{}centers}\PY{p}{,} \PY{n}{model}\PY{p}{(}\PY{n}{bin\PYZus{}centers}\PY{p}{)}\PY{p}{)}

    \PY{n}{plt}\PY{o}{.}\PY{n}{figure}\PY{p}{(}\PY{l+s+s1}{\PYZsq{}}\PY{l+s+s1}{y histogram}\PY{l+s+s1}{\PYZsq{}}\PY{p}{)}
    \PY{n}{obs}\PY{p}{,} \PY{n}{\PYZus{}}\PY{p}{,} \PY{n}{\PYZus{}} \PY{o}{=} \PY{n}{plt}\PY{o}{.}\PY{n}{hist}\PY{p}{(}\PY{n}{y}\PY{p}{,} \PY{n}{bins}\PY{o}{=}\PY{n}{binning}\PY{p}{)}
    \PY{n}{exp} \PY{o}{=} \PY{n}{model}\PY{p}{(}\PY{n}{bin\PYZus{}centers}\PY{p}{)}
    \PY{n}{mask} \PY{o}{=} \PY{n}{obs} \PY{o}{\PYZgt{}} \PY{l+m+mi}{0}
    \PY{n}{chisq} \PY{o}{=} \PY{p}{(}\PY{p}{(}\PY{n}{obs}\PY{p}{[}\PY{n}{mask}\PY{p}{]} \PY{o}{\PYZhy{}} \PY{n}{exp}\PY{p}{[}\PY{n}{mask}\PY{p}{]}\PY{p}{)}\PY{o}{*}\PY{o}{*}\PY{l+m+mi}{2} \PY{o}{/} \PY{n}{exp}\PY{p}{[}\PY{n}{mask}\PY{p}{]}\PY{p}{)}\PY{o}{.}\PY{n}{sum}\PY{p}{(}\PY{p}{)}
    \PY{n+nb}{print}\PY{p}{(}\PY{l+s+sa}{f}\PY{l+s+s1}{\PYZsq{}}\PY{l+s+s1}{y distribution, chisquare = }\PY{l+s+si}{\PYZob{}}\PY{n}{chisq}\PY{l+s+si}{\PYZcb{}}\PY{l+s+s1}{ / }\PY{l+s+si}{\PYZob{}}\PY{n}{mask}\PY{o}{.}\PY{n}{sum}\PY{p}{(}\PY{p}{)}\PY{l+s+si}{\PYZcb{}}\PY{l+s+s1}{ dof}\PY{l+s+s1}{\PYZsq{}}\PY{p}{)}
    \PY{n}{plt}\PY{o}{.}\PY{n}{plot}\PY{p}{(}\PY{n}{bin\PYZus{}centers}\PY{p}{,} \PY{n}{model}\PY{p}{(}\PY{n}{bin\PYZus{}centers}\PY{p}{)}\PY{p}{)}

    \PY{n}{binning} \PY{o}{=} \PY{n}{np}\PY{o}{.}\PY{n}{linspace}\PY{p}{(}\PY{l+m+mi}{0}\PY{p}{,} \PY{l+m+mi}{1000}\PY{p}{,} \PY{l+m+mi}{101}\PY{p}{)}
    \PY{n}{bin\PYZus{}width} \PY{o}{=} \PY{n}{binning}\PY{p}{[}\PY{l+m+mi}{1}\PY{p}{]} \PY{o}{\PYZhy{}} \PY{n}{binning}\PY{p}{[}\PY{l+m+mi}{0}\PY{p}{]}
    \PY{n}{bin\PYZus{}centers} \PY{o}{=} \PY{n}{binning}\PY{p}{[}\PY{p}{:}\PY{o}{\PYZhy{}}\PY{l+m+mi}{1}\PY{p}{]} \PY{o}{+} \PY{n}{bin\PYZus{}width} \PY{o}{/} \PY{l+m+mf}{2.}
    \PY{n}{model} \PY{o}{=} \PY{k}{lambda} \PY{n}{x}\PY{p}{:} \PY{n}{expo}\PY{p}{(}\PY{n}{x}\PY{p}{,} \PY{n}{num\PYZus{}realizations} \PY{o}{*} \PY{n}{bin\PYZus{}width}\PY{p}{,} \PY{n}{num\PYZus{}steps}\PY{p}{)}

    \PY{n}{plt}\PY{o}{.}\PY{n}{figure}\PY{p}{(}\PY{l+s+s1}{\PYZsq{}}\PY{l+s+s1}{d2 histogram}\PY{l+s+s1}{\PYZsq{}}\PY{p}{)}
    \PY{n}{obs}\PY{p}{,} \PY{n}{\PYZus{}}\PY{p}{,} \PY{n}{\PYZus{}} \PY{o}{=} \PY{n}{plt}\PY{o}{.}\PY{n}{hist}\PY{p}{(}\PY{n}{d2}\PY{p}{,} \PY{n}{bins}\PY{o}{=}\PY{n}{binning}\PY{p}{)}
    \PY{n}{exp} \PY{o}{=} \PY{n}{model}\PY{p}{(}\PY{n}{bin\PYZus{}centers}\PY{p}{)}
    \PY{n}{mask} \PY{o}{=} \PY{n}{obs} \PY{o}{\PYZgt{}} \PY{l+m+mi}{0}
    \PY{n}{chisq} \PY{o}{=} \PY{p}{(}\PY{p}{(}\PY{n}{obs}\PY{p}{[}\PY{n}{mask}\PY{p}{]} \PY{o}{\PYZhy{}} \PY{n}{exp}\PY{p}{[}\PY{n}{mask}\PY{p}{]}\PY{p}{)}\PY{o}{*}\PY{o}{*}\PY{l+m+mi}{2} \PY{o}{/} \PY{n}{exp}\PY{p}{[}\PY{n}{mask}\PY{p}{]}\PY{p}{)}\PY{o}{.}\PY{n}{sum}\PY{p}{(}\PY{p}{)}
    \PY{n+nb}{print}\PY{p}{(}\PY{l+s+sa}{f}\PY{l+s+s1}{\PYZsq{}}\PY{l+s+s1}{d\PYZca{}2 distribution, chisquare = }\PY{l+s+si}{\PYZob{}}\PY{n}{chisq}\PY{l+s+si}{\PYZcb{}}\PY{l+s+s1}{ / }\PY{l+s+si}{\PYZob{}}\PY{n}{mask}\PY{o}{.}\PY{n}{sum}\PY{p}{(}\PY{p}{)}\PY{l+s+si}{\PYZcb{}}\PY{l+s+s1}{ dof}\PY{l+s+s1}{\PYZsq{}}\PY{p}{)}
    \PY{n}{plt}\PY{o}{.}\PY{n}{plot}\PY{p}{(}\PY{n}{bin\PYZus{}centers}\PY{p}{,} \PY{n}{model}\PY{p}{(}\PY{n}{bin\PYZus{}centers}\PY{p}{)}\PY{p}{)}

    \PY{n}{binning} \PY{o}{=} \PY{n}{np}\PY{o}{.}\PY{n}{linspace}\PY{p}{(}\PY{l+m+mi}{0}\PY{p}{,} \PY{l+m+mi}{30}\PY{p}{,} \PY{l+m+mi}{51}\PY{p}{)}
    \PY{n}{bin\PYZus{}width} \PY{o}{=} \PY{n}{binning}\PY{p}{[}\PY{l+m+mi}{1}\PY{p}{]} \PY{o}{\PYZhy{}} \PY{n}{binning}\PY{p}{[}\PY{l+m+mi}{0}\PY{p}{]}
    \PY{n}{bin\PYZus{}centers} \PY{o}{=} \PY{n}{binning}\PY{p}{[}\PY{p}{:}\PY{o}{\PYZhy{}}\PY{l+m+mi}{1}\PY{p}{]} \PY{o}{+} \PY{n}{bin\PYZus{}width} \PY{o}{/} \PY{l+m+mf}{2.}
    \PY{n}{model} \PY{o}{=} \PY{k}{lambda} \PY{n}{x}\PY{p}{:} \PY{n}{dmodel}\PY{p}{(}\PY{n}{x}\PY{p}{,} \PY{n}{num\PYZus{}realizations} \PY{o}{*} \PY{n}{bin\PYZus{}width}\PY{p}{,} \PY{n}{num\PYZus{}steps}\PY{p}{)}

    \PY{n}{plt}\PY{o}{.}\PY{n}{figure}\PY{p}{(}\PY{l+s+s1}{\PYZsq{}}\PY{l+s+s1}{d histogram}\PY{l+s+s1}{\PYZsq{}}\PY{p}{)}
    \PY{n}{obs}\PY{p}{,} \PY{n}{\PYZus{}}\PY{p}{,} \PY{n}{\PYZus{}} \PY{o}{=} \PY{n}{plt}\PY{o}{.}\PY{n}{hist}\PY{p}{(}\PY{n}{d}\PY{p}{,} \PY{n}{bins}\PY{o}{=}\PY{n}{binning}\PY{p}{)}
    \PY{n}{exp} \PY{o}{=} \PY{n}{model}\PY{p}{(}\PY{n}{bin\PYZus{}centers}\PY{p}{)}
    \PY{n}{mask} \PY{o}{=} \PY{n}{obs} \PY{o}{\PYZgt{}} \PY{l+m+mi}{0}
    \PY{n}{chisq} \PY{o}{=} \PY{p}{(}\PY{p}{(}\PY{n}{obs}\PY{p}{[}\PY{n}{mask}\PY{p}{]} \PY{o}{\PYZhy{}} \PY{n}{exp}\PY{p}{[}\PY{n}{mask}\PY{p}{]}\PY{p}{)}\PY{o}{*}\PY{o}{*}\PY{l+m+mi}{2} \PY{o}{/} \PY{n}{exp}\PY{p}{[}\PY{n}{mask}\PY{p}{]}\PY{p}{)}\PY{o}{.}\PY{n}{sum}\PY{p}{(}\PY{p}{)}
    \PY{n+nb}{print}\PY{p}{(}\PY{l+s+sa}{f}\PY{l+s+s1}{\PYZsq{}}\PY{l+s+s1}{d\PYZca{}2 distribution, chisquare = }\PY{l+s+si}{\PYZob{}}\PY{n}{chisq}\PY{l+s+si}{\PYZcb{}}\PY{l+s+s1}{ / }\PY{l+s+si}{\PYZob{}}\PY{n}{mask}\PY{o}{.}\PY{n}{sum}\PY{p}{(}\PY{p}{)}\PY{l+s+si}{\PYZcb{}}\PY{l+s+s1}{ dof}\PY{l+s+s1}{\PYZsq{}}\PY{p}{)}
    \PY{n}{plt}\PY{o}{.}\PY{n}{plot}\PY{p}{(}\PY{n}{bin\PYZus{}centers}\PY{p}{,} \PY{n}{model}\PY{p}{(}\PY{n}{bin\PYZus{}centers}\PY{p}{)}\PY{p}{)}

    \PY{n}{num\PYZus{}realizations} \PY{o}{=} \PY{l+m+mi}{1000}
    \PY{n}{N} \PY{o}{=} \PY{p}{(}\PY{l+m+mi}{10}\PY{p}{,} \PY{l+m+mi}{25}\PY{p}{,} \PY{l+m+mi}{50}\PY{p}{,} \PY{l+m+mi}{100}\PY{p}{,} \PY{l+m+mi}{250}\PY{p}{,} \PY{l+m+mi}{500}\PY{p}{,} \PY{l+m+mi}{1000}\PY{p}{)}
    \PY{n}{d} \PY{o}{=} \PY{p}{[}\PY{p}{]}
    \PY{k}{for} \PY{n}{num\PYZus{}steps} \PY{o+ow}{in} \PY{n}{N}\PY{p}{:}
        \PY{n}{\PYZus{}d} \PY{o}{=} \PY{l+m+mf}{0.}
        \PY{k}{for} \PY{n}{\PYZus{}} \PY{o+ow}{in} \PY{n+nb}{range}\PY{p}{(}\PY{n}{num\PYZus{}realizations}\PY{p}{)}\PY{p}{:}
            \PY{n}{\PYZus{}x}\PY{p}{,} \PY{n}{\PYZus{}y} \PY{o}{=} \PY{n}{random\PYZus{}walk\PYZus{}2d\PYZus{}fast}\PY{p}{(}\PY{n}{num\PYZus{}steps}\PY{p}{)}
            \PY{n}{\PYZus{}d} \PY{o}{+}\PY{o}{=} \PY{n}{np}\PY{o}{.}\PY{n}{sqrt}\PY{p}{(}\PY{n}{\PYZus{}x}\PY{o}{*}\PY{o}{*}\PY{l+m+mi}{2} \PY{o}{+} \PY{n}{\PYZus{}y}\PY{o}{*}\PY{o}{*}\PY{l+m+mi}{2}\PY{p}{)}
        \PY{n}{d}\PY{o}{.}\PY{n}{append}\PY{p}{(}\PY{n}{\PYZus{}d} \PY{o}{/} \PY{n}{num\PYZus{}realizations}\PY{p}{)}
    \PY{n}{plt}\PY{o}{.}\PY{n}{figure}\PY{p}{(}\PY{l+s+s1}{\PYZsq{}}\PY{l+s+s1}{scaling}\PY{l+s+s1}{\PYZsq{}}\PY{p}{)}
    \PY{n}{plt}\PY{o}{.}\PY{n}{plot}\PY{p}{(}\PY{n}{N}\PY{p}{,} \PY{n}{d}\PY{p}{,} \PY{l+s+s1}{\PYZsq{}}\PY{l+s+s1}{o}\PY{l+s+s1}{\PYZsq{}}\PY{p}{)}
    \PY{n}{grid} \PY{o}{=} \PY{n}{np}\PY{o}{.}\PY{n}{linspace}\PY{p}{(}\PY{l+m+mf}{0.}\PY{p}{,} \PY{l+m+mf}{1000.}\PY{p}{,} \PY{l+m+mi}{100}\PY{p}{)}
    \PY{n}{plt}\PY{o}{.}\PY{n}{plot}\PY{p}{(}\PY{n}{grid}\PY{p}{,} \PY{n}{np}\PY{o}{.}\PY{n}{sqrt}\PY{p}{(}\PY{n}{np}\PY{o}{.}\PY{n}{pi}\PY{p}{)} \PY{o}{/} \PY{l+m+mi}{2} \PY{o}{*} \PY{n}{np}\PY{o}{.}\PY{n}{sqrt}\PY{p}{(}\PY{n}{grid}\PY{p}{)}\PY{p}{)}
    \PY{n}{plt}\PY{o}{.}\PY{n}{xlabel}\PY{p}{(}\PY{l+s+s1}{\PYZsq{}}\PY{l+s+s1}{\PYZdl{}N\PYZdl{}}\PY{l+s+s1}{\PYZsq{}}\PY{p}{)}
    \PY{n}{plt}\PY{o}{.}\PY{n}{ylabel}\PY{p}{(}\PY{l+s+s1}{\PYZsq{}}\PY{l+s+s1}{\PYZdl{}\PYZlt{}d\PYZgt{}\PYZdl{}}\PY{l+s+s1}{\PYZsq{}}\PY{p}{)}

    \PY{n}{plot\PYZus{}random\PYZus{}walk\PYZus{}2d}\PY{p}{(}\PY{l+m+mi}{1000}\PY{p}{)}

    \PY{n}{plt}\PY{o}{.}\PY{n}{show}\PY{p}{(}\PY{p}{)}
\end{Verbatim}



\end{article}
\end{document}
