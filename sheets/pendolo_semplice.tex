\documentclass{lab1-article}

\title{Pendolo semplice}

\usepackage{fancyvrb}
\input{code/python}

\begin{document}


\begin{article}
\selectlanguage{italian}

\maketitle

\secsummary
Studio della dipendenza del periodo di un pendolo dalla
lunghezza, dalla massa sospesa e dall'ampiezza delle oscillazioni.


\secmaterials

\begin{itemize}
\item 3 solidi regolari di massa diversa, dotati di gancio.
\item Cronometro (risoluzione 0.01~s).
\item Bilancia di precisione (risoluzione 1~mg).
\item Metro a nastro (risoluzione 1~mm).
\item Calibro ventesimale (risoluzione 0.05~mm).
\end{itemize}


\secdefinitions

\begin{figure}[htb!]
  \begin{tikzpicture}[scale=1]
    \pgfmathsetmacro{\xc}{2.25}
    \pgfmathsetmacro{\yc}{0}
    \pgfmathsetmacro{\r}{4.5}
    \pgfmathsetmacro{\rangle}{1}
    \pgfmathsetmacro{\thetazero}{40}
    \node at (0, 0) {};
    \draw[style=densely dashed] (\xc, \yc) -- (\xc, \yc - \r);
    \draw (\xc, \yc) -- (\xc + \r*sin{\thetazero}, \yc - \r*cos{\thetazero});
    \draw[style=densely dashed] (\xc, \yc - \r) arc (270:270+\thetazero:\r);
    \draw[style=densely dashed] (\xc, \yc - \r*cos{\thetazero}) --%
    (\xc + \r*sin{\thetazero}, \yc - \r*cos{\thetazero});
    \draw (\xc, \yc - \rangle) arc (270:270+\thetazero:\rangle);
    \fill (\xc + \r*sin{\thetazero}, \yc - \r*cos{\thetazero}) circle%
          [radius=0.15];
    \node at (\xc - 0.25, 0) {$0$};
    \node[anchor=east] at (\xc , \yc - 0.5*\r) {$l\cos\theta_0$};
    \node[anchor=south] at%
    (\xc + 0.5*\r*sin{\thetazero}, \yc - \r*cos{\thetazero}) {$l\sin\theta_0$};
    \node at (\xc + 0.5*\r*sin{\thetazero} + 0.5, \yc - 0.5*\r*cos{\thetazero})%
          {$l$};
    \node at (\xc + \r*sin{\thetazero} + 0.5, \yc - \r*cos{\thetazero}) {$m$};
    \node at (\xc + 0.6, \yc - 1.3) {$\theta_0$};
  \end{tikzpicture}
  \caption{Schematizzazione dell'apparato sperimentale e definizioni di base.}
  \label{fig:pendolo}
\end{figure}

La lunghezza $l$ di un pendolo \`e la distanza tra il punto di sospensione ed
il centro di massa del pendolo stesso.

L'ampiezza di oscillazione $\theta_0$ \`e l'angolo formato dal filo con la
verticale all'inizio dell'oscillazione.

Il periodo di oscillazione $T$ di un pendolo \`e il tempo che esso impiega a
compiere un'oscillazione completa di andata e ritorno rispetto al punto di
partenza.

L'espressione per il periodo del pendolo si pu\`o sviluppare in serie come
\begin{align}
T = 2\pi\sqrt{\frac{l}{g}} \left( 1 + \frac{1}{16}\theta_0^2 +
\frac{11}{3072}\theta_0^4 + \cdots \right)
\end{align}
(nel limite di piccole oscillazioni solo il primo termine dello sviluppo
\`e importante ed il periodo \`e indipendente dall'ampiezza).


\secmeasurements


\labsubsection{Dipendenza del periodo dalla massa}

Decisa una lunghezza per il pendolo, a ampiezza di oscillazione costante si
misuri il periodo di oscillazione usando le tre masse a disposizione.

Si riportino le misure ottenute in una tabella: i dati mostrano una dipendenza
del periodo al variare della massa?


\labsubsection{Dipendenza del periodo dall'ampiezza}

Fissata la massa, si misuri il periodo di oscillazione aumentando
progressivamente l'ampiezza iniziale $\theta_0$.
Si consigliano almeno 5 ampiezze diverse (ad esempio 10, 20, 30, 40 e 50 gradi
circa).

Si riportino le misure in una tabella e si disegni il grafico del periodo in
funzione dell'ampiezza. C'\`e dipendenza del periodo al variare dell'ampiezza
di oscillazione?


\labsubsection{Dipendenza del periodo dalla lunghezza}

Con massa e ampiezza fissata, si misuri il periodo di oscillazione al variare
della lunghezza. Si consigliano almeno cinque lunghezze diverse (ad esempio da
30 a 100~cm circa).

Si riportino le misure in una tabella ed in un grafico in carta bilogaritmica.
Si faccia un fit al calcolatore con una legge di potenza e se ne stimino i
parametri.

\secconsiderations

\labsubsection{Misura del periodo}

Anche se la risoluzione del cronometro usato vale 0.01~s, \`e illusorio
pensare che questo sia l'errore di misura da attribuire a misurazioni di tempo
manuali.

Per ridurre l'impatto del tempo di reazione, si consiglia
di misurare il tempo $\tau$ che il sistema impiega a compiere $10$
oscillazioni complete. Per stimare l'errore associato a $\tau$ si ripeta la
misure $n$ volte (con $n \geq 5$); il valor medio delle varie misure effettuate
\begin{align}
  m_{\tau} = \frac{1}{n}\sum_{i = 1}^n \tau_i
\end{align}
sar\`a assunto come miglior stima della durata temporale del fenomeno e la
deviazione standard della media
\begin{align}
  s_\tau = \sqrt{\frac{1}{n(n-1)}\sum_{i = 1}^n(\tau_i - m_\tau)^2}
\end{align}
come errore associato.
Va da s\'e che si passa da $\tau$ a $T$ dividendo per $10$ (se abbiamo
misurato $10$ oscillazioni complete) sia la misura che l'errore.

\onecolumn

\documentclass{lab1-article}

\title{Pendolo semplice}

\usepackage{fancyvrb}
\input{code/python}

\begin{document}


\begin{article}
\selectlanguage{italian}

\maketitle

\secsummary
Studio della dipendenza del periodo di un pendolo dalla
lunghezza, dalla massa sospesa e dall'ampiezza delle oscillazioni.


\secmaterials

\begin{itemize}
\item 3 solidi regolari di massa diversa, dotati di gancio.
\item Cronometro (risoluzione 0.01~s).
\item Bilancia di precisione (risoluzione 1~mg).
\item Metro a nastro (risoluzione 1~mm).
\item Calibro ventesimale (risoluzione 0.05~mm).
\end{itemize}


\secdefinitions

\begin{figure}[htb!]
  \begin{tikzpicture}[scale=1]
    \pgfmathsetmacro{\xc}{2.25}
    \pgfmathsetmacro{\yc}{0}
    \pgfmathsetmacro{\r}{4.5}
    \pgfmathsetmacro{\rangle}{1}
    \pgfmathsetmacro{\thetazero}{40}
    \node at (0, 0) {};
    \draw[style=densely dashed] (\xc, \yc) -- (\xc, \yc - \r);
    \draw (\xc, \yc) -- (\xc + \r*sin{\thetazero}, \yc - \r*cos{\thetazero});
    \draw[style=densely dashed] (\xc, \yc - \r) arc (270:270+\thetazero:\r);
    \draw[style=densely dashed] (\xc, \yc - \r*cos{\thetazero}) --%
    (\xc + \r*sin{\thetazero}, \yc - \r*cos{\thetazero});
    \draw (\xc, \yc - \rangle) arc (270:270+\thetazero:\rangle);
    \fill (\xc + \r*sin{\thetazero}, \yc - \r*cos{\thetazero}) circle%
          [radius=0.15];
    \node at (\xc - 0.25, 0) {$0$};
    \node[anchor=east] at (\xc , \yc - 0.5*\r) {$l\cos\theta_0$};
    \node[anchor=south] at%
    (\xc + 0.5*\r*sin{\thetazero}, \yc - \r*cos{\thetazero}) {$l\sin\theta_0$};
    \node at (\xc + 0.5*\r*sin{\thetazero} + 0.5, \yc - 0.5*\r*cos{\thetazero})%
          {$l$};
    \node at (\xc + \r*sin{\thetazero} + 0.5, \yc - \r*cos{\thetazero}) {$m$};
    \node at (\xc + 0.6, \yc - 1.3) {$\theta_0$};
  \end{tikzpicture}
  \caption{Schematizzazione dell'apparato sperimentale e definizioni di base.}
  \label{fig:pendolo}
\end{figure}

La lunghezza $l$ di un pendolo \`e la distanza tra il punto di sospensione ed
il centro di massa del pendolo stesso.

L'ampiezza di oscillazione $\theta_0$ \`e l'angolo formato dal filo con la
verticale all'inizio dell'oscillazione.

Il periodo di oscillazione $T$ di un pendolo \`e il tempo che esso impiega a
compiere un'oscillazione completa di andata e ritorno rispetto al punto di
partenza.

L'espressione per il periodo del pendolo si pu\`o sviluppare in serie come
\begin{align}
T = 2\pi\sqrt{\frac{l}{g}} \left( 1 + \frac{1}{16}\theta_0^2 +
\frac{11}{3072}\theta_0^4 + \cdots \right)
\end{align}
(nel limite di piccole oscillazioni solo il primo termine dello sviluppo
\`e importante ed il periodo \`e indipendente dall'ampiezza).


\secmeasurements


\labsubsection{Dipendenza del periodo dalla massa}

Decisa una lunghezza per il pendolo, a ampiezza di oscillazione costante si
misuri il periodo di oscillazione usando le tre masse a disposizione.

Si riportino le misure ottenute in una tabella: i dati mostrano una dipendenza
del periodo al variare della massa?


\labsubsection{Dipendenza del periodo dall'ampiezza}

Fissata la massa, si misuri il periodo di oscillazione aumentando
progressivamente l'ampiezza iniziale $\theta_0$.
Si consigliano almeno 5 ampiezze diverse (ad esempio 10, 20, 30, 40 e 50 gradi
circa).

Si riportino le misure in una tabella e si disegni il grafico del periodo in
funzione dell'ampiezza. C'\`e dipendenza del periodo al variare dell'ampiezza
di oscillazione?


\labsubsection{Dipendenza del periodo dalla lunghezza}

Con massa e ampiezza fissata, si misuri il periodo di oscillazione al variare
della lunghezza. Si consigliano almeno cinque lunghezze diverse (ad esempio da
30 a 100~cm circa).

Si riportino le misure in una tabella ed in un grafico in carta bilogaritmica.
Si faccia un fit al calcolatore con una legge di potenza e se ne stimino i
parametri.

\secconsiderations

\labsubsection{Misura del periodo}

Anche se la risoluzione del cronometro usato vale 0.01~s, \`e illusorio
pensare che questo sia l'errore di misura da attribuire a misurazioni di tempo
manuali.

Per ridurre l'impatto del tempo di reazione, si consiglia
di misurare il tempo $\tau$ che il sistema impiega a compiere $10$
oscillazioni complete. Per stimare l'errore associato a $\tau$ si ripeta la
misure $n$ volte (con $n \geq 5$); il valor medio delle varie misure effettuate
\begin{align}
  m_{\tau} = \frac{1}{n}\sum_{i = 1}^n \tau_i
\end{align}
sar\`a assunto come miglior stima della durata temporale del fenomeno e la
deviazione standard della media
\begin{align}
  s_\tau = \sqrt{\frac{1}{n(n-1)}\sum_{i = 1}^n(\tau_i - m_\tau)^2}
\end{align}
come errore associato.
Va da s\'e che si passa da $\tau$ a $T$ dividendo per $10$ (se abbiamo
misurato $10$ oscillazioni complete) sia la misura che l'errore.

\onecolumn

\documentclass{lab1-article}

\title{Pendolo semplice}

\usepackage{fancyvrb}
\input{code/python}

\begin{document}


\begin{article}
\selectlanguage{italian}

\maketitle

\secsummary
Studio della dipendenza del periodo di un pendolo dalla
lunghezza, dalla massa sospesa e dall'ampiezza delle oscillazioni.


\secmaterials

\begin{itemize}
\item 3 solidi regolari di massa diversa, dotati di gancio.
\item Cronometro (risoluzione 0.01~s).
\item Bilancia di precisione (risoluzione 1~mg).
\item Metro a nastro (risoluzione 1~mm).
\item Calibro ventesimale (risoluzione 0.05~mm).
\end{itemize}


\secdefinitions

\begin{figure}[htb!]
  \begin{tikzpicture}[scale=1]
    \pgfmathsetmacro{\xc}{2.25}
    \pgfmathsetmacro{\yc}{0}
    \pgfmathsetmacro{\r}{4.5}
    \pgfmathsetmacro{\rangle}{1}
    \pgfmathsetmacro{\thetazero}{40}
    \node at (0, 0) {};
    \draw[style=densely dashed] (\xc, \yc) -- (\xc, \yc - \r);
    \draw (\xc, \yc) -- (\xc + \r*sin{\thetazero}, \yc - \r*cos{\thetazero});
    \draw[style=densely dashed] (\xc, \yc - \r) arc (270:270+\thetazero:\r);
    \draw[style=densely dashed] (\xc, \yc - \r*cos{\thetazero}) --%
    (\xc + \r*sin{\thetazero}, \yc - \r*cos{\thetazero});
    \draw (\xc, \yc - \rangle) arc (270:270+\thetazero:\rangle);
    \fill (\xc + \r*sin{\thetazero}, \yc - \r*cos{\thetazero}) circle%
          [radius=0.15];
    \node at (\xc - 0.25, 0) {$0$};
    \node[anchor=east] at (\xc , \yc - 0.5*\r) {$l\cos\theta_0$};
    \node[anchor=south] at%
    (\xc + 0.5*\r*sin{\thetazero}, \yc - \r*cos{\thetazero}) {$l\sin\theta_0$};
    \node at (\xc + 0.5*\r*sin{\thetazero} + 0.5, \yc - 0.5*\r*cos{\thetazero})%
          {$l$};
    \node at (\xc + \r*sin{\thetazero} + 0.5, \yc - \r*cos{\thetazero}) {$m$};
    \node at (\xc + 0.6, \yc - 1.3) {$\theta_0$};
  \end{tikzpicture}
  \caption{Schematizzazione dell'apparato sperimentale e definizioni di base.}
  \label{fig:pendolo}
\end{figure}

La lunghezza $l$ di un pendolo \`e la distanza tra il punto di sospensione ed
il centro di massa del pendolo stesso.

L'ampiezza di oscillazione $\theta_0$ \`e l'angolo formato dal filo con la
verticale all'inizio dell'oscillazione.

Il periodo di oscillazione $T$ di un pendolo \`e il tempo che esso impiega a
compiere un'oscillazione completa di andata e ritorno rispetto al punto di
partenza.

L'espressione per il periodo del pendolo si pu\`o sviluppare in serie come
\begin{align}
T = 2\pi\sqrt{\frac{l}{g}} \left( 1 + \frac{1}{16}\theta_0^2 +
\frac{11}{3072}\theta_0^4 + \cdots \right)
\end{align}
(nel limite di piccole oscillazioni solo il primo termine dello sviluppo
\`e importante ed il periodo \`e indipendente dall'ampiezza).


\secmeasurements


\labsubsection{Dipendenza del periodo dalla massa}

Decisa una lunghezza per il pendolo, a ampiezza di oscillazione costante si
misuri il periodo di oscillazione usando le tre masse a disposizione.

Si riportino le misure ottenute in una tabella: i dati mostrano una dipendenza
del periodo al variare della massa?


\labsubsection{Dipendenza del periodo dall'ampiezza}

Fissata la massa, si misuri il periodo di oscillazione aumentando
progressivamente l'ampiezza iniziale $\theta_0$.
Si consigliano almeno 5 ampiezze diverse (ad esempio 10, 20, 30, 40 e 50 gradi
circa).

Si riportino le misure in una tabella e si disegni il grafico del periodo in
funzione dell'ampiezza. C'\`e dipendenza del periodo al variare dell'ampiezza
di oscillazione?


\labsubsection{Dipendenza del periodo dalla lunghezza}

Con massa e ampiezza fissata, si misuri il periodo di oscillazione al variare
della lunghezza. Si consigliano almeno cinque lunghezze diverse (ad esempio da
30 a 100~cm circa).

Si riportino le misure in una tabella ed in un grafico in carta bilogaritmica.
Si faccia un fit al calcolatore con una legge di potenza e se ne stimino i
parametri.

\secconsiderations

\labsubsection{Misura del periodo}

Anche se la risoluzione del cronometro usato vale 0.01~s, \`e illusorio
pensare che questo sia l'errore di misura da attribuire a misurazioni di tempo
manuali.

Per ridurre l'impatto del tempo di reazione, si consiglia
di misurare il tempo $\tau$ che il sistema impiega a compiere $10$
oscillazioni complete. Per stimare l'errore associato a $\tau$ si ripeta la
misure $n$ volte (con $n \geq 5$); il valor medio delle varie misure effettuate
\begin{align}
  m_{\tau} = \frac{1}{n}\sum_{i = 1}^n \tau_i
\end{align}
sar\`a assunto come miglior stima della durata temporale del fenomeno e la
deviazione standard della media
\begin{align}
  s_\tau = \sqrt{\frac{1}{n(n-1)}\sum_{i = 1}^n(\tau_i - m_\tau)^2}
\end{align}
come errore associato.
Va da s\'e che si passa da $\tau$ a $T$ dividendo per $10$ (se abbiamo
misurato $10$ oscillazioni complete) sia la misura che l'errore.

\onecolumn

\documentclass{lab1-article}

\title{Pendolo semplice}

\usepackage{fancyvrb}
\input{code/python}

\begin{document}


\begin{article}
\selectlanguage{italian}

\maketitle

\secsummary
Studio della dipendenza del periodo di un pendolo dalla
lunghezza, dalla massa sospesa e dall'ampiezza delle oscillazioni.


\secmaterials

\begin{itemize}
\item 3 solidi regolari di massa diversa, dotati di gancio.
\item Cronometro (risoluzione 0.01~s).
\item Bilancia di precisione (risoluzione 1~mg).
\item Metro a nastro (risoluzione 1~mm).
\item Calibro ventesimale (risoluzione 0.05~mm).
\end{itemize}


\secdefinitions

\begin{figure}[htb!]
  \begin{tikzpicture}[scale=1]
    \pgfmathsetmacro{\xc}{2.25}
    \pgfmathsetmacro{\yc}{0}
    \pgfmathsetmacro{\r}{4.5}
    \pgfmathsetmacro{\rangle}{1}
    \pgfmathsetmacro{\thetazero}{40}
    \node at (0, 0) {};
    \draw[style=densely dashed] (\xc, \yc) -- (\xc, \yc - \r);
    \draw (\xc, \yc) -- (\xc + \r*sin{\thetazero}, \yc - \r*cos{\thetazero});
    \draw[style=densely dashed] (\xc, \yc - \r) arc (270:270+\thetazero:\r);
    \draw[style=densely dashed] (\xc, \yc - \r*cos{\thetazero}) --%
    (\xc + \r*sin{\thetazero}, \yc - \r*cos{\thetazero});
    \draw (\xc, \yc - \rangle) arc (270:270+\thetazero:\rangle);
    \fill (\xc + \r*sin{\thetazero}, \yc - \r*cos{\thetazero}) circle%
          [radius=0.15];
    \node at (\xc - 0.25, 0) {$0$};
    \node[anchor=east] at (\xc , \yc - 0.5*\r) {$l\cos\theta_0$};
    \node[anchor=south] at%
    (\xc + 0.5*\r*sin{\thetazero}, \yc - \r*cos{\thetazero}) {$l\sin\theta_0$};
    \node at (\xc + 0.5*\r*sin{\thetazero} + 0.5, \yc - 0.5*\r*cos{\thetazero})%
          {$l$};
    \node at (\xc + \r*sin{\thetazero} + 0.5, \yc - \r*cos{\thetazero}) {$m$};
    \node at (\xc + 0.6, \yc - 1.3) {$\theta_0$};
  \end{tikzpicture}
  \caption{Schematizzazione dell'apparato sperimentale e definizioni di base.}
  \label{fig:pendolo}
\end{figure}

La lunghezza $l$ di un pendolo \`e la distanza tra il punto di sospensione ed
il centro di massa del pendolo stesso.

L'ampiezza di oscillazione $\theta_0$ \`e l'angolo formato dal filo con la
verticale all'inizio dell'oscillazione.

Il periodo di oscillazione $T$ di un pendolo \`e il tempo che esso impiega a
compiere un'oscillazione completa di andata e ritorno rispetto al punto di
partenza.

L'espressione per il periodo del pendolo si pu\`o sviluppare in serie come
\begin{align}
T = 2\pi\sqrt{\frac{l}{g}} \left( 1 + \frac{1}{16}\theta_0^2 +
\frac{11}{3072}\theta_0^4 + \cdots \right)
\end{align}
(nel limite di piccole oscillazioni solo il primo termine dello sviluppo
\`e importante ed il periodo \`e indipendente dall'ampiezza).


\secmeasurements


\labsubsection{Dipendenza del periodo dalla massa}

Decisa una lunghezza per il pendolo, a ampiezza di oscillazione costante si
misuri il periodo di oscillazione usando le tre masse a disposizione.

Si riportino le misure ottenute in una tabella: i dati mostrano una dipendenza
del periodo al variare della massa?


\labsubsection{Dipendenza del periodo dall'ampiezza}

Fissata la massa, si misuri il periodo di oscillazione aumentando
progressivamente l'ampiezza iniziale $\theta_0$.
Si consigliano almeno 5 ampiezze diverse (ad esempio 10, 20, 30, 40 e 50 gradi
circa).

Si riportino le misure in una tabella e si disegni il grafico del periodo in
funzione dell'ampiezza. C'\`e dipendenza del periodo al variare dell'ampiezza
di oscillazione?


\labsubsection{Dipendenza del periodo dalla lunghezza}

Con massa e ampiezza fissata, si misuri il periodo di oscillazione al variare
della lunghezza. Si consigliano almeno cinque lunghezze diverse (ad esempio da
30 a 100~cm circa).

Si riportino le misure in una tabella ed in un grafico in carta bilogaritmica.
Si faccia un fit al calcolatore con una legge di potenza e se ne stimino i
parametri.

\secconsiderations

\labsubsection{Misura del periodo}

Anche se la risoluzione del cronometro usato vale 0.01~s, \`e illusorio
pensare che questo sia l'errore di misura da attribuire a misurazioni di tempo
manuali.

Per ridurre l'impatto del tempo di reazione, si consiglia
di misurare il tempo $\tau$ che il sistema impiega a compiere $10$
oscillazioni complete. Per stimare l'errore associato a $\tau$ si ripeta la
misure $n$ volte (con $n \geq 5$); il valor medio delle varie misure effettuate
\begin{align}
  m_{\tau} = \frac{1}{n}\sum_{i = 1}^n \tau_i
\end{align}
sar\`a assunto come miglior stima della durata temporale del fenomeno e la
deviazione standard della media
\begin{align}
  s_\tau = \sqrt{\frac{1}{n(n-1)}\sum_{i = 1}^n(\tau_i - m_\tau)^2}
\end{align}
come errore associato.
Va da s\'e che si passa da $\tau$ a $T$ dividendo per $10$ (se abbiamo
misurato $10$ oscillazioni complete) sia la misura che l'errore.

\onecolumn

\input{code/pendolo_semplice}

\end{article}
\end{document}


\end{article}
\end{document}


\end{article}
\end{document}


\end{article}
\end{document}
