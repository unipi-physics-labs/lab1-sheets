\documentclass{lab1-article}

\title{Lente cilindrica}

\usepackage{fancyvrb}
\usepackage{hyperref}
\input{code/python}

\begin{document}


\begin{article}
\selectlanguage{italian}

\maketitle

\secintro


Lo scopo dell'esperienza \`e la misura della lunghezza focale di una lente
cilindrica.


\secmaterialsdad

\begin{itemize}
\item Un bicchiere di forma cilindrica.
\item Metro a nastro.
\item Smartphone.
\end{itemize}


\secmeasurements

In questa esperienza realizzeremo una misura di lunghezza focale con materiale
che dovreste avere tutti a disposizione a casa: un bicchiere pieno d'acqua
utilizzato come lente cilindrica, il led dello \emph{smartphone} utilizzato
come sorgente luminosa ed un qualsiasi oggetto piano e monocolore, e.g., un
blocco di carta, utilizzato come schermo.

\begin{figure}[!htb]
  \includegraphics[width=\linewidth]{figures/lente_cilindrica}
  \caption{Esempio di apparato sperimentale per la misura della focale di una
  lente sferica.}
\end{figure}

Realizzate un \emph{setup} sperimentale, con il materiale a disposizione,
simile a quello in figura. Dovrete misurare una serie di coppie di valori di
$p$ (distanza della sorgente dal centro della lente) e $q$ (distanza del centro
della lente dall'immagine a fuoco), per cui mettetevi nelle condizioni di farlo
agevolmente. (Potete tenere il bicchiere immobile, create un supporto per lo
\emph{smartphone} con un nasino e muovere lo schermo con la mano, ma sentitevi
liberi di dare sfogo alla vostra creativit\`a.)

Fate un grafico di dispersione di $1/p$ vs. $1/q$ ed usate la legge dei punti
coniugati
\begin{align}
  \frac{1}{p} + \frac{1}{q} = \frac{1}{f}
\end{align}
per ricavare il valore della lughezza focale $f$ attraverso un \emph{fit} di tipo
generale. Per completezza, fatto il cambio di variabile $x = 1/p$ e $y = 1/q$,
i punti dovrebbero disporsi su una retta
\begin{align*}
  y = \frac{1}{f} - x
\end{align*}
in cui la lunghezza focale \`e l'inverso dell'intercetta ed il coefficiente
angolare dovrebbe essere compatibile con $-1$. Inutile dire che, in questo caso,
con ogni probabilit\`a le incertezze sulla variabile indipendente non saranno
trascurabili.

\begin{figure}[htb!]
\begin{center}
  \begin{tikzpicture}[scale=2]
    %\draw (0, 0) -- (1,1);
    \draw (0.5, 0) -- (4,0);
    \draw[->] (2, 0) -- (2,1);
    \draw[->] (2, 0) -- (2,-1);
    \draw [fill] (1.5,0) circle [radius=0.02];
    \draw [fill] (2.5,0) circle [radius=0.02];
    \draw[thick, ->] (1.25, 0) -- (1.25,0.5); % object
    \draw (1.25,0.5) -- (2, 0.5) -- (3.5, -1); %()
    \draw (1.25,0.5) -- (3.5, -1); %(2, 0)
    \draw[thick, ->] (3.5, 0) -- (3.5,-1); % image
    % notation
    \draw[<->] (1.25, -1.25) -- (2, -1.25);
    \node [below] at (1.625,-1.25) {$p$};
    \draw[<->] (2, -1.25) -- (3.5,-1.25);
    \node [below] at (2.75,-1.25) {$q$};
  \end{tikzpicture}
  \caption{Schema ottico per la misura delle focale di una lente convergente.}
  \label{fig:convergente}
\end{center}
\end{figure}


Confrontate il valore di $f$ da voi ottenuto con il valore previsto dalla
teoria per una lente spessa nella nostra configurazione
\begin{align}
  \frac{1}{f} =(n-1) \frac{2}{r} \left(1 - \frac{(n-1)}{n} \right)
\end{align}
(dove $r$ \`e il raggio del bicchiere e $n$ l'indice di rifrazione dell'acqua).


\secconsiderations

Assicuratevi che il bicchiere che utilizzate abbia una forma cilindrica---in caso
contrario la modellizzazione diventa pi\`u complicata.

Quando misurerete la distanza $q$ dalla lente all'immagine vi accorgerete che
non \`e banale capire quando quest'ultima \`e esattamente a fuoco. Tenetelo
in considerazione nella stima dell'incertezza.

\end{article}
\end{document}
