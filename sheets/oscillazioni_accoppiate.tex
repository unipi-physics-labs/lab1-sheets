\documentclass{lab1-article}

\title{Oscillazioni accoppiate}

\newcommand{\plasduinodoctext}[1]%
{
  Una volta acceso il calcolatore, selezionare dal men\`u principale
  (in alto a sinistra) \menuitem{Application}~$\rightarrow$~%
  \menuitem{Education}~$\rightarrow$~\menuitem{plasduino}. Questo dovrebbe
  mostrare la finestra principale del programma di acquisizione. Per questa
  esperienza, tra la lista dei moduli, lanciate~\menuitem{#1} (doppio
  click sulla linea corrispondente, oppure selezionate la linea stessa e
  premete \menuitem{Open}).
}


\newcommand{\plasduinodoc}[1]%
{
  \labsection{Note sul programma di acquisizione}

  \plasduinodoctext{#1}
}

\newcommand{\plasduinosave}%
{
  Di norma al termine di ogni sessione di presa dati il programma vi chiede se
  volete salvare una copia del \emph{file} dei dati in una cartella a vostra
  scelta (il che pu\`o essere comodo per l'analisi successiva). Se questa
  funzionalit\`a dovesse essere disabilitata potete ri-abilitarla 
  attraverso il men\`u di plasduino \menuitem{Configuration}~%
  $\rightarrow$~\menuitem{Change settings}: nella finestra che si apre
  selezionate il tab~\menuitem{daq} e abilitate l'opzione
  \menuitem{prompt-save-dialog}.
}


\begin{document}


\begin{article}
\selectlanguage{italian}

\maketitle

\secsummary

Lo scopo di questa esperienza \`e lo studio del moto di due pendoli accoppiati
e, in particolare, del fenomeno dei \emph{battimenti}.


\secmaterials

\begin{itemize}
\item Due pendoli accoppiati attraverso una molla;
\item due smorzatori (galleggianti da pesca);
\item sistema di acquisizione per registrare la posizione di ciascun pendolo
  in funzione del tempo.
\end{itemize}


\secmeasurements

Si considerino due pendoli uguali realizzati con due aste rigide collegate tra
loro da una molla. Sebbene, in generale, il moto del sistema possa essere
molto complicato, vi sono due configurazioni iniziali (corrispondenti ai
cosiddetti \emph{modi normali} di oscillazione) per cui il moto di entrambi i
pendoli \`e armonico: quella in cui essi si muovono in fase e quella
in cui essi si muovono in controfase.


\labsubsection{Oscillazioni di un pendolo singolo}

Si metta in oscillazione un pendolo da solo, senza lo smorzatore, se ne
misuri la pulsazione angolare $\omega_0$ e si confronti il valore con quello
previsto dalla teoria
\begin{align}
  \omega_0 = \sqrt{\frac{mgl}{I}}.
\end{align}
Si aggiunga adesso lo smorzatore e si ripeta la misura, stimando il tempo
di decadimento $\tau$ dell'ampiezza di oscillazione
\begin{align}
  \theta_0(t) = \theta_0(0) e^{-t/\tau}.
\end{align}
Il periodo \`e cambiato significativamente?


\labsubsection{Oscillazioni in fase e controfase}

Si pu\`o eccitare il sistema in modo che oscilli secondo l'uno o l'altro dei due
modi normali: per avere il primo basta spostare i pendoli nello stesso verso,
di uguali ampiezze e lasciarli andare contemporaneamente. In queste condizioni
la molla, non sollecitata, non influenzer\`a il movimento dei due pendoli, che
oscilleranno (sincroni) alla frequenza alla quale oscillerebbe un pendolo non
accoppiato. Se invece i pendoli vengono spostati di ampiezze uguali ma in verso
opposto si hanno oscillazioni in controfase. 

Si misurino le pulsazioni angolari $\omega_{\rm f}$ ed $\omega_{\rm c}$ delle
oscillazioni in fase e controfase. Si verifichi che
$\omega_{\rm f} \sim \omega_0$ e che $\omega_{\rm c} > \omega_{\rm f}$
(se la costante elastica $k$ della molla \`e piccola, la differenza non sar\`a
grande).


\labsubsection{Battimenti}

Se si sposta uno dei due pendoli tenendo l'altro fermo nella sua posizione
di equilibrio e si lascia oscillare il sistema con questa configurazione
iniziale, il moto risultante \`e dato dalla somma (con uguali ampiezze)
dei due modi normali
\begin{align}
  x(t) = A_0 \left[ \cos(\omega_{\rm f}t + \phi_1) +
    \cos(\omega_{\rm c}t + \phi_2)\right]
\end{align}
o, per le formule di prostaferesi
\begin{align}
  x(t) & = 2A_0 \left\{
    \cos\left[\frac{(\omega_{\rm c} + \omega_{\rm f})t}{2} +
      \frac{(\phi_2 + \phi_1)}{2} \right] \cdot \right. \nonumber\\
     & \left. \cos\left[\frac{(\omega_{\rm c} - \omega_{\rm f})t}{2} +
      \frac{(\phi_2 - \phi_1)}{2} \right]
    \right\}.
\end{align}
Fisicamente l'oscillazione risultante, di pulsazione angolare portante
\begin{align}
  \omega_{\rm p} = \frac{(\omega_{\rm c} + \omega_{\rm f})}{2}
  \approx \omega_{\rm c},\omega_{\rm f}
\end{align}
\`e \emph{modulata} da un'onda sinusoidale di pulsazione angolare
$\omega_{\rm b}$ molto pi\`u piccola
\begin{align}
  \omega_{\rm b} = \frac{(\omega_{\rm c} - \omega_{\rm f})}{2} \ll
  \omega_{\rm c},\omega_{\rm f}
\end{align}
(e, di conseguenza, di periodo molto pi\`u grande).

Si misurino le pulsazioni angolari portante e modulante (o dei battimenti)
e si confrontino i valori ottenuti con la teoria, sulla base dei valori
di $\omega_{\rm f}$ e $\omega_{\rm c}$ misurati prima.


\secconsiderations

Una volta terminata l'acquisizione \`e possibile \emph{zooomare} con il
puntatore del \emph{mouse} (tenendo premuto il tasto sinistro) su un rettangolo
generico del grafico.
(Si torna alla visualizzazione intera premendo il tasto destro del mouse
con il puntatore posizionato su un punto qualsiasi del grafico.)
Accanto al puntatore vengono visualizzate le coordinate della posizione
corrente, per cui differenze di tempo e posizione possono essere 
misurate direttamente sul grafico---e questo, in linea di principio, \`e
sufficiente per l'esperienza.

\subsecdataformat

In aggiunta, il programma di acquisizione fornisce un \emph{file} di uscita
contenente quattro colonne che rappresentano, rispettivamente:
\begin{enumerate}
\item il tempo $t_1$, dall'inizio dell'acquisizione, a cui \`e stata campionata
  la posizione del primo pendolo;
\item la posizione del primo pendolo all'istante $t_1$;
\item il tempo $t_2$, dall'inizio dell'acquisizione, a cui \`e stata campionata
  la posizione del secondo pendolo;
\item la posizione del secondo pendolo all'istante $t_2$;
\end{enumerate}
(notate che, per come \`e implementato il sistema di acquisizione, le posizioni
dei due pendoli non sono misurate allo stesso istante).

Il \emph{file} in uscita pu\`o essere utilizzato per un'analisi pi\`u
dettagliata del moto del sistema.


\plasduinodoc{Pendulum View}

\plasduinosave

\end{article}
\end{document}
