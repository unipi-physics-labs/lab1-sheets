\documentclass{lab1-article}

\title{Conducibilit\`a termica}

\usepackage{fancyvrb}
\usepackage{hyperref}
\makeatletter
\def\PY@reset{\let\PY@it=\relax \let\PY@bf=\relax%
    \let\PY@ul=\relax \let\PY@tc=\relax%
    \let\PY@bc=\relax \let\PY@ff=\relax}
\def\PY@tok#1{\csname PY@tok@#1\endcsname}
\def\PY@toks#1+{\ifx\relax#1\empty\else%
    \PY@tok{#1}\expandafter\PY@toks\fi}
\def\PY@do#1{\PY@bc{\PY@tc{\PY@ul{%
    \PY@it{\PY@bf{\PY@ff{#1}}}}}}}
\def\PY#1#2{\PY@reset\PY@toks#1+\relax+\PY@do{#2}}

\expandafter\def\csname PY@tok@gd\endcsname{\def\PY@tc##1{\textcolor[rgb]{0.63,0.00,0.00}{##1}}}
\expandafter\def\csname PY@tok@gu\endcsname{\let\PY@bf=\textbf\def\PY@tc##1{\textcolor[rgb]{0.50,0.00,0.50}{##1}}}
\expandafter\def\csname PY@tok@gt\endcsname{\def\PY@tc##1{\textcolor[rgb]{0.00,0.27,0.87}{##1}}}
\expandafter\def\csname PY@tok@gs\endcsname{\let\PY@bf=\textbf}
\expandafter\def\csname PY@tok@gr\endcsname{\def\PY@tc##1{\textcolor[rgb]{1.00,0.00,0.00}{##1}}}
\expandafter\def\csname PY@tok@cm\endcsname{\let\PY@it=\textit\def\PY@tc##1{\textcolor[rgb]{0.25,0.50,0.50}{##1}}}
\expandafter\def\csname PY@tok@vg\endcsname{\def\PY@tc##1{\textcolor[rgb]{0.10,0.09,0.49}{##1}}}
\expandafter\def\csname PY@tok@m\endcsname{\def\PY@tc##1{\textcolor[rgb]{0.40,0.40,0.40}{##1}}}
\expandafter\def\csname PY@tok@mh\endcsname{\def\PY@tc##1{\textcolor[rgb]{0.40,0.40,0.40}{##1}}}
\expandafter\def\csname PY@tok@go\endcsname{\def\PY@tc##1{\textcolor[rgb]{0.53,0.53,0.53}{##1}}}
\expandafter\def\csname PY@tok@ge\endcsname{\let\PY@it=\textit}
\expandafter\def\csname PY@tok@vc\endcsname{\def\PY@tc##1{\textcolor[rgb]{0.10,0.09,0.49}{##1}}}
\expandafter\def\csname PY@tok@il\endcsname{\def\PY@tc##1{\textcolor[rgb]{0.40,0.40,0.40}{##1}}}
\expandafter\def\csname PY@tok@cs\endcsname{\let\PY@it=\textit\def\PY@tc##1{\textcolor[rgb]{0.25,0.50,0.50}{##1}}}
\expandafter\def\csname PY@tok@cp\endcsname{\def\PY@tc##1{\textcolor[rgb]{0.74,0.48,0.00}{##1}}}
\expandafter\def\csname PY@tok@gi\endcsname{\def\PY@tc##1{\textcolor[rgb]{0.00,0.63,0.00}{##1}}}
\expandafter\def\csname PY@tok@gh\endcsname{\let\PY@bf=\textbf\def\PY@tc##1{\textcolor[rgb]{0.00,0.00,0.50}{##1}}}
\expandafter\def\csname PY@tok@ni\endcsname{\let\PY@bf=\textbf\def\PY@tc##1{\textcolor[rgb]{0.60,0.60,0.60}{##1}}}
\expandafter\def\csname PY@tok@nl\endcsname{\def\PY@tc##1{\textcolor[rgb]{0.63,0.63,0.00}{##1}}}
\expandafter\def\csname PY@tok@nn\endcsname{\let\PY@bf=\textbf\def\PY@tc##1{\textcolor[rgb]{0.00,0.00,1.00}{##1}}}
\expandafter\def\csname PY@tok@no\endcsname{\def\PY@tc##1{\textcolor[rgb]{0.53,0.00,0.00}{##1}}}
\expandafter\def\csname PY@tok@na\endcsname{\def\PY@tc##1{\textcolor[rgb]{0.49,0.56,0.16}{##1}}}
\expandafter\def\csname PY@tok@nb\endcsname{\def\PY@tc##1{\textcolor[rgb]{0.00,0.50,0.00}{##1}}}
\expandafter\def\csname PY@tok@nc\endcsname{\let\PY@bf=\textbf\def\PY@tc##1{\textcolor[rgb]{0.00,0.00,1.00}{##1}}}
\expandafter\def\csname PY@tok@nd\endcsname{\def\PY@tc##1{\textcolor[rgb]{0.67,0.13,1.00}{##1}}}
\expandafter\def\csname PY@tok@ne\endcsname{\let\PY@bf=\textbf\def\PY@tc##1{\textcolor[rgb]{0.82,0.25,0.23}{##1}}}
\expandafter\def\csname PY@tok@nf\endcsname{\def\PY@tc##1{\textcolor[rgb]{0.00,0.00,1.00}{##1}}}
\expandafter\def\csname PY@tok@si\endcsname{\let\PY@bf=\textbf\def\PY@tc##1{\textcolor[rgb]{0.73,0.40,0.53}{##1}}}
\expandafter\def\csname PY@tok@s2\endcsname{\def\PY@tc##1{\textcolor[rgb]{0.73,0.13,0.13}{##1}}}
\expandafter\def\csname PY@tok@vi\endcsname{\def\PY@tc##1{\textcolor[rgb]{0.10,0.09,0.49}{##1}}}
\expandafter\def\csname PY@tok@nt\endcsname{\let\PY@bf=\textbf\def\PY@tc##1{\textcolor[rgb]{0.00,0.50,0.00}{##1}}}
\expandafter\def\csname PY@tok@nv\endcsname{\def\PY@tc##1{\textcolor[rgb]{0.10,0.09,0.49}{##1}}}
\expandafter\def\csname PY@tok@s1\endcsname{\def\PY@tc##1{\textcolor[rgb]{0.73,0.13,0.13}{##1}}}
\expandafter\def\csname PY@tok@sh\endcsname{\def\PY@tc##1{\textcolor[rgb]{0.73,0.13,0.13}{##1}}}
\expandafter\def\csname PY@tok@sc\endcsname{\def\PY@tc##1{\textcolor[rgb]{0.73,0.13,0.13}{##1}}}
\expandafter\def\csname PY@tok@sx\endcsname{\def\PY@tc##1{\textcolor[rgb]{0.00,0.50,0.00}{##1}}}
\expandafter\def\csname PY@tok@bp\endcsname{\def\PY@tc##1{\textcolor[rgb]{0.00,0.50,0.00}{##1}}}
\expandafter\def\csname PY@tok@c1\endcsname{\let\PY@it=\textit\def\PY@tc##1{\textcolor[rgb]{0.25,0.50,0.50}{##1}}}
\expandafter\def\csname PY@tok@kc\endcsname{\let\PY@bf=\textbf\def\PY@tc##1{\textcolor[rgb]{0.00,0.50,0.00}{##1}}}
\expandafter\def\csname PY@tok@c\endcsname{\let\PY@it=\textit\def\PY@tc##1{\textcolor[rgb]{0.25,0.50,0.50}{##1}}}
\expandafter\def\csname PY@tok@mf\endcsname{\def\PY@tc##1{\textcolor[rgb]{0.40,0.40,0.40}{##1}}}
\expandafter\def\csname PY@tok@err\endcsname{\def\PY@bc##1{\setlength{\fboxsep}{0pt}\fcolorbox[rgb]{1.00,0.00,0.00}{1,1,1}{\strut ##1}}}
\expandafter\def\csname PY@tok@kd\endcsname{\let\PY@bf=\textbf\def\PY@tc##1{\textcolor[rgb]{0.00,0.50,0.00}{##1}}}
\expandafter\def\csname PY@tok@ss\endcsname{\def\PY@tc##1{\textcolor[rgb]{0.10,0.09,0.49}{##1}}}
\expandafter\def\csname PY@tok@sr\endcsname{\def\PY@tc##1{\textcolor[rgb]{0.73,0.40,0.53}{##1}}}
\expandafter\def\csname PY@tok@mo\endcsname{\def\PY@tc##1{\textcolor[rgb]{0.40,0.40,0.40}{##1}}}
\expandafter\def\csname PY@tok@kn\endcsname{\let\PY@bf=\textbf\def\PY@tc##1{\textcolor[rgb]{0.00,0.50,0.00}{##1}}}
\expandafter\def\csname PY@tok@mi\endcsname{\def\PY@tc##1{\textcolor[rgb]{0.40,0.40,0.40}{##1}}}
\expandafter\def\csname PY@tok@gp\endcsname{\let\PY@bf=\textbf\def\PY@tc##1{\textcolor[rgb]{0.00,0.00,0.50}{##1}}}
\expandafter\def\csname PY@tok@o\endcsname{\def\PY@tc##1{\textcolor[rgb]{0.40,0.40,0.40}{##1}}}
\expandafter\def\csname PY@tok@kr\endcsname{\let\PY@bf=\textbf\def\PY@tc##1{\textcolor[rgb]{0.00,0.50,0.00}{##1}}}
\expandafter\def\csname PY@tok@s\endcsname{\def\PY@tc##1{\textcolor[rgb]{0.73,0.13,0.13}{##1}}}
\expandafter\def\csname PY@tok@kp\endcsname{\def\PY@tc##1{\textcolor[rgb]{0.00,0.50,0.00}{##1}}}
\expandafter\def\csname PY@tok@w\endcsname{\def\PY@tc##1{\textcolor[rgb]{0.73,0.73,0.73}{##1}}}
\expandafter\def\csname PY@tok@kt\endcsname{\def\PY@tc##1{\textcolor[rgb]{0.69,0.00,0.25}{##1}}}
\expandafter\def\csname PY@tok@ow\endcsname{\let\PY@bf=\textbf\def\PY@tc##1{\textcolor[rgb]{0.67,0.13,1.00}{##1}}}
\expandafter\def\csname PY@tok@sb\endcsname{\def\PY@tc##1{\textcolor[rgb]{0.73,0.13,0.13}{##1}}}
\expandafter\def\csname PY@tok@k\endcsname{\let\PY@bf=\textbf\def\PY@tc##1{\textcolor[rgb]{0.00,0.50,0.00}{##1}}}
\expandafter\def\csname PY@tok@se\endcsname{\let\PY@bf=\textbf\def\PY@tc##1{\textcolor[rgb]{0.73,0.40,0.13}{##1}}}
\expandafter\def\csname PY@tok@sd\endcsname{\let\PY@it=\textit\def\PY@tc##1{\textcolor[rgb]{0.73,0.13,0.13}{##1}}}

\def\PYZbs{\char`\\}
\def\PYZus{\char`\_}
\def\PYZob{\char`\{}
\def\PYZcb{\char`\}}
\def\PYZca{\char`\^}
\def\PYZam{\char`\&}
\def\PYZlt{\char`\<}
\def\PYZgt{\char`\>}
\def\PYZsh{\char`\#}
\def\PYZpc{\char`\%}
\def\PYZdl{\char`\$}
\def\PYZhy{\char`\-}
\def\PYZsq{\char`\'}
\def\PYZdq{\char`\"}
\def\PYZti{\char`\~}
% for compatibility with earlier versions
\def\PYZat{@}
\def\PYZlb{[}
\def\PYZrb{]}
\makeatother


\newcommand{\plasduinodoctext}[1]%
{
  Una volta acceso il calcolatore, selezionare dal men\`u principale
  (in alto a sinistra) \menuitem{Application}~$\rightarrow$~%
  \menuitem{Education}~$\rightarrow$~\menuitem{plasduino}. Questo dovrebbe
  mostrare la finestra principale del programma di acquisizione. Per questa
  esperienza, tra la lista dei moduli, lanciate~\menuitem{#1} (doppio
  click sulla linea corrispondente, oppure selezionate la linea stessa e
  premete \menuitem{Open}).
}


\newcommand{\plasduinodoc}[1]%
{
  \labsection{Note sul programma di acquisizione}

  \plasduinodoctext{#1}
}

\newcommand{\plasduinosave}%
{
  Di norma al termine di ogni sessione di presa dati il programma vi chiede se
  volete salvare una copia del \emph{file} dei dati in una cartella a vostra
  scelta (il che pu\`o essere comodo per l'analisi successiva). Se questa
  funzionalit\`a dovesse essere disabilitata potete ri-abilitarla 
  attraverso il men\`u di plasduino \menuitem{Configuration}~%
  $\rightarrow$~\menuitem{Change settings}: nella finestra che si apre
  selezionate il tab~\menuitem{daq} e abilitate l'opzione
  \menuitem{prompt-save-dialog}.
}



\begin{document}


\begin{article}
\selectlanguage{italian}

\maketitle

\secsummary
Abbiamo a disposizione due barre cilindriche (riscaldate da una resistenza
elettrica da una parte e raffreddate da un sistema di acqua corrente dall'altra)
e possiamo misurare la temperatura all'interno delle barre stesse in
funzione della distanza da un estremo.
Lo scopo dell'esperienza \`e quello di ricavare dalla misure effettuate
la conducibilit\`a termica del materiale.


\secmaterials

\begin{itemize}
\item Due barre cilindriche.
\item Due termistori per la misura di temperatura.
\item Calcolatore con programma di acquisizione.
\item Un alimentatore chiuso su due resistenze in paralello.
\item Un circuito di acqua corrente.
\end{itemize}


\secmeasurements

La quantit\`a di calore che si trasmette per conduzione nell'unit\`a di tempo
\begin{align}
  W = \frac{dQ}{dt}
\end{align}
\`e detta \emph{flusso di calore} e nel sistema MKS si misura in Watt.
All'equilibrio termico il flusso di calore attraverso la barra risulta costante
(se non ci sono perdite tutta l'energia erogata dalla resistenza non pu\`o che
essere assorbita dall'acqua all'altra estremit\`a). Se $S$ \`e la sezione
della barra, $T$ la temperatura ed $x$ la posizione lungo la barra, la
teoria prevede che
\begin{align}\label{eq:conduction}
  W = -\lambda S \frac{\Delta T}{\Delta x},
\end{align}
dove la costante di proporzionalit\`a $\lambda$ \`e detta conducibilit\`a
termica del materiale (dato che il calore si propaga verso zone a temperatura
pi\`u bassa, orientato l'asse $x$ nel verso in cui si propaga il calore,
$\Delta T/\Delta x$ risulta negativo ed \`e necessario il segno meno).


\labsubsection{Misure di temperatura}

Si misurino le temperature $T_i$ di una serie di punti interni della barra,
rispettivamente a distanza $x_i$ dall'estremo riscaldato. Applicando
la~\eqref{eq:conduction} e ponendo $\Delta x = x_i - x_0$ (e anzi
$\Delta x = x_i$, se si fissa l’origine delle ascisse in $x_0$) e
$\Delta T = T_i - T_0$ si avr\`a
\begin{align}
  W = -\lambda S \frac{(T_i - T_0)}{x_i},
\end{align}
ovvero
\begin{align}
  T_i = T_0 - \frac{W}{\lambda S} x_i
\end{align}
(cio\`e ci aspettiamo che la temperatura decresca linearmente dall'estremit\`a
pi\`u calda a quella pi\`u fredda).


\labsubsection{Stima della conducibilit\`a}

Sulla base delle misure effettuate si costruisca il grafico della temperatura
in funzione della posizione lungo ciascuna delle due barre.
Il coefficiente angolare della retta di \emph{best fit} rappresenta la nostra
stima della grandezza $-W/\lambda S$.

Per stimare $\lambda$ \`e necessario utilizzare il fatto che, in assenza di
perdite verso l'ambiente esterno, $W$ \`e il calore ceduto per effetto Joule
dalla resistenza. In termini della tensione $V$ e della corrente $I$ che si
leggono (rispettivamente in Volt e Ampere) sul display dell'alimentatore si ha
\begin{align}
  W = \frac{V I}{2}
\end{align}
(visto che si alimentano due resistenze in parallelo).


\plasduinodoc{Temperature Monitor}

\plasduinosave


\secappendix{valori tabulati}

Si riportano di seguito i valori indicativi di conducibilit\`a termica a
temperatura ambiente per alcuni materiali rilevanti per l'esperienza.
Nel sistema MKS la conducibilit\`a termica si misura in
W~m$^-1$~$^\circ$C$^{-1}$, ma spesso vengono utilizzate anche le unit\`a
ibride W~cm$^-1$~$^\circ$C$^{-1}$.

\medskip

\begin{center}
\begin{tabular}{p{0.65\linewidth}@{}p{0.3\linewidth}}
\hline
Materiale & $\lambda$ [W~m$^-1$~$^\circ$C$^{-1}$]\\
\hline
\hline
Alluminio & $\sim 200$\\
Rame & $\sim 400$\\
Ottone & $\sim 110$\\
\hline
\end{tabular}
\end{center}

\onecolumn

\begin{Verbatim}[label=\makebox{\href{https://github.com/unipi-physics-labs/lab1-sheets/tree/main/snippy/conducibilita_termica.py}{https://github.com/.../conducibilita\_termica.py}},commandchars=\\\{\}]
\PY{k+kn}{import}\PY{+w}{ }\PY{n+nn}{numpy}\PY{+w}{ }\PY{k}{as}\PY{+w}{ }\PY{n+nn}{np}
\PY{k+kn}{from}\PY{+w}{ }\PY{n+nn}{matplotlib}\PY{+w}{ }\PY{k+kn}{import} \PY{n}{pyplot} \PY{k}{as} \PY{n}{plt}
\PY{k+kn}{from}\PY{+w}{ }\PY{n+nn}{scipy}\PY{n+nn}{.}\PY{n+nn}{optimize}\PY{+w}{ }\PY{k+kn}{import} \PY{n}{curve\PYZus{}fit}

\PY{c+c1}{\PYZsh{} Misure dirette delle distanze e delle temperature (mettete i vostri numeri).}
\PY{c+c1}{\PYZsh{} Qui potete anche leggere i dati da file, usando il metodo np.loadtxt(),}
\PY{c+c1}{\PYZsh{} se lo trovate comodo.}
\PY{n}{x} \PY{o}{=} \PY{n}{np}\PY{o}{.}\PY{n}{array}\PY{p}{(}\PY{p}{[}\PY{l+m+mf}{5.0}\PY{p}{,} \PY{l+m+mf}{10.0}\PY{p}{,} \PY{l+m+mf}{15.0}\PY{p}{,} \PY{l+m+mf}{20.0}\PY{p}{,} \PY{l+m+mf}{25.0}\PY{p}{,} \PY{l+m+mf}{30.0}\PY{p}{,} \PY{l+m+mf}{35.0}\PY{p}{]}\PY{p}{)}
\PY{n}{sigma\PYZus{}x} \PY{o}{=} \PY{n}{np}\PY{o}{.}\PY{n}{full}\PY{p}{(}\PY{n}{x}\PY{o}{.}\PY{n}{shape}\PY{p}{,} \PY{l+m+mf}{0.1}\PY{p}{)}
\PY{n}{T} \PY{o}{=} \PY{n}{np}\PY{o}{.}\PY{n}{array}\PY{p}{(}\PY{p}{[}\PY{l+m+mf}{28.6}\PY{p}{,} \PY{l+m+mf}{27.8}\PY{p}{,} \PY{l+m+mf}{26.9}\PY{p}{,} \PY{l+m+mf}{25.9}\PY{p}{,} \PY{l+m+mf}{25.0}\PY{p}{,} \PY{l+m+mf}{24.0}\PY{p}{,} \PY{l+m+mf}{23.1}\PY{p}{]}\PY{p}{)}
\PY{n}{sigma\PYZus{}T} \PY{o}{=} \PY{n}{np}\PY{o}{.}\PY{n}{full}\PY{p}{(}\PY{n}{T}\PY{o}{.}\PY{n}{shape}\PY{p}{,} \PY{l+m+mf}{0.2}\PY{p}{)}

\PY{k}{def}\PY{+w}{ }\PY{n+nf}{line}\PY{p}{(}\PY{n}{x}\PY{p}{,} \PY{n}{m}\PY{p}{,} \PY{n}{q}\PY{p}{)}\PY{p}{:}
\PY{+w}{    }\PY{l+s+sd}{\PYZdq{}\PYZdq{}\PYZdq{}Modello di fit lineare.}
\PY{l+s+sd}{    \PYZdq{}\PYZdq{}\PYZdq{}}
    \PY{k}{return} \PY{n}{m} \PY{o}{*} \PY{n}{x} \PY{o}{+} \PY{n}{q}

\PY{n}{plt}\PY{o}{.}\PY{n}{figure}\PY{p}{(}\PY{l+s+s1}{\PYZsq{}}\PY{l+s+s1}{Grafico posizione\PYZhy{}temperatura}\PY{l+s+s1}{\PYZsq{}}\PY{p}{)}
\PY{c+c1}{\PYZsh{} Grafico dei punti sperimentali.}
\PY{n}{plt}\PY{o}{.}\PY{n}{errorbar}\PY{p}{(}\PY{n}{x}\PY{p}{,} \PY{n}{T}\PY{p}{,} \PY{n}{sigma\PYZus{}T}\PY{p}{,} \PY{n}{sigma\PYZus{}x}\PY{p}{,} \PY{n}{fmt}\PY{o}{=}\PY{l+s+s1}{\PYZsq{}}\PY{l+s+s1}{o}\PY{l+s+s1}{\PYZsq{}}\PY{p}{)}
\PY{c+c1}{\PYZsh{} Fit con una retta.}
\PY{n}{popt}\PY{p}{,} \PY{n}{pcov} \PY{o}{=} \PY{n}{curve\PYZus{}fit}\PY{p}{(}\PY{n}{line}\PY{p}{,} \PY{n}{x}\PY{p}{,} \PY{n}{T}\PY{p}{,} \PY{n}{sigma}\PY{o}{=}\PY{n}{sigma\PYZus{}T}\PY{p}{)}
\PY{n}{m\PYZus{}hat}\PY{p}{,} \PY{n}{q\PYZus{}hat} \PY{o}{=} \PY{n}{popt}
\PY{n}{sigma\PYZus{}m}\PY{p}{,} \PY{n}{sigma\PYZus{}q} \PY{o}{=} \PY{n}{np}\PY{o}{.}\PY{n}{sqrt}\PY{p}{(}\PY{n}{pcov}\PY{o}{.}\PY{n}{diagonal}\PY{p}{(}\PY{p}{)}\PY{p}{)}
\PY{n+nb}{print}\PY{p}{(}\PY{n}{m\PYZus{}hat}\PY{p}{,} \PY{n}{sigma\PYZus{}m}\PY{p}{,} \PY{n}{q\PYZus{}hat}\PY{p}{,} \PY{n}{sigma\PYZus{}q}\PY{p}{)}
\PY{c+c1}{\PYZsh{} Grafico del modello di best fit.}
\PY{n}{x} \PY{o}{=} \PY{n}{np}\PY{o}{.}\PY{n}{linspace}\PY{p}{(}\PY{l+m+mf}{0.}\PY{p}{,} \PY{l+m+mf}{40.}\PY{p}{,} \PY{l+m+mi}{100}\PY{p}{)}
\PY{n}{plt}\PY{o}{.}\PY{n}{plot}\PY{p}{(}\PY{n}{x}\PY{p}{,} \PY{n}{line}\PY{p}{(}\PY{n}{x}\PY{p}{,} \PY{n}{m\PYZus{}hat}\PY{p}{,} \PY{n}{q\PYZus{}hat}\PY{p}{)}\PY{p}{)}
\PY{c+c1}{\PYZsh{} Formattazione del grafico.}
\PY{n}{plt}\PY{o}{.}\PY{n}{xlabel}\PY{p}{(}\PY{l+s+s1}{\PYZsq{}}\PY{l+s+s1}{Posizione [cm]}\PY{l+s+s1}{\PYZsq{}}\PY{p}{)}
\PY{n}{plt}\PY{o}{.}\PY{n}{ylabel}\PY{p}{(}\PY{l+s+s1}{\PYZsq{}}\PY{l+s+s1}{Temperatura [\PYZdl{}\PYZca{}}\PY{l+s+se}{\PYZbs{}\PYZbs{}}\PY{l+s+s1}{circ\PYZdl{}C]}\PY{l+s+s1}{\PYZsq{}}\PY{p}{)}
\PY{n}{plt}\PY{o}{.}\PY{n}{grid}\PY{p}{(}\PY{n}{which}\PY{o}{=}\PY{l+s+s1}{\PYZsq{}}\PY{l+s+s1}{both}\PY{l+s+s1}{\PYZsq{}}\PY{p}{,} \PY{n}{ls}\PY{o}{=}\PY{l+s+s1}{\PYZsq{}}\PY{l+s+s1}{dashed}\PY{l+s+s1}{\PYZsq{}}\PY{p}{,} \PY{n}{color}\PY{o}{=}\PY{l+s+s1}{\PYZsq{}}\PY{l+s+s1}{gray}\PY{l+s+s1}{\PYZsq{}}\PY{p}{)}
\PY{n}{plt}\PY{o}{.}\PY{n}{savefig}\PY{p}{(}\PY{l+s+s1}{\PYZsq{}}\PY{l+s+s1}{posizione\PYZus{}temperatura.pdf}\PY{l+s+s1}{\PYZsq{}}\PY{p}{)}

\PY{c+c1}{\PYZsh{} A questo punto potete usare i risultati del fit per stimare la}
\PY{c+c1}{\PYZsh{} conducibilita` vera e propria...}

\PY{n}{plt}\PY{o}{.}\PY{n}{show}\PY{p}{(}\PY{p}{)}
\end{Verbatim}



\end{article}
\end{document}
