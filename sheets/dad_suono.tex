\documentclass{lab1-article}

\title{Misura della velocit\`a del suono}

\usepackage{fancyvrb}
\usepackage{hyperref}
\makeatletter
\def\PY@reset{\let\PY@it=\relax \let\PY@bf=\relax%
    \let\PY@ul=\relax \let\PY@tc=\relax%
    \let\PY@bc=\relax \let\PY@ff=\relax}
\def\PY@tok#1{\csname PY@tok@#1\endcsname}
\def\PY@toks#1+{\ifx\relax#1\empty\else%
    \PY@tok{#1}\expandafter\PY@toks\fi}
\def\PY@do#1{\PY@bc{\PY@tc{\PY@ul{%
    \PY@it{\PY@bf{\PY@ff{#1}}}}}}}
\def\PY#1#2{\PY@reset\PY@toks#1+\relax+\PY@do{#2}}

\expandafter\def\csname PY@tok@gd\endcsname{\def\PY@tc##1{\textcolor[rgb]{0.63,0.00,0.00}{##1}}}
\expandafter\def\csname PY@tok@gu\endcsname{\let\PY@bf=\textbf\def\PY@tc##1{\textcolor[rgb]{0.50,0.00,0.50}{##1}}}
\expandafter\def\csname PY@tok@gt\endcsname{\def\PY@tc##1{\textcolor[rgb]{0.00,0.27,0.87}{##1}}}
\expandafter\def\csname PY@tok@gs\endcsname{\let\PY@bf=\textbf}
\expandafter\def\csname PY@tok@gr\endcsname{\def\PY@tc##1{\textcolor[rgb]{1.00,0.00,0.00}{##1}}}
\expandafter\def\csname PY@tok@cm\endcsname{\let\PY@it=\textit\def\PY@tc##1{\textcolor[rgb]{0.25,0.50,0.50}{##1}}}
\expandafter\def\csname PY@tok@vg\endcsname{\def\PY@tc##1{\textcolor[rgb]{0.10,0.09,0.49}{##1}}}
\expandafter\def\csname PY@tok@m\endcsname{\def\PY@tc##1{\textcolor[rgb]{0.40,0.40,0.40}{##1}}}
\expandafter\def\csname PY@tok@mh\endcsname{\def\PY@tc##1{\textcolor[rgb]{0.40,0.40,0.40}{##1}}}
\expandafter\def\csname PY@tok@go\endcsname{\def\PY@tc##1{\textcolor[rgb]{0.53,0.53,0.53}{##1}}}
\expandafter\def\csname PY@tok@ge\endcsname{\let\PY@it=\textit}
\expandafter\def\csname PY@tok@vc\endcsname{\def\PY@tc##1{\textcolor[rgb]{0.10,0.09,0.49}{##1}}}
\expandafter\def\csname PY@tok@il\endcsname{\def\PY@tc##1{\textcolor[rgb]{0.40,0.40,0.40}{##1}}}
\expandafter\def\csname PY@tok@cs\endcsname{\let\PY@it=\textit\def\PY@tc##1{\textcolor[rgb]{0.25,0.50,0.50}{##1}}}
\expandafter\def\csname PY@tok@cp\endcsname{\def\PY@tc##1{\textcolor[rgb]{0.74,0.48,0.00}{##1}}}
\expandafter\def\csname PY@tok@gi\endcsname{\def\PY@tc##1{\textcolor[rgb]{0.00,0.63,0.00}{##1}}}
\expandafter\def\csname PY@tok@gh\endcsname{\let\PY@bf=\textbf\def\PY@tc##1{\textcolor[rgb]{0.00,0.00,0.50}{##1}}}
\expandafter\def\csname PY@tok@ni\endcsname{\let\PY@bf=\textbf\def\PY@tc##1{\textcolor[rgb]{0.60,0.60,0.60}{##1}}}
\expandafter\def\csname PY@tok@nl\endcsname{\def\PY@tc##1{\textcolor[rgb]{0.63,0.63,0.00}{##1}}}
\expandafter\def\csname PY@tok@nn\endcsname{\let\PY@bf=\textbf\def\PY@tc##1{\textcolor[rgb]{0.00,0.00,1.00}{##1}}}
\expandafter\def\csname PY@tok@no\endcsname{\def\PY@tc##1{\textcolor[rgb]{0.53,0.00,0.00}{##1}}}
\expandafter\def\csname PY@tok@na\endcsname{\def\PY@tc##1{\textcolor[rgb]{0.49,0.56,0.16}{##1}}}
\expandafter\def\csname PY@tok@nb\endcsname{\def\PY@tc##1{\textcolor[rgb]{0.00,0.50,0.00}{##1}}}
\expandafter\def\csname PY@tok@nc\endcsname{\let\PY@bf=\textbf\def\PY@tc##1{\textcolor[rgb]{0.00,0.00,1.00}{##1}}}
\expandafter\def\csname PY@tok@nd\endcsname{\def\PY@tc##1{\textcolor[rgb]{0.67,0.13,1.00}{##1}}}
\expandafter\def\csname PY@tok@ne\endcsname{\let\PY@bf=\textbf\def\PY@tc##1{\textcolor[rgb]{0.82,0.25,0.23}{##1}}}
\expandafter\def\csname PY@tok@nf\endcsname{\def\PY@tc##1{\textcolor[rgb]{0.00,0.00,1.00}{##1}}}
\expandafter\def\csname PY@tok@si\endcsname{\let\PY@bf=\textbf\def\PY@tc##1{\textcolor[rgb]{0.73,0.40,0.53}{##1}}}
\expandafter\def\csname PY@tok@s2\endcsname{\def\PY@tc##1{\textcolor[rgb]{0.73,0.13,0.13}{##1}}}
\expandafter\def\csname PY@tok@vi\endcsname{\def\PY@tc##1{\textcolor[rgb]{0.10,0.09,0.49}{##1}}}
\expandafter\def\csname PY@tok@nt\endcsname{\let\PY@bf=\textbf\def\PY@tc##1{\textcolor[rgb]{0.00,0.50,0.00}{##1}}}
\expandafter\def\csname PY@tok@nv\endcsname{\def\PY@tc##1{\textcolor[rgb]{0.10,0.09,0.49}{##1}}}
\expandafter\def\csname PY@tok@s1\endcsname{\def\PY@tc##1{\textcolor[rgb]{0.73,0.13,0.13}{##1}}}
\expandafter\def\csname PY@tok@sh\endcsname{\def\PY@tc##1{\textcolor[rgb]{0.73,0.13,0.13}{##1}}}
\expandafter\def\csname PY@tok@sc\endcsname{\def\PY@tc##1{\textcolor[rgb]{0.73,0.13,0.13}{##1}}}
\expandafter\def\csname PY@tok@sx\endcsname{\def\PY@tc##1{\textcolor[rgb]{0.00,0.50,0.00}{##1}}}
\expandafter\def\csname PY@tok@bp\endcsname{\def\PY@tc##1{\textcolor[rgb]{0.00,0.50,0.00}{##1}}}
\expandafter\def\csname PY@tok@c1\endcsname{\let\PY@it=\textit\def\PY@tc##1{\textcolor[rgb]{0.25,0.50,0.50}{##1}}}
\expandafter\def\csname PY@tok@kc\endcsname{\let\PY@bf=\textbf\def\PY@tc##1{\textcolor[rgb]{0.00,0.50,0.00}{##1}}}
\expandafter\def\csname PY@tok@c\endcsname{\let\PY@it=\textit\def\PY@tc##1{\textcolor[rgb]{0.25,0.50,0.50}{##1}}}
\expandafter\def\csname PY@tok@mf\endcsname{\def\PY@tc##1{\textcolor[rgb]{0.40,0.40,0.40}{##1}}}
\expandafter\def\csname PY@tok@err\endcsname{\def\PY@bc##1{\setlength{\fboxsep}{0pt}\fcolorbox[rgb]{1.00,0.00,0.00}{1,1,1}{\strut ##1}}}
\expandafter\def\csname PY@tok@kd\endcsname{\let\PY@bf=\textbf\def\PY@tc##1{\textcolor[rgb]{0.00,0.50,0.00}{##1}}}
\expandafter\def\csname PY@tok@ss\endcsname{\def\PY@tc##1{\textcolor[rgb]{0.10,0.09,0.49}{##1}}}
\expandafter\def\csname PY@tok@sr\endcsname{\def\PY@tc##1{\textcolor[rgb]{0.73,0.40,0.53}{##1}}}
\expandafter\def\csname PY@tok@mo\endcsname{\def\PY@tc##1{\textcolor[rgb]{0.40,0.40,0.40}{##1}}}
\expandafter\def\csname PY@tok@kn\endcsname{\let\PY@bf=\textbf\def\PY@tc##1{\textcolor[rgb]{0.00,0.50,0.00}{##1}}}
\expandafter\def\csname PY@tok@mi\endcsname{\def\PY@tc##1{\textcolor[rgb]{0.40,0.40,0.40}{##1}}}
\expandafter\def\csname PY@tok@gp\endcsname{\let\PY@bf=\textbf\def\PY@tc##1{\textcolor[rgb]{0.00,0.00,0.50}{##1}}}
\expandafter\def\csname PY@tok@o\endcsname{\def\PY@tc##1{\textcolor[rgb]{0.40,0.40,0.40}{##1}}}
\expandafter\def\csname PY@tok@kr\endcsname{\let\PY@bf=\textbf\def\PY@tc##1{\textcolor[rgb]{0.00,0.50,0.00}{##1}}}
\expandafter\def\csname PY@tok@s\endcsname{\def\PY@tc##1{\textcolor[rgb]{0.73,0.13,0.13}{##1}}}
\expandafter\def\csname PY@tok@kp\endcsname{\def\PY@tc##1{\textcolor[rgb]{0.00,0.50,0.00}{##1}}}
\expandafter\def\csname PY@tok@w\endcsname{\def\PY@tc##1{\textcolor[rgb]{0.73,0.73,0.73}{##1}}}
\expandafter\def\csname PY@tok@kt\endcsname{\def\PY@tc##1{\textcolor[rgb]{0.69,0.00,0.25}{##1}}}
\expandafter\def\csname PY@tok@ow\endcsname{\let\PY@bf=\textbf\def\PY@tc##1{\textcolor[rgb]{0.67,0.13,1.00}{##1}}}
\expandafter\def\csname PY@tok@sb\endcsname{\def\PY@tc##1{\textcolor[rgb]{0.73,0.13,0.13}{##1}}}
\expandafter\def\csname PY@tok@k\endcsname{\let\PY@bf=\textbf\def\PY@tc##1{\textcolor[rgb]{0.00,0.50,0.00}{##1}}}
\expandafter\def\csname PY@tok@se\endcsname{\let\PY@bf=\textbf\def\PY@tc##1{\textcolor[rgb]{0.73,0.40,0.13}{##1}}}
\expandafter\def\csname PY@tok@sd\endcsname{\let\PY@it=\textit\def\PY@tc##1{\textcolor[rgb]{0.73,0.13,0.13}{##1}}}

\def\PYZbs{\char`\\}
\def\PYZus{\char`\_}
\def\PYZob{\char`\{}
\def\PYZcb{\char`\}}
\def\PYZca{\char`\^}
\def\PYZam{\char`\&}
\def\PYZlt{\char`\<}
\def\PYZgt{\char`\>}
\def\PYZsh{\char`\#}
\def\PYZpc{\char`\%}
\def\PYZdl{\char`\$}
\def\PYZhy{\char`\-}
\def\PYZsq{\char`\'}
\def\PYZdq{\char`\"}
\def\PYZti{\char`\~}
% for compatibility with earlier versions
\def\PYZat{@}
\def\PYZlb{[}
\def\PYZrb{]}
\makeatother


\begin{document}


\begin{article}
\selectlanguage{italian}

\maketitle

\secintro

Il suono si propaga nell'ambiente con una velocit\`a caratteristica del mezzo
e delle propriet\`a di quest'ultimo (e.g., la temperatura)---la velocit\`a del
suono in aria a 20$^\circ$~C, ad esempio, \`e di circa $v_s = 343$~m~s$^{-1}$.

Lo scopo dell'esperienza \`e la \emph{progettazione} e l'esecuzione di un
semplice esperimento per la misura della velocit\`a del suono nell'aria.

\secmaterialsdad

\begin{itemize}
  \item Uno \emph{smartphone} da utilizzare come registratore di suoni per
    le misure di tempo;
  \item un metronomo (o un secondo \emph{smartphone} con una delle innumerevoli
    \emph{app} per metronomo a disposizione);
  \item calcolatore con un'istallazione del \emph{software} utilizzato in
    laboratorio.
\end{itemize}


\secmeasurements

Si tratta dell'ultima esperienza del secondo semestre, e l'idea \`e quella di
mettere a frutto tutto il bagaglio di conoscenze che avete acquisito durante
l'anno per una vera misura di Fisica.

Supponiamo di far partire il nostro metronomo con una frequenza $f_0$ e di
registrare con lo \emph{smartphone} una serie di battiti, producendo una
traccia audio da analizzare esattamente come avete fatto per l'esperienza dei
rimbalzi di una pallina. Se il registratore \`e in quiete rispetto al
metronomo, dovreste rivelare una serie di impulsi che distano tra loro
esattamente
\begin{align*}
  \Delta_T(0) = \frac{1}{f_0}
\end{align*}
(e.g., se il metronomo batte a 120 bpm, o 2~Hz, dovreste registrare un impulso
ogni mezzo secondo.)
Notiamo eplicitamente che in questo caso il tempo che il suono impiega a
propararsi tra la sorgente ed il registratore \`e lo stesso per tutti gli
impulsi, per cui \`e irrilevante nella misura delle differenze di tempo.

Supponiamo adesso di eseguire una nuova registrazione, ma in questo caso con
lo \emph{smartphone} in movimento rispetto al metronomo---o viceversa, \`e
esattamente la stessa cosa---con una velocit\`a costante $v$. In questo caso,
nell'intervallo di tempo tra due battiti successivi, il nostro registratore
si \`e spostato di una distanza
\begin{align*}
  \Delta_x = v \Delta_T(0) = \frac{v}{f_0},
\end{align*}
e la distanza temporale \emph{misurata} risulta pi\`u grande o pi\`u piccola
di quella alla sorgente (a seconda che ci allontaniamo o ci avviciniamo ad essa):
\begin{align}\label{eq:deltat}
  \Delta_T(v) \approx \frac{1}{f_0}\left( 1 + \frac{v}{v_s} \right)
\end{align}

A questo il gioco \`e fatto: se misuriamo la distanza media tra i battiti per diversi
valori di $v$, possiamo fittare i nostri dati con una retta ed utilizzare
la~\eqref{eq:deltat} per esprimere la velocit\`a del suono $v_s$ in funzione
del coefficiente angolare di \emph{best-fit} m:
\begin{align}
  m = \frac{1}{f_0 v_s} \quad \text{ovvero} \quad v_s = \frac{1}{m f_0}
\end{align}


\section*{Progettazione dell'esperimento}

Dedicate la prima parte della relazione ad una illustrazione dettagliata di
come avete \emph{progettato} la misura, ovverosia di come avete scelto i parametri
liberi. Per aiutarvi, provate a rispondere alle seguenti domande:
\begin{itemize}
  \item qual \`e la risoluzione temporale minima che serve per una misura
    sensata, dato il valore tabulato di $v_s$?
  \item con la risoluzione temporale praticamente ottenibile con il vostro
    sistema, quale incertezza potete attendervi sulla misura di $v_s$?
  \item quali considerazioni intervengono nella scelta della frequenza $f_0$ del
    metronomo?
  \item qual \`e la velocit\`a massima $\left| v \right|$ che potete
    realizzare in pratica?
  \item come si potrebbe migliorare l'esperimento?
\end{itemize}


\secconsiderations

Per le misure in movimento potete semplicemente camminare con lo \emph{smatrphone}
in mano cercando di muoveri con velocit\`a costante, e stimare quest'ultima
con il rapporto tra lo spazio percorso ed il tempo impiegato. In principio
potete eseguire un numero arbitrario di misure, con valori diversi di $v$; in
pratica, data la difficolt\`a nel controllare la velocit\`a, fate come minimo
tre misure: una a riposo, una avvicinandovi al metronomo ed una allontanandovi da
esso.

Dato che, assumendo $v$ costante, misurare una serie di differenze di tempo
che dovrebbero essere identiche, per ogni valore di $v$ potete stimare i tempi
di tutti i battiti del metronomo, fare le differenze di tempo tra battiti
consecutivi, ed utilizzare la media e la deviazione standard della media di
queste ultime come valori centrali ed incertezze da utilizzare nel vostro fit
come variabile dipendente.

(Suggerimento: maggiore \`e la velcocit\`a $v$ a cui vi muovete, maggiore \`e
l'effetto su $\Delta_T$---e quindi pi\`u facile da misurare. D'altra parte, con
ogni probabilit\`a, avrete uno spazio limitato in cui muovervi, per cui
pi\`u grande \`e la velcit\`a, minore \`e il numero di battiti utili del
metronomo che potete utilizzare nella vostra misura. Come giocano le due cose?)


\end{article}
\end{document}
