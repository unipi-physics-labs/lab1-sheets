\documentclass{lab1-article}

\title{Conteggi}

\usepackage{fancyvrb}
\usepackage{hyperref}
\input{python}
  
\begin{document}

\begin{article}
\selectlanguage{italian}

\maketitle

\secintro

Lo scopo dell'esperienza \`e quello di familiarizzare con i concetti di base
della statistica attraverso misure di conteggi.

\secmaterialsdad

\begin{itemize}
\item carta e penna
\item cronometro
\item calcolatore
\end{itemize}


\secmeasurements
L'evento che vogliamo studiare \`e il seguente \emph{una persona entra in un negozio}. Scegliete un negozio abbastanza frequentato (ad esempio un supermercato) e prendete le misure in due momenti diversi (due acquisizioni): uno in cui il negozio \`e poco frequentato, e uno in cui il negozio è più frequentato.

\labsubsection{Numero di eventi al minuto}

Contate il numero di persone che entrano nel negozio in un minuto per (almeno) $N=30$ minuti. Costruite il vettore delle occorrenze $l_i$ $(i=1, \ldots, N)$, dove il valore dell'elemento $l_i$ \`e pari al numero di persone entrate al minuto $i-$esimo. Realizzate un istogramma del numero di persone che entrano al minuto cos\`i misurato.

Calcolate la media campione:
\begin{align}
  m_l = \frac{1}{N} \sum_{i=1}^{N} l_i
\end{align}
e verificate l'ipotesi che la distribuzione di $l$ sia una Poissoniana con
media $m_l$ tramite un test del $\chi^2$:
\begin{align}
  \chi^2 = \sum_{j=1}^{n} \frac{(o_j - e_j)^2}{e_j}.
\end{align}
La somma sui $j$ \`e estesa su tutti gli $n$ bin dell'istogramma ed i valori
attesi $e_j$ si calcolano utilizzando la distribuzione di Poisson con media $m_l$.

Ripetete l'analisi per le due acquisizioni di dati (due distribuzioni, due medie, due test del  $\chi^2$ \ldots).

\vspace{0.5cm}
Cercate di rispondere alle seguenti domande:
\begin{enumerate}
    \item Vi \emph{aspettate} che il processo sia Poissoniano? Perch\`e?
    \item Il processo osservato è Poissoniano? Perch\`e?
    \item Come si confrontano le due acquisizioni?
\end{enumerate}

\labsubsection{Unione delle Acquisizioni}
Unite adesso le due acquisizioni: se $\vec{l}$ e $\vec{k}$ sono i vettori delle occorrenze delle due acquisizioni, l'unione va fatta sommando i vettori elemento per elemento:
$s_i = l_i + k_i$. Ripetete l'analisi fatta precedentemente su questo unico set di dati, con il vettore delle occorrenze $\vec{s}$.

\vspace{0.5cm}
Cercate di rispondere alle seguenti domande:
\begin{enumerate}
    \item Vi \emph{aspettate} che il processo "somma" delle due acquisizioni sia Poissoniano? Perch\`e?
    \item Il processo "somma" osservato è Poissoniano? Perch\`e?
\end{enumerate}

\secconsiderations

Essendo tutti i conteggi in questione variabili discrete, assicuratevi di avere
un solo valore per ciascuna classe degli istogrammi rilevanti (ovvero, scegliete
il \emph{binning} in modo che abbia passo unitario). In questo modo eviterete
effetti di \emph{risonanza}, i.e., canali vuoti per il semplice motivo che non
contengono nessun valore possibile della grandezza e/o canali che contengono pi\`u
di un valore possibile di.

Generalmente il test del $\chi^2$ per una distribuzione richiede che il contenuto
di ciascun canale dell'istogramma sia abbastanza grande da poter approssimare
la relativa distribuzione con una Gaussiana. In pratica \`e impossible dare una
ricetta che funzioni in ogni situazione, ma evitate di avere un numero
cospicuo di canali con 0, 1 o 2 conteggi, ad esempio raggruppandoli insieme.

\onecolumn
\begin{Verbatim}[label=\makebox{\href{https://github.com/unipi-physics-labs/lab1-sheets/tree/main/snippy/dad_conteggi.py}{https://github.com/.../dad\_conteggi.py}},commandchars=\\\{\}]
\PY{k+kn}{import}\PY{+w}{ }\PY{n+nn}{numpy}\PY{+w}{ }\PY{k}{as}\PY{+w}{ }\PY{n+nn}{np}
\PY{k+kn}{import}\PY{+w}{ }\PY{n+nn}{matplotlib}\PY{n+nn}{.}\PY{n+nn}{pyplot}\PY{+w}{ }\PY{k}{as}\PY{+w}{ }\PY{n+nn}{plt}
\PY{k+kn}{from}\PY{+w}{ }\PY{n+nn}{scipy}\PY{n+nn}{.}\PY{n+nn}{stats}\PY{+w}{ }\PY{k+kn}{import} \PY{n}{poisson}\PY{p}{,} \PY{n}{norm}\PY{p}{,} \PY{n}{binom}

\PY{c+c1}{\PYZsh{} Array con la lunghezza dei versi\PYZhy{}\PYZhy{}\PYZhy{}potete definirlo a mano nello script}
\PY{c+c1}{\PYZsh{} oppure leggerlo da file.}
\PY{n}{l} \PY{o}{=} \PY{o}{.}\PY{o}{.}\PY{o}{.}

\PY{c+c1}{\PYZsh{} Calcolo della statistica del campione delle lunghezze dei versi, utilizzando le}
\PY{c+c1}{\PYZsh{} funzioni appropriate di numpy.}
\PY{n}{N} \PY{o}{=} \PY{n+nb}{len}\PY{p}{(}\PY{n}{l}\PY{p}{)}
\PY{n}{m} \PY{o}{=} \PY{n}{l}\PY{o}{.}\PY{n}{mean}\PY{p}{(}\PY{p}{)}
\PY{n}{s} \PY{o}{=} \PY{n}{l}\PY{o}{.}\PY{n}{std}\PY{p}{(}\PY{n}{ddof}\PY{o}{=}\PY{l+m+mi}{1}\PY{p}{)}
\PY{n+nb}{print}\PY{p}{(}\PY{l+s+sa}{f}\PY{l+s+s1}{\PYZsq{}}\PY{l+s+s1}{Numero di versi: }\PY{l+s+si}{\PYZob{}}\PY{n}{N}\PY{l+s+si}{\PYZcb{}}\PY{l+s+s1}{\PYZsq{}}\PY{p}{)}
\PY{n+nb}{print}\PY{p}{(}\PY{l+s+sa}{f}\PY{l+s+s1}{\PYZsq{}}\PY{l+s+s1}{Lunghezza media dei varsi: }\PY{l+s+si}{\PYZob{}}\PY{n}{m}\PY{l+s+si}{\PYZcb{}}\PY{l+s+s1}{\PYZsq{}}\PY{p}{)}
\PY{n+nb}{print}\PY{p}{(}\PY{l+s+sa}{f}\PY{l+s+s1}{\PYZsq{}}\PY{l+s+s1}{Deviazione standard delle lunghezze: }\PY{l+s+si}{\PYZob{}}\PY{n}{s}\PY{l+s+si}{\PYZcb{}}\PY{l+s+s1}{\PYZsq{}}\PY{p}{)}

\PY{c+c1}{\PYZsh{} Definizione dei canali dell\PYZsq{}istogramma\PYZhy{}\PYZhy{}\PYZhy{}attenzione ad avere esattamente un valore}
\PY{c+c1}{\PYZsh{} itntero all\PYZsq{}interno di ogni bin! I \PYZhy{}0.5 e +1.5 servono per far apparire le barre al}
\PY{c+c1}{\PYZsh{} centro dei canali. Stampare per credere.}
\PY{n}{binning} \PY{o}{=} \PY{n}{np}\PY{o}{.}\PY{n}{arange}\PY{p}{(}\PY{n}{l}\PY{o}{.}\PY{n}{min}\PY{p}{(}\PY{p}{)} \PY{o}{\PYZhy{}} \PY{l+m+mf}{0.5}\PY{p}{,} \PY{n}{l}\PY{o}{.}\PY{n}{max}\PY{p}{(}\PY{p}{)} \PY{o}{+} \PY{l+m+mf}{1.5}\PY{p}{)}

\PY{c+c1}{\PYZsh{} Creazione dell\PYZsq{}istogramma. La funzione hist() di matplotlib restituisce tre variabili,}
\PY{c+c1}{\PYZsh{} ma a noi interessa solo la prima, ovvero il contenuto dei canali dell\PYZsq{}istogtramma.}
\PY{n}{plt}\PY{o}{.}\PY{n}{figure}\PY{p}{(}\PY{l+s+s1}{\PYZsq{}}\PY{l+s+s1}{Lunghezza dei versi}\PY{l+s+s1}{\PYZsq{}}\PY{p}{)}
\PY{n}{o}\PY{p}{,} \PY{n}{\PYZus{}}\PY{p}{,} \PY{n}{\PYZus{}} \PY{o}{=} \PY{n}{plt}\PY{o}{.}\PY{n}{hist}\PY{p}{(}\PY{n}{l}\PY{p}{,} \PY{n}{bins}\PY{o}{=}\PY{n}{binning}\PY{p}{,} \PY{n}{rwidth}\PY{o}{=}\PY{l+m+mf}{0.25}\PY{p}{,} \PY{n}{label}\PY{o}{=}\PY{l+s+s1}{\PYZsq{}}\PY{l+s+s1}{Conteggi}\PY{l+s+s1}{\PYZsq{}}\PY{p}{)}
\PY{n}{plt}\PY{o}{.}\PY{n}{xlabel}\PY{p}{(}\PY{l+s+s1}{\PYZsq{}}\PY{l+s+s1}{Numero di caratteri per verso}\PY{l+s+s1}{\PYZsq{}}\PY{p}{)}
\PY{n}{plt}\PY{o}{.}\PY{n}{ylabel}\PY{p}{(}\PY{l+s+s1}{\PYZsq{}}\PY{l+s+s1}{Occorrenze}\PY{l+s+s1}{\PYZsq{}}\PY{p}{)}

\PY{c+c1}{\PYZsh{} Calcolo dei valori attesi nel modello Poissoniano e gaussiano. Verificate che la}
\PY{c+c1}{\PYZsh{} definizione di k alla riga qui sotto corrisponde a tutti i valori interi compresi tra}
\PY{c+c1}{\PYZsh{} la minima e la massima lunghezza dei versi. Notate anche la differenza tra le funzioni}
\PY{c+c1}{\PYZsh{} di scipy pmf(), per il calcolo della probabilita` per una distribuzione dicreta, e}
\PY{c+c1}{\PYZsh{} pdf(), per il calcolo della densita` di probabilita` per una continua.}
\PY{n}{k} \PY{o}{=} \PY{n}{np}\PY{o}{.}\PY{n}{arange}\PY{p}{(}\PY{n}{l}\PY{o}{.}\PY{n}{min}\PY{p}{(}\PY{p}{)}\PY{p}{,} \PY{n}{l}\PY{o}{.}\PY{n}{max}\PY{p}{(}\PY{p}{)} \PY{o}{+} \PY{l+m+mi}{1}\PY{p}{)}
\PY{n}{e\PYZus{}poisson} \PY{o}{=} \PY{n}{N} \PY{o}{*} \PY{n}{poisson}\PY{o}{.}\PY{n}{pmf}\PY{p}{(}\PY{n}{k}\PY{p}{,} \PY{n}{m}\PY{p}{)}
\PY{n}{e\PYZus{}gauss} \PY{o}{=} \PY{n}{N} \PY{o}{*} \PY{n}{norm}\PY{o}{.}\PY{n}{pdf}\PY{p}{(}\PY{n}{k}\PY{p}{,} \PY{n}{m}\PY{p}{,} \PY{n}{s}\PY{p}{)}
\PY{c+c1}{\PYZsh{} Gia` che ci siamo, disegnamo i valori attesi sull\PYZsq{}istogramma di partenza.}
\PY{c+c1}{\PYZsh{} Notate che disegnamo la distribuzione di Poisson di best fit come un grafico a barre}
\PY{c+c1}{\PYZsh{} (spostato rigidamente di 0.3 unita` sulla sinistra per far si` che non si sovrapponga}
\PY{c+c1}{\PYZsh{} all\PYZsq{}istogramma di partenza) e la distribuzione di Gauss come una linea.}
\PY{n}{plt}\PY{o}{.}\PY{n}{bar}\PY{p}{(}\PY{n}{k} \PY{o}{\PYZhy{}} \PY{l+m+mf}{0.3}\PY{p}{,} \PY{n}{e\PYZus{}poisson}\PY{p}{,} \PY{n}{width}\PY{o}{=}\PY{l+m+mf}{0.25}\PY{p}{,} \PY{n}{color}\PY{o}{=}\PY{l+s+s1}{\PYZsq{}}\PY{l+s+s1}{\PYZsh{}ff7f0e}\PY{l+s+s1}{\PYZsq{}}\PY{p}{,} \PY{n}{label}\PY{o}{=}\PY{l+s+s1}{\PYZsq{}}\PY{l+s+s1}{Poisson}\PY{l+s+s1}{\PYZsq{}}\PY{p}{)}
\PY{n}{plt}\PY{o}{.}\PY{n}{plot}\PY{p}{(}\PY{n}{k}\PY{p}{,} \PY{n}{e\PYZus{}gauss}\PY{p}{,} \PY{n}{color}\PY{o}{=}\PY{l+s+s1}{\PYZsq{}}\PY{l+s+s1}{\PYZsh{}2ca02c}\PY{l+s+s1}{\PYZsq{}}\PY{p}{,} \PY{n}{label}\PY{o}{=}\PY{l+s+s1}{\PYZsq{}}\PY{l+s+s1}{Gauss}\PY{l+s+s1}{\PYZsq{}}\PY{p}{)}

\PY{c+c1}{\PYZsh{} Calcolo del chi quadro nelle due ipotesi. Assicuratevi di capire esattamente perche\PYZsq{}}
\PY{c+c1}{\PYZsh{} le due righe seguenti corrispondono all\PYZsq{}espressione che abbiamo studiato.}
\PY{n}{chi2\PYZus{}poisson} \PY{o}{=} \PY{p}{(}\PY{p}{(}\PY{n}{o} \PY{o}{\PYZhy{}} \PY{n}{e\PYZus{}poisson}\PY{p}{)}\PY{o}{*}\PY{o}{*}\PY{l+m+mf}{2.} \PY{o}{/} \PY{n}{e\PYZus{}poisson}\PY{p}{)}\PY{o}{.}\PY{n}{sum}\PY{p}{(}\PY{p}{)}
\PY{n}{chi2\PYZus{}gauss} \PY{o}{=} \PY{p}{(}\PY{p}{(}\PY{n}{o} \PY{o}{\PYZhy{}} \PY{n}{e\PYZus{}gauss}\PY{p}{)}\PY{o}{*}\PY{o}{*}\PY{l+m+mf}{2.} \PY{o}{/} \PY{n}{e\PYZus{}gauss}\PY{p}{)}\PY{o}{.}\PY{n}{sum}\PY{p}{(}\PY{p}{)}
\PY{n}{dof\PYZus{}poisson} \PY{o}{=} \PY{n+nb}{len}\PY{p}{(}\PY{n}{k}\PY{p}{)} \PY{o}{\PYZhy{}} \PY{l+m+mi}{1} \PY{o}{\PYZhy{}} \PY{l+m+mi}{1}
\PY{n}{dof\PYZus{}gauss} \PY{o}{=} \PY{n+nb}{len}\PY{p}{(}\PY{n}{k}\PY{p}{)} \PY{o}{\PYZhy{}} \PY{l+m+mi}{1} \PY{o}{\PYZhy{}} \PY{l+m+mi}{2}
\PY{n+nb}{print}\PY{p}{(}\PY{l+s+sa}{f}\PY{l+s+s1}{\PYZsq{}}\PY{l+s+s1}{chi2 per la Poissoniana: }\PY{l+s+si}{\PYZob{}}\PY{n}{chi2\PYZus{}poisson}\PY{l+s+si}{\PYZcb{}}\PY{l+s+s1}{ / }\PY{l+s+si}{\PYZob{}}\PY{n}{dof\PYZus{}poisson}\PY{l+s+si}{\PYZcb{}}\PY{l+s+s1}{ dof}\PY{l+s+s1}{\PYZsq{}}\PY{p}{)}
\PY{n+nb}{print}\PY{p}{(}\PY{l+s+sa}{f}\PY{l+s+s1}{\PYZsq{}}\PY{l+s+s1}{chi2 per la Gaussiana: }\PY{l+s+si}{\PYZob{}}\PY{n}{chi2\PYZus{}gauss}\PY{l+s+si}{\PYZcb{}}\PY{l+s+s1}{ / }\PY{l+s+si}{\PYZob{}}\PY{n}{dof\PYZus{}gauss}\PY{l+s+si}{\PYZcb{}}\PY{l+s+s1}{ dof}\PY{l+s+s1}{\PYZsq{}}\PY{p}{)}

\PY{n}{plt}\PY{o}{.}\PY{n}{legend}\PY{p}{(}\PY{p}{)}
\PY{n}{plt}\PY{o}{.}\PY{n}{show}\PY{p}{(}\PY{p}{)}
\end{Verbatim}


\end{article}
\end{document}
