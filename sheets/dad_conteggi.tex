\documentclass{lab1-article}

\title{Conteggi}

\usepackage{fancyvrb}
\usepackage{hyperref}
\makeatletter
\def\PY@reset{\let\PY@it=\relax \let\PY@bf=\relax%
    \let\PY@ul=\relax \let\PY@tc=\relax%
    \let\PY@bc=\relax \let\PY@ff=\relax}
\def\PY@tok#1{\csname PY@tok@#1\endcsname}
\def\PY@toks#1+{\ifx\relax#1\empty\else%
    \PY@tok{#1}\expandafter\PY@toks\fi}
\def\PY@do#1{\PY@bc{\PY@tc{\PY@ul{%
    \PY@it{\PY@bf{\PY@ff{#1}}}}}}}
\def\PY#1#2{\PY@reset\PY@toks#1+\relax+\PY@do{#2}}

\expandafter\def\csname PY@tok@gd\endcsname{\def\PY@tc##1{\textcolor[rgb]{0.63,0.00,0.00}{##1}}}
\expandafter\def\csname PY@tok@gu\endcsname{\let\PY@bf=\textbf\def\PY@tc##1{\textcolor[rgb]{0.50,0.00,0.50}{##1}}}
\expandafter\def\csname PY@tok@gt\endcsname{\def\PY@tc##1{\textcolor[rgb]{0.00,0.27,0.87}{##1}}}
\expandafter\def\csname PY@tok@gs\endcsname{\let\PY@bf=\textbf}
\expandafter\def\csname PY@tok@gr\endcsname{\def\PY@tc##1{\textcolor[rgb]{1.00,0.00,0.00}{##1}}}
\expandafter\def\csname PY@tok@cm\endcsname{\let\PY@it=\textit\def\PY@tc##1{\textcolor[rgb]{0.25,0.50,0.50}{##1}}}
\expandafter\def\csname PY@tok@vg\endcsname{\def\PY@tc##1{\textcolor[rgb]{0.10,0.09,0.49}{##1}}}
\expandafter\def\csname PY@tok@m\endcsname{\def\PY@tc##1{\textcolor[rgb]{0.40,0.40,0.40}{##1}}}
\expandafter\def\csname PY@tok@mh\endcsname{\def\PY@tc##1{\textcolor[rgb]{0.40,0.40,0.40}{##1}}}
\expandafter\def\csname PY@tok@go\endcsname{\def\PY@tc##1{\textcolor[rgb]{0.53,0.53,0.53}{##1}}}
\expandafter\def\csname PY@tok@ge\endcsname{\let\PY@it=\textit}
\expandafter\def\csname PY@tok@vc\endcsname{\def\PY@tc##1{\textcolor[rgb]{0.10,0.09,0.49}{##1}}}
\expandafter\def\csname PY@tok@il\endcsname{\def\PY@tc##1{\textcolor[rgb]{0.40,0.40,0.40}{##1}}}
\expandafter\def\csname PY@tok@cs\endcsname{\let\PY@it=\textit\def\PY@tc##1{\textcolor[rgb]{0.25,0.50,0.50}{##1}}}
\expandafter\def\csname PY@tok@cp\endcsname{\def\PY@tc##1{\textcolor[rgb]{0.74,0.48,0.00}{##1}}}
\expandafter\def\csname PY@tok@gi\endcsname{\def\PY@tc##1{\textcolor[rgb]{0.00,0.63,0.00}{##1}}}
\expandafter\def\csname PY@tok@gh\endcsname{\let\PY@bf=\textbf\def\PY@tc##1{\textcolor[rgb]{0.00,0.00,0.50}{##1}}}
\expandafter\def\csname PY@tok@ni\endcsname{\let\PY@bf=\textbf\def\PY@tc##1{\textcolor[rgb]{0.60,0.60,0.60}{##1}}}
\expandafter\def\csname PY@tok@nl\endcsname{\def\PY@tc##1{\textcolor[rgb]{0.63,0.63,0.00}{##1}}}
\expandafter\def\csname PY@tok@nn\endcsname{\let\PY@bf=\textbf\def\PY@tc##1{\textcolor[rgb]{0.00,0.00,1.00}{##1}}}
\expandafter\def\csname PY@tok@no\endcsname{\def\PY@tc##1{\textcolor[rgb]{0.53,0.00,0.00}{##1}}}
\expandafter\def\csname PY@tok@na\endcsname{\def\PY@tc##1{\textcolor[rgb]{0.49,0.56,0.16}{##1}}}
\expandafter\def\csname PY@tok@nb\endcsname{\def\PY@tc##1{\textcolor[rgb]{0.00,0.50,0.00}{##1}}}
\expandafter\def\csname PY@tok@nc\endcsname{\let\PY@bf=\textbf\def\PY@tc##1{\textcolor[rgb]{0.00,0.00,1.00}{##1}}}
\expandafter\def\csname PY@tok@nd\endcsname{\def\PY@tc##1{\textcolor[rgb]{0.67,0.13,1.00}{##1}}}
\expandafter\def\csname PY@tok@ne\endcsname{\let\PY@bf=\textbf\def\PY@tc##1{\textcolor[rgb]{0.82,0.25,0.23}{##1}}}
\expandafter\def\csname PY@tok@nf\endcsname{\def\PY@tc##1{\textcolor[rgb]{0.00,0.00,1.00}{##1}}}
\expandafter\def\csname PY@tok@si\endcsname{\let\PY@bf=\textbf\def\PY@tc##1{\textcolor[rgb]{0.73,0.40,0.53}{##1}}}
\expandafter\def\csname PY@tok@s2\endcsname{\def\PY@tc##1{\textcolor[rgb]{0.73,0.13,0.13}{##1}}}
\expandafter\def\csname PY@tok@vi\endcsname{\def\PY@tc##1{\textcolor[rgb]{0.10,0.09,0.49}{##1}}}
\expandafter\def\csname PY@tok@nt\endcsname{\let\PY@bf=\textbf\def\PY@tc##1{\textcolor[rgb]{0.00,0.50,0.00}{##1}}}
\expandafter\def\csname PY@tok@nv\endcsname{\def\PY@tc##1{\textcolor[rgb]{0.10,0.09,0.49}{##1}}}
\expandafter\def\csname PY@tok@s1\endcsname{\def\PY@tc##1{\textcolor[rgb]{0.73,0.13,0.13}{##1}}}
\expandafter\def\csname PY@tok@sh\endcsname{\def\PY@tc##1{\textcolor[rgb]{0.73,0.13,0.13}{##1}}}
\expandafter\def\csname PY@tok@sc\endcsname{\def\PY@tc##1{\textcolor[rgb]{0.73,0.13,0.13}{##1}}}
\expandafter\def\csname PY@tok@sx\endcsname{\def\PY@tc##1{\textcolor[rgb]{0.00,0.50,0.00}{##1}}}
\expandafter\def\csname PY@tok@bp\endcsname{\def\PY@tc##1{\textcolor[rgb]{0.00,0.50,0.00}{##1}}}
\expandafter\def\csname PY@tok@c1\endcsname{\let\PY@it=\textit\def\PY@tc##1{\textcolor[rgb]{0.25,0.50,0.50}{##1}}}
\expandafter\def\csname PY@tok@kc\endcsname{\let\PY@bf=\textbf\def\PY@tc##1{\textcolor[rgb]{0.00,0.50,0.00}{##1}}}
\expandafter\def\csname PY@tok@c\endcsname{\let\PY@it=\textit\def\PY@tc##1{\textcolor[rgb]{0.25,0.50,0.50}{##1}}}
\expandafter\def\csname PY@tok@mf\endcsname{\def\PY@tc##1{\textcolor[rgb]{0.40,0.40,0.40}{##1}}}
\expandafter\def\csname PY@tok@err\endcsname{\def\PY@bc##1{\setlength{\fboxsep}{0pt}\fcolorbox[rgb]{1.00,0.00,0.00}{1,1,1}{\strut ##1}}}
\expandafter\def\csname PY@tok@kd\endcsname{\let\PY@bf=\textbf\def\PY@tc##1{\textcolor[rgb]{0.00,0.50,0.00}{##1}}}
\expandafter\def\csname PY@tok@ss\endcsname{\def\PY@tc##1{\textcolor[rgb]{0.10,0.09,0.49}{##1}}}
\expandafter\def\csname PY@tok@sr\endcsname{\def\PY@tc##1{\textcolor[rgb]{0.73,0.40,0.53}{##1}}}
\expandafter\def\csname PY@tok@mo\endcsname{\def\PY@tc##1{\textcolor[rgb]{0.40,0.40,0.40}{##1}}}
\expandafter\def\csname PY@tok@kn\endcsname{\let\PY@bf=\textbf\def\PY@tc##1{\textcolor[rgb]{0.00,0.50,0.00}{##1}}}
\expandafter\def\csname PY@tok@mi\endcsname{\def\PY@tc##1{\textcolor[rgb]{0.40,0.40,0.40}{##1}}}
\expandafter\def\csname PY@tok@gp\endcsname{\let\PY@bf=\textbf\def\PY@tc##1{\textcolor[rgb]{0.00,0.00,0.50}{##1}}}
\expandafter\def\csname PY@tok@o\endcsname{\def\PY@tc##1{\textcolor[rgb]{0.40,0.40,0.40}{##1}}}
\expandafter\def\csname PY@tok@kr\endcsname{\let\PY@bf=\textbf\def\PY@tc##1{\textcolor[rgb]{0.00,0.50,0.00}{##1}}}
\expandafter\def\csname PY@tok@s\endcsname{\def\PY@tc##1{\textcolor[rgb]{0.73,0.13,0.13}{##1}}}
\expandafter\def\csname PY@tok@kp\endcsname{\def\PY@tc##1{\textcolor[rgb]{0.00,0.50,0.00}{##1}}}
\expandafter\def\csname PY@tok@w\endcsname{\def\PY@tc##1{\textcolor[rgb]{0.73,0.73,0.73}{##1}}}
\expandafter\def\csname PY@tok@kt\endcsname{\def\PY@tc##1{\textcolor[rgb]{0.69,0.00,0.25}{##1}}}
\expandafter\def\csname PY@tok@ow\endcsname{\let\PY@bf=\textbf\def\PY@tc##1{\textcolor[rgb]{0.67,0.13,1.00}{##1}}}
\expandafter\def\csname PY@tok@sb\endcsname{\def\PY@tc##1{\textcolor[rgb]{0.73,0.13,0.13}{##1}}}
\expandafter\def\csname PY@tok@k\endcsname{\let\PY@bf=\textbf\def\PY@tc##1{\textcolor[rgb]{0.00,0.50,0.00}{##1}}}
\expandafter\def\csname PY@tok@se\endcsname{\let\PY@bf=\textbf\def\PY@tc##1{\textcolor[rgb]{0.73,0.40,0.13}{##1}}}
\expandafter\def\csname PY@tok@sd\endcsname{\let\PY@it=\textit\def\PY@tc##1{\textcolor[rgb]{0.73,0.13,0.13}{##1}}}

\def\PYZbs{\char`\\}
\def\PYZus{\char`\_}
\def\PYZob{\char`\{}
\def\PYZcb{\char`\}}
\def\PYZca{\char`\^}
\def\PYZam{\char`\&}
\def\PYZlt{\char`\<}
\def\PYZgt{\char`\>}
\def\PYZsh{\char`\#}
\def\PYZpc{\char`\%}
\def\PYZdl{\char`\$}
\def\PYZhy{\char`\-}
\def\PYZsq{\char`\'}
\def\PYZdq{\char`\"}
\def\PYZti{\char`\~}
% for compatibility with earlier versions
\def\PYZat{@}
\def\PYZlb{[}
\def\PYZrb{]}
\makeatother


\begin{document}


\begin{article}
\selectlanguage{italian}

\maketitle

\secintro

Lo scopo dell'esperienza \`e quello di familiarizzare con i concetti di base
della statistica dei conteggi attraverso l'analisi di una porzione di un testo
letterario---in particolare il primo canto dell'\emph{Inferno} della
\emph{Divina Commedia}.

\secmaterialsdad

\begin{itemize}
\item Una copia della \emph{Divina Commedia}, in formato cartaceo o elettronico;
\item calcolatore.
\end{itemize}


\secmeasurements

Il primo canto dell'\emph{Inferno} \`e composto da 45 terzine ed un verso finale,
per un totale di $N_v = 136$ versi e $N_c = 4849$ caratteri.

(Qui e nel seguito per \emph{caratteri} intendimo le lettere, gli spazi, gli
apostrofi, le virgolette del discorso diretto e la punteggiatura. Per comodit\`a
non distinguiamo le lettere maiuscole dalle minuscole, i.e., assumiamo che
siano tutte minuscole.)


\labsubsection{Lunghezza dei versi}

Contate il numero $l_i$ di caratteri $(i=1, \ldots, 136)$ in ogni verso e
realizzate un istogramma della lunghezza dei versi cos\`i misurata.

Calcolate la media campione della lunghezza dei versi
\begin{align}
  m_l = \frac{1}{N_v} \sum_{i=1}^{N_v} l_i
\end{align}
e verificate l'ipotesi che la distribuzione di $l$ sia una Poissoniana con
media $m_l$ tramite un test del $\chi^2$:
\begin{align}
  \chi^2 = \sum_{i=1}^{n} \frac{(o_i - e_i)^2}{e_i}.
\end{align}
(La somma sui $i$ \`e estesa su tutti i bin dell'istogramma ed i valori
attesi si calcolano utilizzando la distribuzione di Poisson.)


Calcolate la varianza campione della lunghezza dei versi
\begin{align}
  s^2_l = \frac{1}{(N_v - 1)} \sum_{i=1}^{N_v} (l_i - m_l)^2
\end{align}
e verificate tramite un test del $\chi^2$ l'ipotesi che la distribuzione
della lunghezza dei versi sia una Gaussiana con media $m_l$ e
deviazione standard $s_l$.

\emph{Cosa impariamo dal confronto tra i due valori del $\chi^2$ sul modo in
  cui il testo \`e stato scritto?}


\labsubsection{Occorrenze di una lettera}

Contate il numero di occorrenze di una lettera, ad esempio la lettera "a",
all'interno di ogni verso, e costruite, come nel caso della lunghezza dei
versi, l'istogramma corrispondente.

Provate a verificare, tramite un test del $\chi^2$, l'ipotesi che la
distribuzione sia binominale, utilizzando:
\begin{itemize}
  \item come probabilit\`a elementare $p$ il rapporto tra il numero totale di
  occorrenze della lettera "a" ed il numero totale di caratteri $N_c$;
  \item come numero \emph{efficace} di tentativi $N$ la lunghezza media dei
  versi $m_l$, calcolata prima, arrotondata all'intero pi\`u vicino.
\end{itemize}

Verificare infine, sempre mediante un test del $\chi^2$, l'ipotesi che
la distribuzione delle occorrenze della lettera "a" sia una Poissoniana con
media $\mu = p m_l$. Quale dei due modelli risulta migliore?


\secconsiderations

Potete contare i caratteri manualmente, utilizzando una copia cartacea
della \emph{Divina Commedia}, oppure attraverso un piccolo programma in
Python, utilizzando una copia elettronica dell'opera (se ne trovano
svariate su web). In questo secondo caso, se siete particolarmente motivati
potete estendere l'analisi a pi\`u di un canto, a tutto l'\emph{Inferno},
o addirittura all'opera intera.

Essendo tutti i conteggi in questione variabili discrete, assicuratevi di avere
un solo valore per ciascuna classe degli istogrammi rilevanti (ovvero, scegliete
il \emph{binning} in modo che abbia passo unitario). In questo modo eviterete
effetti di \emph{risonanza}, i.e., canali vuoti per il semplice motivo che non
contengono nessun valore possibile della grandezza e/o canali che contengono pi\`u
di un valore possibile di.

Generalmente il test del $\chi^2$ per una distribuzione richiede che il contenuto
di ciascun canale dell'istogramma sia abbastanza grande da poter approssimare
la relativa distribuzione con una Gaussiana. In pratica \`e impossible dare una
ricetta che funzioni in ogni situazione, ma evitate di avere un numero
cospicuo di canali con 0, 1 o 2 conteggi, ad esempio raggruppandoli insieme.

Per il test del $\chi^2$ con la distribuzione di Gauss (che \`e intrinsecamente
di variabile discreta) le occorrenze attese $e_i$ si calcolano integrando la
funzione di distribuzione entro ciascun canale dell'istogramma (i.e.,
utilizzando la \emph{error function}). Di quanto sbagliate se approssimate
questo integrale utilizzando il valore della densit\`a di probabilit\`a al
centro del \emph{bin}?


\onecolumn

\begin{Verbatim}[label=\makebox{\href{https://github.com/unipi-physics-labs/lab1-sheets/tree/main/snippy/dad_conteggi.py}{https://github.com/.../dad\_conteggi.py}},commandchars=\\\{\}]
\PY{k+kn}{import}\PY{+w}{ }\PY{n+nn}{numpy}\PY{+w}{ }\PY{k}{as}\PY{+w}{ }\PY{n+nn}{np}
\PY{k+kn}{import}\PY{+w}{ }\PY{n+nn}{matplotlib}\PY{n+nn}{.}\PY{n+nn}{pyplot}\PY{+w}{ }\PY{k}{as}\PY{+w}{ }\PY{n+nn}{plt}
\PY{k+kn}{from}\PY{+w}{ }\PY{n+nn}{scipy}\PY{n+nn}{.}\PY{n+nn}{stats}\PY{+w}{ }\PY{k+kn}{import} \PY{n}{poisson}\PY{p}{,} \PY{n}{norm}\PY{p}{,} \PY{n}{binom}

\PY{c+c1}{\PYZsh{} Array con la lunghezza dei versi\PYZhy{}\PYZhy{}\PYZhy{}potete definirlo a mano nello script}
\PY{c+c1}{\PYZsh{} oppure leggerlo da file.}
\PY{n}{l} \PY{o}{=} \PY{o}{.}\PY{o}{.}\PY{o}{.}

\PY{c+c1}{\PYZsh{} Calcolo della statistica del campione delle lunghezze dei versi, utilizzando le}
\PY{c+c1}{\PYZsh{} funzioni appropriate di numpy.}
\PY{n}{N} \PY{o}{=} \PY{n+nb}{len}\PY{p}{(}\PY{n}{l}\PY{p}{)}
\PY{n}{m} \PY{o}{=} \PY{n}{l}\PY{o}{.}\PY{n}{mean}\PY{p}{(}\PY{p}{)}
\PY{n}{s} \PY{o}{=} \PY{n}{l}\PY{o}{.}\PY{n}{std}\PY{p}{(}\PY{n}{ddof}\PY{o}{=}\PY{l+m+mi}{1}\PY{p}{)}
\PY{n+nb}{print}\PY{p}{(}\PY{l+s+sa}{f}\PY{l+s+s1}{\PYZsq{}}\PY{l+s+s1}{Numero di versi: }\PY{l+s+si}{\PYZob{}}\PY{n}{N}\PY{l+s+si}{\PYZcb{}}\PY{l+s+s1}{\PYZsq{}}\PY{p}{)}
\PY{n+nb}{print}\PY{p}{(}\PY{l+s+sa}{f}\PY{l+s+s1}{\PYZsq{}}\PY{l+s+s1}{Lunghezza media dei varsi: }\PY{l+s+si}{\PYZob{}}\PY{n}{m}\PY{l+s+si}{\PYZcb{}}\PY{l+s+s1}{\PYZsq{}}\PY{p}{)}
\PY{n+nb}{print}\PY{p}{(}\PY{l+s+sa}{f}\PY{l+s+s1}{\PYZsq{}}\PY{l+s+s1}{Deviazione standard delle lunghezze: }\PY{l+s+si}{\PYZob{}}\PY{n}{s}\PY{l+s+si}{\PYZcb{}}\PY{l+s+s1}{\PYZsq{}}\PY{p}{)}

\PY{c+c1}{\PYZsh{} Definizione dei canali dell\PYZsq{}istogramma\PYZhy{}\PYZhy{}\PYZhy{}attenzione ad avere esattamente un valore}
\PY{c+c1}{\PYZsh{} itntero all\PYZsq{}interno di ogni bin! I \PYZhy{}0.5 e +1.5 servono per far apparire le barre al}
\PY{c+c1}{\PYZsh{} centro dei canali. Stampare per credere.}
\PY{n}{binning} \PY{o}{=} \PY{n}{np}\PY{o}{.}\PY{n}{arange}\PY{p}{(}\PY{n}{l}\PY{o}{.}\PY{n}{min}\PY{p}{(}\PY{p}{)} \PY{o}{\PYZhy{}} \PY{l+m+mf}{0.5}\PY{p}{,} \PY{n}{l}\PY{o}{.}\PY{n}{max}\PY{p}{(}\PY{p}{)} \PY{o}{+} \PY{l+m+mf}{1.5}\PY{p}{)}

\PY{c+c1}{\PYZsh{} Creazione dell\PYZsq{}istogramma. La funzione hist() di matplotlib restituisce tre variabili,}
\PY{c+c1}{\PYZsh{} ma a noi interessa solo la prima, ovvero il contenuto dei canali dell\PYZsq{}istogtramma.}
\PY{n}{plt}\PY{o}{.}\PY{n}{figure}\PY{p}{(}\PY{l+s+s1}{\PYZsq{}}\PY{l+s+s1}{Lunghezza dei versi}\PY{l+s+s1}{\PYZsq{}}\PY{p}{)}
\PY{n}{o}\PY{p}{,} \PY{n}{\PYZus{}}\PY{p}{,} \PY{n}{\PYZus{}} \PY{o}{=} \PY{n}{plt}\PY{o}{.}\PY{n}{hist}\PY{p}{(}\PY{n}{l}\PY{p}{,} \PY{n}{bins}\PY{o}{=}\PY{n}{binning}\PY{p}{,} \PY{n}{rwidth}\PY{o}{=}\PY{l+m+mf}{0.25}\PY{p}{,} \PY{n}{label}\PY{o}{=}\PY{l+s+s1}{\PYZsq{}}\PY{l+s+s1}{Conteggi}\PY{l+s+s1}{\PYZsq{}}\PY{p}{)}
\PY{n}{plt}\PY{o}{.}\PY{n}{xlabel}\PY{p}{(}\PY{l+s+s1}{\PYZsq{}}\PY{l+s+s1}{Numero di caratteri per verso}\PY{l+s+s1}{\PYZsq{}}\PY{p}{)}
\PY{n}{plt}\PY{o}{.}\PY{n}{ylabel}\PY{p}{(}\PY{l+s+s1}{\PYZsq{}}\PY{l+s+s1}{Occorrenze}\PY{l+s+s1}{\PYZsq{}}\PY{p}{)}

\PY{c+c1}{\PYZsh{} Calcolo dei valori attesi nel modello Poissoniano e gaussiano. Verificate che la}
\PY{c+c1}{\PYZsh{} definizione di k alla riga qui sotto corrisponde a tutti i valori interi compresi tra}
\PY{c+c1}{\PYZsh{} la minima e la massima lunghezza dei versi. Notate anche la differenza tra le funzioni}
\PY{c+c1}{\PYZsh{} di scipy pmf(), per il calcolo della probabilita` per una distribuzione dicreta, e}
\PY{c+c1}{\PYZsh{} pdf(), per il calcolo della densita` di probabilita` per una continua.}
\PY{n}{k} \PY{o}{=} \PY{n}{np}\PY{o}{.}\PY{n}{arange}\PY{p}{(}\PY{n}{l}\PY{o}{.}\PY{n}{min}\PY{p}{(}\PY{p}{)}\PY{p}{,} \PY{n}{l}\PY{o}{.}\PY{n}{max}\PY{p}{(}\PY{p}{)} \PY{o}{+} \PY{l+m+mi}{1}\PY{p}{)}
\PY{n}{e\PYZus{}poisson} \PY{o}{=} \PY{n}{N} \PY{o}{*} \PY{n}{poisson}\PY{o}{.}\PY{n}{pmf}\PY{p}{(}\PY{n}{k}\PY{p}{,} \PY{n}{m}\PY{p}{)}
\PY{n}{e\PYZus{}gauss} \PY{o}{=} \PY{n}{N} \PY{o}{*} \PY{n}{norm}\PY{o}{.}\PY{n}{pdf}\PY{p}{(}\PY{n}{k}\PY{p}{,} \PY{n}{m}\PY{p}{,} \PY{n}{s}\PY{p}{)}
\PY{c+c1}{\PYZsh{} Gia` che ci siamo, disegnamo i valori attesi sull\PYZsq{}istogramma di partenza.}
\PY{c+c1}{\PYZsh{} Notate che disegnamo la distribuzione di Poisson di best fit come un grafico a barre}
\PY{c+c1}{\PYZsh{} (spostato rigidamente di 0.3 unita` sulla sinistra per far si` che non si sovrapponga}
\PY{c+c1}{\PYZsh{} all\PYZsq{}istogramma di partenza) e la distribuzione di Gauss come una linea.}
\PY{n}{plt}\PY{o}{.}\PY{n}{bar}\PY{p}{(}\PY{n}{k} \PY{o}{\PYZhy{}} \PY{l+m+mf}{0.3}\PY{p}{,} \PY{n}{e\PYZus{}poisson}\PY{p}{,} \PY{n}{width}\PY{o}{=}\PY{l+m+mf}{0.25}\PY{p}{,} \PY{n}{color}\PY{o}{=}\PY{l+s+s1}{\PYZsq{}}\PY{l+s+s1}{\PYZsh{}ff7f0e}\PY{l+s+s1}{\PYZsq{}}\PY{p}{,} \PY{n}{label}\PY{o}{=}\PY{l+s+s1}{\PYZsq{}}\PY{l+s+s1}{Poisson}\PY{l+s+s1}{\PYZsq{}}\PY{p}{)}
\PY{n}{plt}\PY{o}{.}\PY{n}{plot}\PY{p}{(}\PY{n}{k}\PY{p}{,} \PY{n}{e\PYZus{}gauss}\PY{p}{,} \PY{n}{color}\PY{o}{=}\PY{l+s+s1}{\PYZsq{}}\PY{l+s+s1}{\PYZsh{}2ca02c}\PY{l+s+s1}{\PYZsq{}}\PY{p}{,} \PY{n}{label}\PY{o}{=}\PY{l+s+s1}{\PYZsq{}}\PY{l+s+s1}{Gauss}\PY{l+s+s1}{\PYZsq{}}\PY{p}{)}

\PY{c+c1}{\PYZsh{} Calcolo del chi quadro nelle due ipotesi. Assicuratevi di capire esattamente perche\PYZsq{}}
\PY{c+c1}{\PYZsh{} le due righe seguenti corrispondono all\PYZsq{}espressione che abbiamo studiato.}
\PY{n}{chi2\PYZus{}poisson} \PY{o}{=} \PY{p}{(}\PY{p}{(}\PY{n}{o} \PY{o}{\PYZhy{}} \PY{n}{e\PYZus{}poisson}\PY{p}{)}\PY{o}{*}\PY{o}{*}\PY{l+m+mf}{2.} \PY{o}{/} \PY{n}{e\PYZus{}poisson}\PY{p}{)}\PY{o}{.}\PY{n}{sum}\PY{p}{(}\PY{p}{)}
\PY{n}{chi2\PYZus{}gauss} \PY{o}{=} \PY{p}{(}\PY{p}{(}\PY{n}{o} \PY{o}{\PYZhy{}} \PY{n}{e\PYZus{}gauss}\PY{p}{)}\PY{o}{*}\PY{o}{*}\PY{l+m+mf}{2.} \PY{o}{/} \PY{n}{e\PYZus{}gauss}\PY{p}{)}\PY{o}{.}\PY{n}{sum}\PY{p}{(}\PY{p}{)}
\PY{n}{dof\PYZus{}poisson} \PY{o}{=} \PY{n+nb}{len}\PY{p}{(}\PY{n}{k}\PY{p}{)} \PY{o}{\PYZhy{}} \PY{l+m+mi}{1} \PY{o}{\PYZhy{}} \PY{l+m+mi}{1}
\PY{n}{dof\PYZus{}gauss} \PY{o}{=} \PY{n+nb}{len}\PY{p}{(}\PY{n}{k}\PY{p}{)} \PY{o}{\PYZhy{}} \PY{l+m+mi}{1} \PY{o}{\PYZhy{}} \PY{l+m+mi}{2}
\PY{n+nb}{print}\PY{p}{(}\PY{l+s+sa}{f}\PY{l+s+s1}{\PYZsq{}}\PY{l+s+s1}{chi2 per la Poissoniana: }\PY{l+s+si}{\PYZob{}}\PY{n}{chi2\PYZus{}poisson}\PY{l+s+si}{\PYZcb{}}\PY{l+s+s1}{ / }\PY{l+s+si}{\PYZob{}}\PY{n}{dof\PYZus{}poisson}\PY{l+s+si}{\PYZcb{}}\PY{l+s+s1}{ dof}\PY{l+s+s1}{\PYZsq{}}\PY{p}{)}
\PY{n+nb}{print}\PY{p}{(}\PY{l+s+sa}{f}\PY{l+s+s1}{\PYZsq{}}\PY{l+s+s1}{chi2 per la Gaussiana: }\PY{l+s+si}{\PYZob{}}\PY{n}{chi2\PYZus{}gauss}\PY{l+s+si}{\PYZcb{}}\PY{l+s+s1}{ / }\PY{l+s+si}{\PYZob{}}\PY{n}{dof\PYZus{}gauss}\PY{l+s+si}{\PYZcb{}}\PY{l+s+s1}{ dof}\PY{l+s+s1}{\PYZsq{}}\PY{p}{)}

\PY{n}{plt}\PY{o}{.}\PY{n}{legend}\PY{p}{(}\PY{p}{)}
\PY{n}{plt}\PY{o}{.}\PY{n}{show}\PY{p}{(}\PY{p}{)}
\end{Verbatim}



\end{article}
\end{document}
