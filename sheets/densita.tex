\documentclass{lab1-article}

\title{Misure di densit\`a}

\usepackage{fancyvrb}
\makeatletter
\def\PY@reset{\let\PY@it=\relax \let\PY@bf=\relax%
    \let\PY@ul=\relax \let\PY@tc=\relax%
    \let\PY@bc=\relax \let\PY@ff=\relax}
\def\PY@tok#1{\csname PY@tok@#1\endcsname}
\def\PY@toks#1+{\ifx\relax#1\empty\else%
    \PY@tok{#1}\expandafter\PY@toks\fi}
\def\PY@do#1{\PY@bc{\PY@tc{\PY@ul{%
    \PY@it{\PY@bf{\PY@ff{#1}}}}}}}
\def\PY#1#2{\PY@reset\PY@toks#1+\relax+\PY@do{#2}}

\expandafter\def\csname PY@tok@gd\endcsname{\def\PY@tc##1{\textcolor[rgb]{0.63,0.00,0.00}{##1}}}
\expandafter\def\csname PY@tok@gu\endcsname{\let\PY@bf=\textbf\def\PY@tc##1{\textcolor[rgb]{0.50,0.00,0.50}{##1}}}
\expandafter\def\csname PY@tok@gt\endcsname{\def\PY@tc##1{\textcolor[rgb]{0.00,0.27,0.87}{##1}}}
\expandafter\def\csname PY@tok@gs\endcsname{\let\PY@bf=\textbf}
\expandafter\def\csname PY@tok@gr\endcsname{\def\PY@tc##1{\textcolor[rgb]{1.00,0.00,0.00}{##1}}}
\expandafter\def\csname PY@tok@cm\endcsname{\let\PY@it=\textit\def\PY@tc##1{\textcolor[rgb]{0.25,0.50,0.50}{##1}}}
\expandafter\def\csname PY@tok@vg\endcsname{\def\PY@tc##1{\textcolor[rgb]{0.10,0.09,0.49}{##1}}}
\expandafter\def\csname PY@tok@m\endcsname{\def\PY@tc##1{\textcolor[rgb]{0.40,0.40,0.40}{##1}}}
\expandafter\def\csname PY@tok@mh\endcsname{\def\PY@tc##1{\textcolor[rgb]{0.40,0.40,0.40}{##1}}}
\expandafter\def\csname PY@tok@go\endcsname{\def\PY@tc##1{\textcolor[rgb]{0.53,0.53,0.53}{##1}}}
\expandafter\def\csname PY@tok@ge\endcsname{\let\PY@it=\textit}
\expandafter\def\csname PY@tok@vc\endcsname{\def\PY@tc##1{\textcolor[rgb]{0.10,0.09,0.49}{##1}}}
\expandafter\def\csname PY@tok@il\endcsname{\def\PY@tc##1{\textcolor[rgb]{0.40,0.40,0.40}{##1}}}
\expandafter\def\csname PY@tok@cs\endcsname{\let\PY@it=\textit\def\PY@tc##1{\textcolor[rgb]{0.25,0.50,0.50}{##1}}}
\expandafter\def\csname PY@tok@cp\endcsname{\def\PY@tc##1{\textcolor[rgb]{0.74,0.48,0.00}{##1}}}
\expandafter\def\csname PY@tok@gi\endcsname{\def\PY@tc##1{\textcolor[rgb]{0.00,0.63,0.00}{##1}}}
\expandafter\def\csname PY@tok@gh\endcsname{\let\PY@bf=\textbf\def\PY@tc##1{\textcolor[rgb]{0.00,0.00,0.50}{##1}}}
\expandafter\def\csname PY@tok@ni\endcsname{\let\PY@bf=\textbf\def\PY@tc##1{\textcolor[rgb]{0.60,0.60,0.60}{##1}}}
\expandafter\def\csname PY@tok@nl\endcsname{\def\PY@tc##1{\textcolor[rgb]{0.63,0.63,0.00}{##1}}}
\expandafter\def\csname PY@tok@nn\endcsname{\let\PY@bf=\textbf\def\PY@tc##1{\textcolor[rgb]{0.00,0.00,1.00}{##1}}}
\expandafter\def\csname PY@tok@no\endcsname{\def\PY@tc##1{\textcolor[rgb]{0.53,0.00,0.00}{##1}}}
\expandafter\def\csname PY@tok@na\endcsname{\def\PY@tc##1{\textcolor[rgb]{0.49,0.56,0.16}{##1}}}
\expandafter\def\csname PY@tok@nb\endcsname{\def\PY@tc##1{\textcolor[rgb]{0.00,0.50,0.00}{##1}}}
\expandafter\def\csname PY@tok@nc\endcsname{\let\PY@bf=\textbf\def\PY@tc##1{\textcolor[rgb]{0.00,0.00,1.00}{##1}}}
\expandafter\def\csname PY@tok@nd\endcsname{\def\PY@tc##1{\textcolor[rgb]{0.67,0.13,1.00}{##1}}}
\expandafter\def\csname PY@tok@ne\endcsname{\let\PY@bf=\textbf\def\PY@tc##1{\textcolor[rgb]{0.82,0.25,0.23}{##1}}}
\expandafter\def\csname PY@tok@nf\endcsname{\def\PY@tc##1{\textcolor[rgb]{0.00,0.00,1.00}{##1}}}
\expandafter\def\csname PY@tok@si\endcsname{\let\PY@bf=\textbf\def\PY@tc##1{\textcolor[rgb]{0.73,0.40,0.53}{##1}}}
\expandafter\def\csname PY@tok@s2\endcsname{\def\PY@tc##1{\textcolor[rgb]{0.73,0.13,0.13}{##1}}}
\expandafter\def\csname PY@tok@vi\endcsname{\def\PY@tc##1{\textcolor[rgb]{0.10,0.09,0.49}{##1}}}
\expandafter\def\csname PY@tok@nt\endcsname{\let\PY@bf=\textbf\def\PY@tc##1{\textcolor[rgb]{0.00,0.50,0.00}{##1}}}
\expandafter\def\csname PY@tok@nv\endcsname{\def\PY@tc##1{\textcolor[rgb]{0.10,0.09,0.49}{##1}}}
\expandafter\def\csname PY@tok@s1\endcsname{\def\PY@tc##1{\textcolor[rgb]{0.73,0.13,0.13}{##1}}}
\expandafter\def\csname PY@tok@sh\endcsname{\def\PY@tc##1{\textcolor[rgb]{0.73,0.13,0.13}{##1}}}
\expandafter\def\csname PY@tok@sc\endcsname{\def\PY@tc##1{\textcolor[rgb]{0.73,0.13,0.13}{##1}}}
\expandafter\def\csname PY@tok@sx\endcsname{\def\PY@tc##1{\textcolor[rgb]{0.00,0.50,0.00}{##1}}}
\expandafter\def\csname PY@tok@bp\endcsname{\def\PY@tc##1{\textcolor[rgb]{0.00,0.50,0.00}{##1}}}
\expandafter\def\csname PY@tok@c1\endcsname{\let\PY@it=\textit\def\PY@tc##1{\textcolor[rgb]{0.25,0.50,0.50}{##1}}}
\expandafter\def\csname PY@tok@kc\endcsname{\let\PY@bf=\textbf\def\PY@tc##1{\textcolor[rgb]{0.00,0.50,0.00}{##1}}}
\expandafter\def\csname PY@tok@c\endcsname{\let\PY@it=\textit\def\PY@tc##1{\textcolor[rgb]{0.25,0.50,0.50}{##1}}}
\expandafter\def\csname PY@tok@mf\endcsname{\def\PY@tc##1{\textcolor[rgb]{0.40,0.40,0.40}{##1}}}
\expandafter\def\csname PY@tok@err\endcsname{\def\PY@bc##1{\setlength{\fboxsep}{0pt}\fcolorbox[rgb]{1.00,0.00,0.00}{1,1,1}{\strut ##1}}}
\expandafter\def\csname PY@tok@kd\endcsname{\let\PY@bf=\textbf\def\PY@tc##1{\textcolor[rgb]{0.00,0.50,0.00}{##1}}}
\expandafter\def\csname PY@tok@ss\endcsname{\def\PY@tc##1{\textcolor[rgb]{0.10,0.09,0.49}{##1}}}
\expandafter\def\csname PY@tok@sr\endcsname{\def\PY@tc##1{\textcolor[rgb]{0.73,0.40,0.53}{##1}}}
\expandafter\def\csname PY@tok@mo\endcsname{\def\PY@tc##1{\textcolor[rgb]{0.40,0.40,0.40}{##1}}}
\expandafter\def\csname PY@tok@kn\endcsname{\let\PY@bf=\textbf\def\PY@tc##1{\textcolor[rgb]{0.00,0.50,0.00}{##1}}}
\expandafter\def\csname PY@tok@mi\endcsname{\def\PY@tc##1{\textcolor[rgb]{0.40,0.40,0.40}{##1}}}
\expandafter\def\csname PY@tok@gp\endcsname{\let\PY@bf=\textbf\def\PY@tc##1{\textcolor[rgb]{0.00,0.00,0.50}{##1}}}
\expandafter\def\csname PY@tok@o\endcsname{\def\PY@tc##1{\textcolor[rgb]{0.40,0.40,0.40}{##1}}}
\expandafter\def\csname PY@tok@kr\endcsname{\let\PY@bf=\textbf\def\PY@tc##1{\textcolor[rgb]{0.00,0.50,0.00}{##1}}}
\expandafter\def\csname PY@tok@s\endcsname{\def\PY@tc##1{\textcolor[rgb]{0.73,0.13,0.13}{##1}}}
\expandafter\def\csname PY@tok@kp\endcsname{\def\PY@tc##1{\textcolor[rgb]{0.00,0.50,0.00}{##1}}}
\expandafter\def\csname PY@tok@w\endcsname{\def\PY@tc##1{\textcolor[rgb]{0.73,0.73,0.73}{##1}}}
\expandafter\def\csname PY@tok@kt\endcsname{\def\PY@tc##1{\textcolor[rgb]{0.69,0.00,0.25}{##1}}}
\expandafter\def\csname PY@tok@ow\endcsname{\let\PY@bf=\textbf\def\PY@tc##1{\textcolor[rgb]{0.67,0.13,1.00}{##1}}}
\expandafter\def\csname PY@tok@sb\endcsname{\def\PY@tc##1{\textcolor[rgb]{0.73,0.13,0.13}{##1}}}
\expandafter\def\csname PY@tok@k\endcsname{\let\PY@bf=\textbf\def\PY@tc##1{\textcolor[rgb]{0.00,0.50,0.00}{##1}}}
\expandafter\def\csname PY@tok@se\endcsname{\let\PY@bf=\textbf\def\PY@tc##1{\textcolor[rgb]{0.73,0.40,0.13}{##1}}}
\expandafter\def\csname PY@tok@sd\endcsname{\let\PY@it=\textit\def\PY@tc##1{\textcolor[rgb]{0.73,0.13,0.13}{##1}}}

\def\PYZbs{\char`\\}
\def\PYZus{\char`\_}
\def\PYZob{\char`\{}
\def\PYZcb{\char`\}}
\def\PYZca{\char`\^}
\def\PYZam{\char`\&}
\def\PYZlt{\char`\<}
\def\PYZgt{\char`\>}
\def\PYZsh{\char`\#}
\def\PYZpc{\char`\%}
\def\PYZdl{\char`\$}
\def\PYZhy{\char`\-}
\def\PYZsq{\char`\'}
\def\PYZdq{\char`\"}
\def\PYZti{\char`\~}
% for compatibility with earlier versions
\def\PYZat{@}
\def\PYZlb{[}
\def\PYZrb{]}
\makeatother


\begin{document}


\begin{article}
\selectlanguage{italian}

\maketitle

\secsummary
Sappiamo che una quantit\`a fissata di qualunque sostanza o materiale occupa un
volume che varia soltanto se variano le condizioni in cui tale sostanza o
materiale si trova (ad esempio se dovesse passare dallo stato solido allo stato
liquido, o se cambiano la temperatura o la pressione).

La massa per unit\`a di volume \`e nota come densit\`a:
\begin{align}
  \rho = \frac{m}{V}, \quad [\rho]= \mathrm{kg/m}^3.
\end{align}
Si pu\`o distinguere, ad esempio, un metallo da un altro misurandone la
densit\`a.


\secmaterials

\begin{itemize}
\item Calibro ventesimale (risoluzione 0.05~mm).
\item Calibro Palmer (risoluzione 0.01~mm).
\item Bilancia di precisione (risoluzione 1~mg).
\item Una serie di solidi in alluminio, acciaio e ottone.

\end{itemize}


%\secdefinitions


%\secconsiderations


\secmeasurements

\labsubsection{Misure preliminari}

Si misurino le dimensioni (raggi, altezze, spessori, etc.) dei vari corpi
solidi e se ne calcoli il volume e la corrispondente incertezza di
misura---attraverso le regole usuali della propagazione dell'errore statistico
per grandezze indipendenti (i.e., utilizzando la somma in quadratura).

Si misuri anche la massa dei vari corpi e si costruisca una tabella
contenente i valore dei volumi e delle masse.


\labsubsection{Stima delle densit\`a}

Su un grafico cartesiano si riportino i valori dei volumi in ascisse e quelli
delle masse in ordinate. Poich\'e
\begin{align}
  m = \rho V,
\end{align}
i gruppi di punti corrispondenti allo stesso materiale dovrebbero disporsi su
linee rette passanti per l'origine, il cui coefficiente angolare coincide
proprio con la densit\`a del materiale in questione.

Si esegua un \emph{fit} al calcolatore a ciascuno dei gruppi di punti che
si dispongono su queste rette e si stimi la densit\`a per i vari materiali.


\labsubsection{Legge di scala per le sfere}

Si considerino le sole sfere e si costruisca un grafico, in scala
bilogaritmica, della massa $m$ in funzione del raggio $r$. Poich\'e la relazione
tra massa e raggio \`e di tipo legge di potenza:
\begin{align}
  m = \frac{4}{3}\pi r^3 \rho = kr^3,
\end{align}
i punti in scala bilogaritmica dovrebbero disporsi su una retta.
Si confronti il coefficiente angolare della retta con il valore atteso
$3$ e l'intercetta (con l'asse $r = 1$) con il valore
\begin{align}
  k = \frac{4}{3}\pi\rho,
\end{align}
data la densit\`a misurata precedentemente.


\secappendix{densit\`a tabulate}

Si riportano di seguito i valori tabulati per i materiali rilevanti
per l'esperienza.

\medskip

\begin{center}
  \begin{tabular}{p{0.65\linewidth}@{}p{0.3\linewidth}}
    \hline
    Materiale & $\rho$ kg/m$^3$\\
    \hline
    \hline
    Alluminio & 2710\\
    Acciaio inossidabile  & 7480--8000\\
    Ottone (lega Cu-Zn)   & 8400--8700\\
    \hline
  \end{tabular}
\end{center}

\onecolumn

\begin{Verbatim}[label=\makebox{\href{https://github.com/unipi-physics-labs/lab1-sheets/tree/main/snippy/densita.py}{https://github.com/.../densita.py}},commandchars=\\\{\}]
\PY{k+kn}{import}\PY{+w}{ }\PY{n+nn}{numpy}\PY{+w}{ }\PY{k}{as}\PY{+w}{ }\PY{n+nn}{np}
\PY{k+kn}{from}\PY{+w}{ }\PY{n+nn}{matplotlib}\PY{+w}{ }\PY{k+kn}{import} \PY{n}{pyplot} \PY{k}{as} \PY{n}{plt}
\PY{k+kn}{from}\PY{+w}{ }\PY{n+nn}{scipy}\PY{n+nn}{.}\PY{n+nn}{optimize}\PY{+w}{ }\PY{k+kn}{import} \PY{n}{curve\PYZus{}fit}

\PY{c+c1}{\PYZsh{} Misure dirette dei diametri e delle masse per le sfere (mettete i vostri numeri).}
\PY{c+c1}{\PYZsh{} Potete anche leggere i dati da file, usando np.loadtxt(), se lo trovate comodo.}
\PY{n}{m} \PY{o}{=} \PY{n}{np}\PY{o}{.}\PY{n}{array}\PY{p}{(}\PY{p}{[}\PY{l+m+mf}{24.769}\PY{p}{,} \PY{l+m+mf}{11.887}\PY{p}{,} \PY{l+m+mf}{8.374}\PY{p}{,} \PY{l+m+mf}{3.528}\PY{p}{]}\PY{p}{)}
\PY{n}{sigma\PYZus{}m} \PY{o}{=} \PY{n}{np}\PY{o}{.}\PY{n}{full}\PY{p}{(}\PY{n}{m}\PY{o}{.}\PY{n}{shape}\PY{p}{,} \PY{l+m+mf}{0.001}\PY{p}{)}
\PY{n}{d} \PY{o}{=} \PY{n}{np}\PY{o}{.}\PY{n}{array}\PY{p}{(}\PY{p}{[}\PY{l+m+mf}{18.25}\PY{p}{,} \PY{l+m+mf}{14.28}\PY{p}{,} \PY{l+m+mf}{12.69}\PY{p}{,} \PY{l+m+mf}{9.52}\PY{p}{]}\PY{p}{)}
\PY{n}{sigma\PYZus{}d} \PY{o}{=} \PY{n}{np}\PY{o}{.}\PY{n}{full}\PY{p}{(}\PY{n}{d}\PY{o}{.}\PY{n}{shape}\PY{p}{,} \PY{l+m+mf}{0.01}\PY{p}{)}
\PY{c+c1}{\PYZsh{} Calcolo del volume (notate la propagazione dell\PYZsq{}errore su V!)}
\PY{n}{r} \PY{o}{=} \PY{n}{d} \PY{o}{/} \PY{l+m+mf}{2.0}
\PY{n}{sigma\PYZus{}r} \PY{o}{=} \PY{n}{sigma\PYZus{}d} \PY{o}{/} \PY{l+m+mf}{2.0}
\PY{n}{V} \PY{o}{=} \PY{l+m+mf}{4.0} \PY{o}{/} \PY{l+m+mf}{3.0} \PY{o}{*} \PY{n}{np}\PY{o}{.}\PY{n}{pi} \PY{o}{*} \PY{n}{r}\PY{o}{*}\PY{o}{*}\PY{l+m+mf}{3.0}
\PY{n}{sigma\PYZus{}V} \PY{o}{=} \PY{n}{V} \PY{o}{*} \PY{l+m+mf}{3.0} \PY{o}{*} \PY{n}{sigma\PYZus{}d} \PY{o}{/} \PY{n}{d}

\PY{k}{def}\PY{+w}{ }\PY{n+nf}{line}\PY{p}{(}\PY{n}{x}\PY{p}{,} \PY{n}{m}\PY{p}{,} \PY{n}{q}\PY{p}{)}\PY{p}{:}
\PY{+w}{    }\PY{l+s+sd}{\PYZdq{}\PYZdq{}\PYZdq{}Modello lineare di fit.}
\PY{l+s+sd}{    \PYZdq{}\PYZdq{}\PYZdq{}}
    \PY{k}{return} \PY{n}{m} \PY{o}{*} \PY{n}{x} \PY{o}{+} \PY{n}{q}

\PY{k}{def}\PY{+w}{ }\PY{n+nf}{power\PYZus{}law}\PY{p}{(}\PY{n}{x}\PY{p}{,} \PY{n}{norm}\PY{p}{,} \PY{n}{index}\PY{p}{)}\PY{p}{:}
\PY{+w}{    }\PY{l+s+sd}{\PYZdq{}\PYZdq{}\PYZdq{}Modello di tipo legge di potenza.}
\PY{l+s+sd}{    \PYZdq{}\PYZdq{}\PYZdq{}}
    \PY{k}{return} \PY{n}{norm} \PY{o}{*} \PY{p}{(}\PY{n}{x}\PY{o}{*}\PY{o}{*}\PY{n}{index}\PY{p}{)}

\PY{n}{plt}\PY{o}{.}\PY{n}{figure}\PY{p}{(}\PY{l+s+s1}{\PYZsq{}}\PY{l+s+s1}{Grafico massa\PYZhy{}volume}\PY{l+s+s1}{\PYZsq{}}\PY{p}{)}
\PY{n}{plt}\PY{o}{.}\PY{n}{errorbar}\PY{p}{(}\PY{n}{V}\PY{p}{,} \PY{n}{m}\PY{p}{,} \PY{n}{sigma\PYZus{}m}\PY{p}{,} \PY{n}{sigma\PYZus{}V}\PY{p}{,} \PY{n}{fmt}\PY{o}{=}\PY{l+s+s1}{\PYZsq{}}\PY{l+s+s1}{o}\PY{l+s+s1}{\PYZsq{}}\PY{p}{)}
\PY{n}{popt}\PY{p}{,} \PY{n}{pcov} \PY{o}{=} \PY{n}{curve\PYZus{}fit}\PY{p}{(}\PY{n}{line}\PY{p}{,} \PY{n}{V}\PY{p}{,} \PY{n}{m}\PY{p}{)}
\PY{n}{m\PYZus{}hat}\PY{p}{,} \PY{n}{q\PYZus{}hat} \PY{o}{=} \PY{n}{popt}
\PY{n}{sigma\PYZus{}m}\PY{p}{,} \PY{n}{sigma\PYZus{}q} \PY{o}{=} \PY{n}{np}\PY{o}{.}\PY{n}{sqrt}\PY{p}{(}\PY{n}{pcov}\PY{o}{.}\PY{n}{diagonal}\PY{p}{(}\PY{p}{)}\PY{p}{)}
\PY{n+nb}{print}\PY{p}{(}\PY{n}{m\PYZus{}hat}\PY{p}{,} \PY{n}{sigma\PYZus{}m}\PY{p}{,} \PY{n}{q\PYZus{}hat}\PY{p}{,} \PY{n}{sigma\PYZus{}q}\PY{p}{)}
\PY{c+c1}{\PYZsh{} Grafico del modello di best fit.}
\PY{n}{x} \PY{o}{=} \PY{n}{np}\PY{o}{.}\PY{n}{linspace}\PY{p}{(}\PY{l+m+mf}{0.}\PY{p}{,} \PY{l+m+mf}{4000.}\PY{p}{,} \PY{l+m+mi}{100}\PY{p}{)}
\PY{n}{plt}\PY{o}{.}\PY{n}{plot}\PY{p}{(}\PY{n}{x}\PY{p}{,} \PY{n}{line}\PY{p}{(}\PY{n}{x}\PY{p}{,} \PY{n}{m\PYZus{}hat}\PY{p}{,} \PY{n}{q\PYZus{}hat}\PY{p}{)}\PY{p}{)}
\PY{n}{plt}\PY{o}{.}\PY{n}{xlabel}\PY{p}{(}\PY{l+s+s1}{\PYZsq{}}\PY{l+s+s1}{Volume [mm\PYZdl{}\PYZca{}3\PYZdl{}]}\PY{l+s+s1}{\PYZsq{}}\PY{p}{)}
\PY{n}{plt}\PY{o}{.}\PY{n}{ylabel}\PY{p}{(}\PY{l+s+s1}{\PYZsq{}}\PY{l+s+s1}{Massa [g]}\PY{l+s+s1}{\PYZsq{}}\PY{p}{)}
\PY{n}{plt}\PY{o}{.}\PY{n}{grid}\PY{p}{(}\PY{n}{which}\PY{o}{=}\PY{l+s+s1}{\PYZsq{}}\PY{l+s+s1}{both}\PY{l+s+s1}{\PYZsq{}}\PY{p}{,} \PY{n}{ls}\PY{o}{=}\PY{l+s+s1}{\PYZsq{}}\PY{l+s+s1}{dashed}\PY{l+s+s1}{\PYZsq{}}\PY{p}{,} \PY{n}{color}\PY{o}{=}\PY{l+s+s1}{\PYZsq{}}\PY{l+s+s1}{gray}\PY{l+s+s1}{\PYZsq{}}\PY{p}{)}
\PY{n}{plt}\PY{o}{.}\PY{n}{savefig}\PY{p}{(}\PY{l+s+s1}{\PYZsq{}}\PY{l+s+s1}{massa\PYZus{}volume.pdf}\PY{l+s+s1}{\PYZsq{}}\PY{p}{)}

\PY{n}{plt}\PY{o}{.}\PY{n}{figure}\PY{p}{(}\PY{l+s+s1}{\PYZsq{}}\PY{l+s+s1}{Grafico massa\PYZhy{}raggio}\PY{l+s+s1}{\PYZsq{}}\PY{p}{)}
\PY{n}{plt}\PY{o}{.}\PY{n}{errorbar}\PY{p}{(}\PY{n}{r}\PY{p}{,} \PY{n}{m}\PY{p}{,} \PY{n}{sigma\PYZus{}m}\PY{p}{,} \PY{n}{sigma\PYZus{}r}\PY{p}{,} \PY{n}{fmt}\PY{o}{=}\PY{l+s+s1}{\PYZsq{}}\PY{l+s+s1}{o}\PY{l+s+s1}{\PYZsq{}}\PY{p}{)}
\PY{n}{popt}\PY{p}{,} \PY{n}{pcov} \PY{o}{=} \PY{n}{curve\PYZus{}fit}\PY{p}{(}\PY{n}{power\PYZus{}law}\PY{p}{,} \PY{n}{r}\PY{p}{,} \PY{n}{m}\PY{p}{)}
\PY{n}{norm\PYZus{}hat}\PY{p}{,} \PY{n}{index\PYZus{}hat} \PY{o}{=} \PY{n}{popt}
\PY{n}{sigma\PYZus{}norm}\PY{p}{,} \PY{n}{sigma\PYZus{}index} \PY{o}{=} \PY{n}{np}\PY{o}{.}\PY{n}{sqrt}\PY{p}{(}\PY{n}{pcov}\PY{o}{.}\PY{n}{diagonal}\PY{p}{(}\PY{p}{)}\PY{p}{)}
\PY{n+nb}{print}\PY{p}{(}\PY{n}{norm\PYZus{}hat}\PY{p}{,} \PY{n}{sigma\PYZus{}norm}\PY{p}{,} \PY{n}{index\PYZus{}hat}\PY{p}{,} \PY{n}{sigma\PYZus{}index}\PY{p}{)}
\PY{n}{x} \PY{o}{=} \PY{n}{np}\PY{o}{.}\PY{n}{linspace}\PY{p}{(}\PY{l+m+mf}{4.}\PY{p}{,} \PY{l+m+mf}{10.}\PY{p}{,} \PY{l+m+mi}{100}\PY{p}{)}
\PY{n}{plt}\PY{o}{.}\PY{n}{plot}\PY{p}{(}\PY{n}{x}\PY{p}{,} \PY{n}{power\PYZus{}law}\PY{p}{(}\PY{n}{x}\PY{p}{,} \PY{n}{norm\PYZus{}hat}\PY{p}{,} \PY{n}{index\PYZus{}hat}\PY{p}{)}\PY{p}{)}
\PY{n}{plt}\PY{o}{.}\PY{n}{xscale}\PY{p}{(}\PY{l+s+s1}{\PYZsq{}}\PY{l+s+s1}{log}\PY{l+s+s1}{\PYZsq{}}\PY{p}{)}
\PY{n}{plt}\PY{o}{.}\PY{n}{yscale}\PY{p}{(}\PY{l+s+s1}{\PYZsq{}}\PY{l+s+s1}{log}\PY{l+s+s1}{\PYZsq{}}\PY{p}{)}
\PY{n}{plt}\PY{o}{.}\PY{n}{xlabel}\PY{p}{(}\PY{l+s+s1}{\PYZsq{}}\PY{l+s+s1}{Raggio [mm]}\PY{l+s+s1}{\PYZsq{}}\PY{p}{)}
\PY{n}{plt}\PY{o}{.}\PY{n}{ylabel}\PY{p}{(}\PY{l+s+s1}{\PYZsq{}}\PY{l+s+s1}{Massa [g]}\PY{l+s+s1}{\PYZsq{}}\PY{p}{)}
\PY{n}{plt}\PY{o}{.}\PY{n}{grid}\PY{p}{(}\PY{n}{which}\PY{o}{=}\PY{l+s+s1}{\PYZsq{}}\PY{l+s+s1}{both}\PY{l+s+s1}{\PYZsq{}}\PY{p}{,} \PY{n}{ls}\PY{o}{=}\PY{l+s+s1}{\PYZsq{}}\PY{l+s+s1}{dashed}\PY{l+s+s1}{\PYZsq{}}\PY{p}{,} \PY{n}{color}\PY{o}{=}\PY{l+s+s1}{\PYZsq{}}\PY{l+s+s1}{gray}\PY{l+s+s1}{\PYZsq{}}\PY{p}{)}
\PY{n}{plt}\PY{o}{.}\PY{n}{savefig}\PY{p}{(}\PY{l+s+s1}{\PYZsq{}}\PY{l+s+s1}{massa\PYZus{}raggio.pdf}\PY{l+s+s1}{\PYZsq{}}\PY{p}{)}

\PY{n}{plt}\PY{o}{.}\PY{n}{show}\PY{p}{(}\PY{p}{)}
\end{Verbatim}



\end{article}
\end{document}
