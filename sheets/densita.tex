\documentclass{lab1-article}

\title{Misure di densit\`a}

\usepackage{fancyvrb}
\input{python}

\begin{document}


\begin{article}
\selectlanguage{italian}

\maketitle

\secsummary
Sappiamo che una quantit\`a fissata di qualunque sostanza o materiale occupa un
volume che varia soltanto se variano le condizioni in cui tale sostanza o
materiale si trova (ad esempio se dovesse passare dallo stato solido allo stato
liquido, o se cambiano la temperatura o la pressione).

La massa per unit\`a di volume \`e nota come densit\`a:
\begin{align}
  \rho = \frac{m}{V}~\left[\mathrm{kg/m}^3\right].
\end{align}
Si pu\`o distinguere, ad esempio, un metallo da un altro misurandone la
densit\`a.


\secmaterials

\begin{itemize}
\item Calibro ventesimale (risoluzione 0.05~mm).
\item Calibro cinquantesimale (risoluzione 0.02~mm).
\item Calibro Palmer (risoluzione 0.01~mm).
\item Bilancia di precisione (risoluzione 1~mg).
\item Una serie di solidi in alluminio, acciaio e ottone.

\end{itemize}


%\secdefinitions


%\secconsiderations


\secmeasurements

\labsubsection{Misure preliminari}

Si misurino le dimensioni (raggi, altezze, spessori, etc.) dei vari corpi
solidi e se ne calcoli il volume e la corrispondente incertezza di
misura---attraverso le regole usuali della propagazione dell'errore statistico
per grandezze indipendenti (i.e., utilizzando la somma in quadratura).

Si misuri anche la massa dei vari corpi e si costruisca una tabella
contenente i valore dei volumi e delle masse.


\labsubsection{Stima delle densit\`a}

Su un grafico cartesiano si riportino i valori delle masse in ascisse e quelli
dei volumi in ordinate. Poich\'e
\begin{align}
  V = \frac{m}{\rho},
\end{align}
i gruppi di punti corrispondenti allo stesso materiale dovrebbero disporsi su
linee rette passanti per l'origine, il cui coefficiente angolare coincide
 con l'inverso della densit\`a del materiale in questione.

Si esegua un \emph{fit} al calcolatore a ciascuno dei gruppi di punti che
si dispongono su queste rette e si stimi la densit\`a per i vari materiali,
propagando l'incertezza sul reciproco della densit\`a.


\labsubsection{Legge di scala per le sfere}

Si considerino le sole sfere e si costruisca un grafico, in scala
bilogaritmica, della massa $m$ in funzione del raggio $r$. Poich\'e la relazione
tra massa e raggio \`e di tipo legge di potenza:
\begin{align}
  m = \frac{4}{3}\pi\ r^3 \rho = kr^3,
\end{align}
i punti in scala bilogaritmica dovrebbero disporsi su una retta. Si trovi, con
l'utilizzo di un \emph{tool} grafico, la retta che meglio descrive i punti e si
confronti il coefficiente angolare della retta con il valore atteso
$3$ e l'intercetta (con l'asse $r = 1$) con il valore
\begin{align}
  k = \frac{4}{3}\pi\rho,
\end{align}
data la densit\`a misurata precedentemente.


\secappendix{densit\`a tabulate}

Si riportano di seguito i valori tabulati per i materiali rilevanti
per l'esperienza.

\medskip

\begin{center}
  \begin{tabular}{p{0.65\linewidth}@{}p{0.3\linewidth}}
    \hline
    Materiale & $\rho$ kg/m$^3$\\
    \hline
    \hline
    Alluminio & 2710\\
    Acciaio inossidabile  & 7480--8000\\
    Ottone (lega Cu-Zn)   & 8400--8700\\
    \hline
  \end{tabular}
\end{center}

\onecolumn

\begin{Verbatim}[label=\makebox{\href{https://github.com/unipi-physics-labs/lab1-sheets/tree/main/snippy/densita.py}{https://github.com/.../densita.py}},commandchars=\\\{\}]
\PY{k+kn}{import}\PY{+w}{ }\PY{n+nn}{numpy}\PY{+w}{ }\PY{k}{as}\PY{+w}{ }\PY{n+nn}{np}
\PY{k+kn}{from}\PY{+w}{ }\PY{n+nn}{matplotlib}\PY{+w}{ }\PY{k+kn}{import} \PY{n}{pyplot} \PY{k}{as} \PY{n}{plt}
\PY{k+kn}{from}\PY{+w}{ }\PY{n+nn}{scipy}\PY{n+nn}{.}\PY{n+nn}{optimize}\PY{+w}{ }\PY{k+kn}{import} \PY{n}{curve\PYZus{}fit}

\PY{c+c1}{\PYZsh{} Misure dirette dei diametri e delle masse per le sfere (mettete i vostri numeri).}
\PY{c+c1}{\PYZsh{} Potete anche leggere i dati da file, usando np.loadtxt(), se lo trovate comodo.}
\PY{n}{m} \PY{o}{=} \PY{n}{np}\PY{o}{.}\PY{n}{array}\PY{p}{(}\PY{p}{[}\PY{l+m+mf}{24.769}\PY{p}{,} \PY{l+m+mf}{11.887}\PY{p}{,} \PY{l+m+mf}{8.374}\PY{p}{,} \PY{l+m+mf}{3.528}\PY{p}{]}\PY{p}{)}
\PY{n}{sigma\PYZus{}m} \PY{o}{=} \PY{n}{np}\PY{o}{.}\PY{n}{full}\PY{p}{(}\PY{n}{m}\PY{o}{.}\PY{n}{shape}\PY{p}{,} \PY{l+m+mf}{0.001}\PY{p}{)}
\PY{n}{d} \PY{o}{=} \PY{n}{np}\PY{o}{.}\PY{n}{array}\PY{p}{(}\PY{p}{[}\PY{l+m+mf}{18.25}\PY{p}{,} \PY{l+m+mf}{14.28}\PY{p}{,} \PY{l+m+mf}{12.69}\PY{p}{,} \PY{l+m+mf}{9.52}\PY{p}{]}\PY{p}{)}
\PY{n}{sigma\PYZus{}d} \PY{o}{=} \PY{n}{np}\PY{o}{.}\PY{n}{full}\PY{p}{(}\PY{n}{d}\PY{o}{.}\PY{n}{shape}\PY{p}{,} \PY{l+m+mf}{0.01}\PY{p}{)}
\PY{c+c1}{\PYZsh{} Calcolo del volume (notate la propagazione dell\PYZsq{}errore su V!)}
\PY{n}{r} \PY{o}{=} \PY{n}{d} \PY{o}{/} \PY{l+m+mf}{2.0}
\PY{n}{sigma\PYZus{}r} \PY{o}{=} \PY{n}{sigma\PYZus{}d} \PY{o}{/} \PY{l+m+mf}{2.0}
\PY{n}{V} \PY{o}{=} \PY{l+m+mf}{4.0} \PY{o}{/} \PY{l+m+mf}{3.0} \PY{o}{*} \PY{n}{np}\PY{o}{.}\PY{n}{pi} \PY{o}{*} \PY{n}{r}\PY{o}{*}\PY{o}{*}\PY{l+m+mf}{3.0}
\PY{n}{sigma\PYZus{}V} \PY{o}{=} \PY{n}{V} \PY{o}{*} \PY{l+m+mf}{3.0} \PY{o}{*} \PY{n}{sigma\PYZus{}d} \PY{o}{/} \PY{n}{d}

\PY{k}{def}\PY{+w}{ }\PY{n+nf}{line}\PY{p}{(}\PY{n}{x}\PY{p}{,} \PY{n}{a}\PY{p}{,} \PY{n}{b}\PY{p}{)}\PY{p}{:}
    \PY{k}{return} \PY{n}{a} \PY{o}{*} \PY{n}{x} \PY{o}{+} \PY{n}{b}

\PY{n}{plt}\PY{o}{.}\PY{n}{figure}\PY{p}{(}\PY{l+s+s1}{\PYZsq{}}\PY{l+s+s1}{Grafico massa\PYZhy{}volume}\PY{l+s+s1}{\PYZsq{}}\PY{p}{)}
\PY{n}{plt}\PY{o}{.}\PY{n}{errorbar}\PY{p}{(}\PY{n}{m}\PY{p}{,} \PY{n}{V}\PY{p}{,} \PY{n}{sigma\PYZus{}V}\PY{p}{,} \PY{n}{sigma\PYZus{}m}\PY{p}{,} \PY{n}{fmt}\PY{o}{=}\PY{l+s+s1}{\PYZsq{}}\PY{l+s+s1}{o}\PY{l+s+s1}{\PYZsq{}}\PY{p}{)}
\PY{n}{popt}\PY{p}{,} \PY{n}{pcov} \PY{o}{=} \PY{n}{curve\PYZus{}fit}\PY{p}{(}\PY{n}{line}\PY{p}{,} \PY{n}{m}\PY{p}{,} \PY{n}{V}\PY{p}{)}
\PY{n}{a0}\PY{p}{,} \PY{n}{b0} \PY{o}{=} \PY{n}{popt}
\PY{n}{sigma\PYZus{}a}\PY{p}{,} \PY{n}{sigma\PYZus{}b} \PY{o}{=} \PY{n}{np}\PY{o}{.}\PY{n}{sqrt}\PY{p}{(}\PY{n}{pcov}\PY{o}{.}\PY{n}{diagonal}\PY{p}{(}\PY{p}{)}\PY{p}{)}
\PY{c+c1}{\PYZsh{} Attenzione alle cifre significative quando}
\PY{c+c1}{\PYZsh{} si riportano questi valori sulla relazione:}
\PY{n+nb}{print}\PY{p}{(}\PY{l+s+sa}{f}\PY{l+s+s1}{\PYZsq{}}\PY{l+s+s1}{a = }\PY{l+s+si}{\PYZob{}}\PY{n}{a0}\PY{l+s+si}{\PYZcb{}}\PY{l+s+s1}{ +/\PYZhy{} }\PY{l+s+si}{\PYZob{}}\PY{n}{sigma\PYZus{}a}\PY{l+s+si}{\PYZcb{}}\PY{l+s+s1}{\PYZsq{}}\PY{p}{)}
\PY{n+nb}{print}\PY{p}{(}\PY{l+s+sa}{f}\PY{l+s+s1}{\PYZsq{}}\PY{l+s+s1}{b = }\PY{l+s+si}{\PYZob{}}\PY{n}{b0}\PY{l+s+si}{\PYZcb{}}\PY{l+s+s1}{ +/\PYZhy{} }\PY{l+s+si}{\PYZob{}}\PY{n}{sigma\PYZus{}b}\PY{l+s+si}{\PYZcb{}}\PY{l+s+s1}{\PYZsq{}}\PY{p}{)}
\PY{c+c1}{\PYZsh{} Grafico del modello di best fit.}
\PY{n}{x} \PY{o}{=} \PY{n}{np}\PY{o}{.}\PY{n}{linspace}\PY{p}{(}\PY{l+m+mf}{0.}\PY{p}{,} \PY{l+m+mf}{30.}\PY{p}{,} \PY{l+m+mi}{100}\PY{p}{)}
\PY{n}{plt}\PY{o}{.}\PY{n}{plot}\PY{p}{(}\PY{n}{x}\PY{p}{,} \PY{n}{line}\PY{p}{(}\PY{n}{x}\PY{p}{,} \PY{n}{a0}\PY{p}{,} \PY{n}{b0}\PY{p}{)}\PY{p}{)}
\PY{n}{plt}\PY{o}{.}\PY{n}{ylabel}\PY{p}{(}\PY{l+s+s1}{\PYZsq{}}\PY{l+s+s1}{Volume [mm\PYZdl{}\PYZca{}3\PYZdl{}]}\PY{l+s+s1}{\PYZsq{}}\PY{p}{)}
\PY{n}{plt}\PY{o}{.}\PY{n}{xlabel}\PY{p}{(}\PY{l+s+s1}{\PYZsq{}}\PY{l+s+s1}{Massa [g]}\PY{l+s+s1}{\PYZsq{}}\PY{p}{)}
\PY{n}{plt}\PY{o}{.}\PY{n}{grid}\PY{p}{(}\PY{n}{which}\PY{o}{=}\PY{l+s+s1}{\PYZsq{}}\PY{l+s+s1}{both}\PY{l+s+s1}{\PYZsq{}}\PY{p}{,} \PY{n}{ls}\PY{o}{=}\PY{l+s+s1}{\PYZsq{}}\PY{l+s+s1}{dashed}\PY{l+s+s1}{\PYZsq{}}\PY{p}{,} \PY{n}{color}\PY{o}{=}\PY{l+s+s1}{\PYZsq{}}\PY{l+s+s1}{gray}\PY{l+s+s1}{\PYZsq{}}\PY{p}{)}
\PY{n}{plt}\PY{o}{.}\PY{n}{savefig}\PY{p}{(}\PY{l+s+s1}{\PYZsq{}}\PY{l+s+s1}{massa\PYZus{}volume.pdf}\PY{l+s+s1}{\PYZsq{}}\PY{p}{)}

\PY{n}{plt}\PY{o}{.}\PY{n}{figure}\PY{p}{(}\PY{l+s+s1}{\PYZsq{}}\PY{l+s+s1}{Grafico massa\PYZhy{}raggio}\PY{l+s+s1}{\PYZsq{}}\PY{p}{)}
\PY{n}{plt}\PY{o}{.}\PY{n}{errorbar}\PY{p}{(}\PY{n}{r}\PY{p}{,} \PY{n}{m}\PY{p}{,} \PY{n}{sigma\PYZus{}m}\PY{p}{,} \PY{n}{sigma\PYZus{}r}\PY{p}{,} \PY{n}{fmt}\PY{o}{=}\PY{l+s+s1}{\PYZsq{}}\PY{l+s+s1}{o}\PY{l+s+s1}{\PYZsq{}}\PY{p}{)}
\PY{n}{plt}\PY{o}{.}\PY{n}{xscale}\PY{p}{(}\PY{l+s+s1}{\PYZsq{}}\PY{l+s+s1}{log}\PY{l+s+s1}{\PYZsq{}}\PY{p}{)}
\PY{n}{plt}\PY{o}{.}\PY{n}{yscale}\PY{p}{(}\PY{l+s+s1}{\PYZsq{}}\PY{l+s+s1}{log}\PY{l+s+s1}{\PYZsq{}}\PY{p}{)}
\PY{n}{plt}\PY{o}{.}\PY{n}{xlabel}\PY{p}{(}\PY{l+s+s1}{\PYZsq{}}\PY{l+s+s1}{Raggio [mm]}\PY{l+s+s1}{\PYZsq{}}\PY{p}{)}
\PY{n}{plt}\PY{o}{.}\PY{n}{ylabel}\PY{p}{(}\PY{l+s+s1}{\PYZsq{}}\PY{l+s+s1}{Massa [g]}\PY{l+s+s1}{\PYZsq{}}\PY{p}{)}
\PY{n}{plt}\PY{o}{.}\PY{n}{grid}\PY{p}{(}\PY{n}{which}\PY{o}{=}\PY{l+s+s1}{\PYZsq{}}\PY{l+s+s1}{both}\PY{l+s+s1}{\PYZsq{}}\PY{p}{,} \PY{n}{ls}\PY{o}{=}\PY{l+s+s1}{\PYZsq{}}\PY{l+s+s1}{dashed}\PY{l+s+s1}{\PYZsq{}}\PY{p}{,} \PY{n}{color}\PY{o}{=}\PY{l+s+s1}{\PYZsq{}}\PY{l+s+s1}{gray}\PY{l+s+s1}{\PYZsq{}}\PY{p}{)}
\PY{n}{plt}\PY{o}{.}\PY{n}{savefig}\PY{p}{(}\PY{l+s+s1}{\PYZsq{}}\PY{l+s+s1}{massa\PYZus{}raggio.pdf}\PY{l+s+s1}{\PYZsq{}}\PY{p}{)}

\PY{n}{plt}\PY{o}{.}\PY{n}{show}\PY{p}{(}\PY{p}{)}
\end{Verbatim}



\end{article}
\end{document}
