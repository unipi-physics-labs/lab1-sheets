\documentclass{lab1-article}

\title{Test del $\chi^2$}


\begin{document}


\begin{article}
\selectlanguage{italian}

\maketitle

\secsummary

In questa esperienza cercheremo di familiarizzare con il test del $\chi^2$ in
alcune situazioni tipiche.


\secmaterials

\begin{itemize}
\item Computer, carta e penna.
\item Dati del pendolo fisico e della misura di g, buona volont\`a.
\end{itemize}


\secdefinitions

Data una serie di $n$ dati sperimentali $(x_i, y_i)$, con errori associati
$\sigma_{x_i}$ e $\sigma_{y_i}$, ed un modello $f(x_i;p_1\ldots p_m)$ dipendente
da $m$ parametri \emph{stimati dai dati}, la quantit\`a
\begin{align}\label{eq:chi2_series}
  S = \sum_{i = 1}^{n}
  \left( \frac{y_i - f(x_i;p_1\ldots p_m)}{\sigma_{y_i}} \right)^2
\end{align}
\`e una misura della \emph{distanza} dei dati stessi dal modello. Nell'ipotesi
in cui gli errori $\sigma_{x_i}$ siano \emph{trascurabili} ed i valori $y_i$
fluttuino gaussianamente attorno al modello con deviazione standard
$\sigma_{y_i}$, la variabile casuale $S$ \`e distribuita come un $\chi^2$ a
$\nu = n - m$ gradi di libert\`a.

Nel caso di una distribuzione, in cui si confrontano i conteggi osservati
$o_i$ nei canali del nostro istogramma sperimentale con i valori $e_i$ predetti
dal modello, la \eqref{eq:chi2_series} pu\`o essere riscritta come
\begin{align}
   S = \sum_{i = 1}^{n} \left( \frac{o_i - e_i}{\sqrt{e_i}} \right)^2 =
   \sum_{i = 1}^{n} \frac{(o_i - e_i)^2}{e_i}.
\end{align}
In questo caso, se il numero totale $n$ di canali \`e fissato,
il numero di gradi di libert\`a \`e $n - m - 1$.


\secconsiderations

\labsubsection{Stima dei conteggi attesi per una distribuzione}

Supponiamo di avere un istogramma (realizzato sperimentalmente) il cui
i-esimo canale contenga $o_i$ conteggi. Supponiamo ancora che il nostro
modello $f(x;p_1\ldots p_m)$ dipenda dalla variabile continua $x$.
Detti $x^L_i$ ed $x^H_i$ gli estremi inferiore e superiore del canale i-esimo,
il valore predetto dal modello sar\`a
\begin{align}
  e_i = n\int_{x^L_i}^{x^H_i} f(x;p_1\ldots p_m) \text{d}x,
\end{align}
dove $n$ \`e il numero totale di conteggi nell'istogramma. Notiamo
esplicitamente che, mentre $o_i$ \`e una variabile discreta, $e_i$ non
\`e necessariamente un numero intero.

Nel caso in cui il modello sia gaussiano non \`e possibile calcolare
analiticamente l'integrale, ma bisogna passare alla variabile in forma
standard ed utilizzare le tavole o il calcolatore.


\labsubsection{Realizzazione di un istogramma}

Il numero e la larghezza dei canali dovrebbero essere scelti in modo
che il numero medio di conteggi per canale sia ragionevole---in modo, cio\`e,
da non avere molti canali senza nessun conteggio o tutti i conteggi concentrati
in pochi canali. Per un numero totale di conteggi $n$, $~\sqrt{n}$
canali \`e tipicamente una scelta \emph{ragionevole}.

Per il test del $\chi^2$ si consiglia di raggruppare insieme i canali meno
popolati dell'istogramma in modo da non avere canali con meno di $2$ o
$3$ conteggi (ovviamente i corrispondenti valori attesi si sommano).


\secmeasurements

\labsubsection{Test per un modello completamente specificato}

Si consideri la serie di dati relativa all'esperienza del pendolo fisico
(in cui si misurava il periodo $T$ in funzione della distanza $d$ tra il
punto di sospensione ed il centro di massa) e si esegua un test del $\chi^2$
utilizzando il modello
\begin{align}
  T(d) = 2\pi\sqrt{\frac{(l^2/12 + d^2)}{gd}}.
\end{align}

Si consideri il modello \emph{completamente specificato} (cio\`e si
prendano i valori misurati di $l$ e $d$).
In questo caso, dati $n$ punti sperimentali, il numero di gradi di libert\`a
\`e semplicemente $\nu = n$.


\labsubsection{Test per un fit ad una serie di dati}

Si prenda una serie di dati, tra quelle acquisite in laboratorio durante il
semestre, che possa essere ragionevolmente descritta da un modello lineare---ad
esempio i periodi al quadrato, $T^2$, della molla con massa sul piattello $m$.
Si eseguano due fit con le funzioni
\begin{align}
  f_1(x) = a_1 x,\;\;f_2(x) = a_2 x + b
\end{align}
e si stimino i parametri $a_1, a_2, b$. Si esegua il test del $\chi^2$ per
scegliere la funzione che descrive meglio i dati calcolando i \emph{p-value}
corrispondenti. Si determini la massa \emph{mancante} a $m$.

\labsubsection{Test per una distribuzione}

Si considerino i dati nella tabella in appendice, in cui ognuno dei $70$ valori
corrisponde alla media di $10$~misure della stessa grandezza (nel caso
specifico il periodo di un pendolo, misurato in s).

Si realizzi un istogramma dei dati stessi, scegliendo opportunamente il numero
e la larghezza dei canali. Si calcolino quindi la media $m$ e la deviazione
standard $s$ del campione e si esegua un test di compatibilit\`a del $\chi^2$
con una distribuzione gaussiana di media $m$ e deviazione standard $s$.
In questo caso, dati $n$ canali per l'istogramma, il numero di gradi di
libert\`a \`e $\nu = n - 2$.


\begin{table*}
  \renewcommand\arraystretch{1.25}
  \begin{center}
    \begin{Verbatim}[label=\makebox{\href{https://github.com/unipi-physics-labs/lab1-sheets/tree/main/snippy/pendolo.py}{https://github.com/.../pendolo.py}},commandchars=\\\{\}]
\PY{k+kn}{import}\PY{+w}{ }\PY{n+nn}{ROOT}
\PY{n}{ROOT}\PY{o}{.}\PY{n}{gROOT}\PY{o}{.}\PY{n}{SetStyle}\PY{p}{(}\PY{l+s+s1}{\PYZsq{}}\PY{l+s+s1}{Plain}\PY{l+s+s1}{\PYZsq{}}\PY{p}{)}


\PY{n}{INPUT\PYZus{}FILE\PYZus{}PATH} \PY{o}{=} \PY{l+s+s1}{\PYZsq{}}\PY{l+s+s1}{data/pendolo.txt}\PY{l+s+s1}{\PYZsq{}}
\PY{n}{OUTPUT\PYZus{}FILE\PYZus{}PATH} \PY{o}{=} \PY{l+s+s1}{\PYZsq{}}\PY{l+s+s1}{data/pendolo.tex}\PY{l+s+s1}{\PYZsq{}}
\PY{n}{NUM\PYZus{}TABLE\PYZus{}COLS} \PY{o}{=} \PY{l+m+mi}{7}


\PY{n}{h} \PY{o}{=} \PY{n}{ROOT}\PY{o}{.}\PY{n}{TH1F}\PY{p}{(}\PY{l+s+s1}{\PYZsq{}}\PY{l+s+s1}{h}\PY{l+s+s1}{\PYZsq{}}\PY{p}{,} \PY{l+s+s1}{\PYZsq{}}\PY{l+s+s1}{h}\PY{l+s+s1}{\PYZsq{}}\PY{p}{,} \PY{l+m+mi}{14}\PY{p}{,} \PY{l+m+mf}{19.8}\PY{p}{,} \PY{l+m+mf}{20.8}\PY{p}{)}
\PY{n}{vals} \PY{o}{=} \PY{p}{[}\PY{p}{]}

\PY{n+nb}{print}\PY{p}{(}\PY{l+s+s1}{\PYZsq{}}\PY{l+s+s1}{Reading input file }\PY{l+s+si}{\PYZpc{}s}\PY{l+s+s1}{...}\PY{l+s+s1}{\PYZsq{}} \PY{o}{\PYZpc{}} \PY{n}{INPUT\PYZus{}FILE\PYZus{}PATH}\PY{p}{)}
\PY{k}{for} \PY{n}{line} \PY{o+ow}{in} \PY{n+nb}{open}\PY{p}{(}\PY{n}{INPUT\PYZus{}FILE\PYZus{}PATH}\PY{p}{)}\PY{p}{:}
    \PY{n}{line} \PY{o}{=} \PY{n}{line}\PY{o}{.}\PY{n}{strip}\PY{p}{(}\PY{l+s+s1}{\PYZsq{}}\PY{l+s+se}{\PYZbs{}n}\PY{l+s+s1}{\PYZsq{}}\PY{p}{)}
    \PY{k}{if} \PY{o+ow}{not} \PY{n}{line}\PY{o}{.}\PY{n}{startswith}\PY{p}{(}\PY{l+s+s1}{\PYZsq{}}\PY{l+s+s1}{\PYZsh{}}\PY{l+s+s1}{\PYZsq{}}\PY{p}{)} \PY{o+ow}{and} \PY{n+nb}{len}\PY{p}{(}\PY{n}{line}\PY{p}{)}\PY{p}{:}
        \PY{n}{val} \PY{o}{=} \PY{n+nb}{float}\PY{p}{(}\PY{n}{line}\PY{p}{)}
        \PY{n}{vals}\PY{o}{.}\PY{n}{append}\PY{p}{(}\PY{n}{val}\PY{p}{)}
        \PY{n}{h}\PY{o}{.}\PY{n}{Fill}\PY{p}{(}\PY{n}{val}\PY{p}{)}
\PY{n+nb}{print}\PY{p}{(}\PY{l+s+s1}{\PYZsq{}}\PY{l+s+s1}{Done. }\PY{l+s+si}{\PYZpc{}d}\PY{l+s+s1}{ value(s) found.}\PY{l+s+s1}{\PYZsq{}} \PY{o}{\PYZpc{}} \PY{n+nb}{len}\PY{p}{(}\PY{n}{vals}\PY{p}{)}\PY{p}{)}

\PY{n}{vals}\PY{o}{.}\PY{n}{sort}\PY{p}{(}\PY{p}{)}
\PY{n}{h}\PY{o}{.}\PY{n}{Draw}\PY{p}{(}\PY{p}{)}
\PY{n}{f} \PY{o}{=} \PY{n}{ROOT}\PY{o}{.}\PY{n}{TF1}\PY{p}{(}\PY{l+s+s1}{\PYZsq{}}\PY{l+s+s1}{f}\PY{l+s+s1}{\PYZsq{}}\PY{p}{,} \PY{l+s+s1}{\PYZsq{}}\PY{l+s+s1}{gaus}\PY{l+s+s1}{\PYZsq{}}\PY{p}{)}
\PY{n}{h}\PY{o}{.}\PY{n}{Fit}\PY{p}{(}\PY{l+s+s1}{\PYZsq{}}\PY{l+s+s1}{f}\PY{l+s+s1}{\PYZsq{}}\PY{p}{)}
\PY{n}{ROOT}\PY{o}{.}\PY{n}{gPad}\PY{o}{.}\PY{n}{Update}\PY{p}{(}\PY{p}{)}

\PY{n+nb}{print}\PY{p}{(}\PY{l+s+s1}{\PYZsq{}}\PY{l+s+s1}{Fit chisquare = }\PY{l+s+si}{\PYZpc{}.3f}\PY{l+s+s1}{/}\PY{l+s+si}{\PYZpc{}d}\PY{l+s+s1}{\PYZsq{}} \PY{o}{\PYZpc{}} \PY{p}{(}\PY{n}{f}\PY{o}{.}\PY{n}{GetChisquare}\PY{p}{(}\PY{p}{)}\PY{p}{,} \PY{n}{f}\PY{o}{.}\PY{n}{GetNDF}\PY{p}{(}\PY{p}{)}\PY{p}{)}\PY{p}{)}

\PY{n}{outputFile} \PY{o}{=} \PY{n+nb}{open}\PY{p}{(}\PY{n}{OUTPUT\PYZus{}FILE\PYZus{}PATH}\PY{p}{,} \PY{l+s+s1}{\PYZsq{}}\PY{l+s+s1}{w}\PY{l+s+s1}{\PYZsq{}}\PY{p}{)}

\PY{k}{def}\PY{+w}{ }\PY{n+nf}{write}\PY{p}{(}\PY{n}{stuff}\PY{p}{)}\PY{p}{:}
    \PY{n}{outputFile}\PY{o}{.}\PY{n}{write}\PY{p}{(}\PY{n}{stuff}\PY{p}{)}

\PY{n}{fmt} \PY{o}{=} \PY{l+s+s1}{\PYZsq{}}\PY{l+s+s1}{l}\PY{l+s+s1}{\PYZsq{}} \PY{o}{*} \PY{n}{NUM\PYZus{}TABLE\PYZus{}COLS}
\PY{n}{write}\PY{p}{(}\PY{l+s+s1}{\PYZsq{}}\PY{l+s+se}{\PYZbs{}\PYZbs{}}\PY{l+s+s1}{begin}\PY{l+s+si}{\PYZob{}tabular\PYZcb{}}\PY{l+s+s1}{\PYZob{}}\PY{l+s+si}{\PYZpc{}s}\PY{l+s+s1}{\PYZcb{}}\PY{l+s+se}{\PYZbs{}n}\PY{l+s+s1}{\PYZsq{}} \PY{o}{\PYZpc{}} \PY{n}{fmt}\PY{p}{)}
\PY{n}{write}\PY{p}{(}\PY{l+s+s1}{\PYZsq{}}\PY{l+s+se}{\PYZbs{}\PYZbs{}}\PY{l+s+s1}{hline}\PY{l+s+se}{\PYZbs{}n}\PY{l+s+s1}{\PYZsq{}}\PY{p}{)}
\PY{n}{numRows} \PY{o}{=} \PY{n+nb}{int}\PY{p}{(}\PY{n+nb}{len}\PY{p}{(}\PY{n}{vals}\PY{p}{)}\PY{o}{/}\PY{n}{NUM\PYZus{}TABLE\PYZus{}COLS} \PY{o}{\PYZhy{}} \PY{l+m+mf}{0.0001}\PY{p}{)} \PY{o}{+} \PY{l+m+mi}{1}
\PY{k}{for} \PY{n}{row} \PY{o+ow}{in} \PY{n+nb}{range}\PY{p}{(}\PY{n}{numRows}\PY{p}{)}\PY{p}{:}
    \PY{n}{text} \PY{o}{=} \PY{l+s+s1}{\PYZsq{}}\PY{l+s+s1}{\PYZsq{}}
    \PY{k}{for} \PY{n}{col} \PY{o+ow}{in} \PY{n+nb}{range}\PY{p}{(}\PY{n}{NUM\PYZus{}TABLE\PYZus{}COLS}\PY{p}{)}\PY{p}{:}
        \PY{n}{i} \PY{o}{=} \PY{n}{row} \PY{o}{+} \PY{n}{numRows}\PY{o}{*}\PY{n}{col}
        \PY{k}{try}\PY{p}{:}
            \PY{n}{val} \PY{o}{=} \PY{n}{vals}\PY{p}{[}\PY{n}{i}\PY{p}{]}
            \PY{n}{text} \PY{o}{+}\PY{o}{=} \PY{l+s+s1}{\PYZsq{}}\PY{l+s+s1}{\PYZbs{}}\PY{l+s+s1}{makebox[28pt]}\PY{l+s+s1}{\PYZob{}}\PY{l+s+s1}{\PYZbs{}}\PY{l+s+s1}{hfill[}\PY{l+s+si}{\PYZpc{}d}\PY{l+s+s1}{]\PYZcb{}\PYZti{}}\PY{l+s+si}{\PYZpc{}.3f}\PY{l+s+s1}{ \PYZam{} }\PY{l+s+s1}{\PYZsq{}} \PY{o}{\PYZpc{}} \PY{p}{(}\PY{n}{i} \PY{o}{+} \PY{l+m+mi}{1}\PY{p}{,} \PY{n}{val}\PY{p}{)}
        \PY{k}{except}\PY{p}{:}
            \PY{k}{pass}
    \PY{n}{text} \PY{o}{=} \PY{n}{text}\PY{o}{.}\PY{n}{strip}\PY{p}{(}\PY{l+s+s1}{\PYZsq{}}\PY{l+s+s1}{ \PYZam{}}\PY{l+s+s1}{\PYZsq{}}\PY{p}{)}
    \PY{n}{write}\PY{p}{(}\PY{l+s+s1}{\PYZsq{}}\PY{l+s+si}{\PYZpc{}s}\PY{l+s+se}{\PYZbs{}\PYZbs{}}\PY{l+s+se}{\PYZbs{}\PYZbs{}}\PY{l+s+se}{\PYZbs{}n}\PY{l+s+s1}{\PYZsq{}} \PY{o}{\PYZpc{}} \PY{n}{text}\PY{p}{)}
\PY{n}{write}\PY{p}{(}\PY{l+s+s1}{\PYZsq{}}\PY{l+s+se}{\PYZbs{}\PYZbs{}}\PY{l+s+s1}{hline}\PY{l+s+se}{\PYZbs{}n}\PY{l+s+s1}{\PYZsq{}}\PY{p}{)} 
\PY{n}{write}\PY{p}{(}\PY{l+s+s1}{\PYZsq{}}\PY{l+s+se}{\PYZbs{}\PYZbs{}}\PY{l+s+s1}{end}\PY{l+s+si}{\PYZob{}tabular\PYZcb{}}\PY{l+s+se}{\PYZbs{}n}\PY{l+s+s1}{\PYZsq{}}\PY{p}{)}
\PY{n}{outputFile}\PY{o}{.}\PY{n}{close}\PY{p}{(}\PY{p}{)}
\end{Verbatim}

  \end{center}
  \caption{Tabella delle misure di periodo. Ogni valore rappresenta la media
  di $10$ misurazioni dirette. Per facilitare la realizzazione dell'istogramma
  le medie sono riportate in ordine crescente.}
\end{table*}


\end{article}
\end{document}
