\documentclass{lab1-article}

\title{La relazione di Laboratorio: alcuni suggerimenti}
\usepackage[italian]{babel}
\usepackage{enumitem}
\begin{document}


\begin{article}
\selectlanguage{italian}

\maketitle

\secsummary
Lo scopo della relazione di laboratorio \`e quella di descrivere, in modo pi\`u chiaro possibile, le procedure utilizzate e i
risultati ottenuti durante l'esperienza.\\ 
Comunicare il risultato del proprio lavoro \`e fondamentale nella ricerca scientifica,
e le relazioni di laboratorio sono un primo passo verso l'acquisizione delle adeguate capacit\`a di comunicazione.\\
Chi non ha mai scritto una relazione (ad es. alle scuole superiori), pu\`o incontrare alcune difficolt\`a iniziali. Per
questo motivo abbiamo riportato qui alcuni suggerimenti e linee guida per la preparazione delle relazioni.
\section{Considerazioni generali}
Nella stesura della relazione, sono fondamentali la \emph{chiarezza} e la \emph{completezza}. Chi legge il testo, ad es. un
docente o anche un altro studente, deve capire cosa \`e stato fatto e come sono stati ottenuti i risultati. Infatti, leggendo
la relazione, un altro sperimentatore deve poter essere in grado di ripetere esattamente la misura. Inoltre, se la
relazione \`e stata scritta in modo chiaro, sar\`a pi\`u facile rileggerla a distanza di tempo, ad es. durante la
preparazione all'esame di fine corso.\\
Un altro aspetto molto importante \`e la \emph{concisione}. La relazione infatti \emph{non \`e un trattato di fisica}. E' giusto introdurre
brevemente i concetti di fisica e le equazioni utilizzate nella relazione, ma senza dilungarsi troppo.\\ 
Inoltre, la relazione
\emph{non \`e un diario delle attivit\`a} svolte in laboratorio. Bisogna senza dubbio spiegare come sono state effettuate le misure
e le difficolt\`a incontrate, ma evitando i dettagli non rilevanti. Ad es., \`e opportuno ridurre al minimo i commenti
soggettivi e mantenere lo stile asciutto. E' molto pi\`u importante concentrarsi \emph{sulle misure effettuate, sull'analisi dei dati
e sui risultati ottenuti}.
\section{La struttura della relazione}
Ecco un possibile schema per la stesura della relazione:
\begin{enumerate}[label={\alph*)}]
\item Titolo e data
\item Nome e cognome del/i partecipante/i;
\item Definizione degli obiettivi:
\begin{itemize}
 \item Spiegare, \emph{brevemente}, lo scopo dell'esperienza;
\end{itemize}
\item Introduzione teorica e formule utilizzate:
\begin{itemize}
 \item Citare il risultato teorico senza ricavare la teoria del fenomeno, e poi scrivere le formule nella forma usata durante l'esperienza (ad es., passare da $T$ a $T^{2}$ per evideziare andamenti costanti o lineari).
% Presentare i concetti fisici e le equazioni utilizzate durante l’esperienza. Un paio di paragrafi sono pi\`uche sufficienti;
\end{itemize}
\item Apparato sperimentale e strumenti di misura:
\begin{itemize}
 \item Descrivere l'apparato e gli strumenti utilizzati, riportando per ciascuno la risoluzione;
\end{itemize}
\item Descrizione delle misure:
\begin{itemize}
 \item Descrivere le misure effettuate e le incertezze associate. E' buona norma riportare i dati grezzi, quando non sono in numero eccessivo.
 \item Bisogna dire cosa \`e stato fatto davvero (difficolt\`a e imprevisti), non quello che sarebbe dovuto succedere!
\end{itemize}
\item Analisi dei dati:
\begin{itemize}
 \item Descrivere i metodi di analisi utilizzati (ad es. i \emph{fit}), includendo le tabelle con i dati raccolti e gli
eventuali grafici;
 \item E' anche molto importante riportare come \`e stata stimata e propagata l'incertezza sulle varie quantit\`a;
\end{itemize}
\item Risultati finali e Discussione:
\begin{itemize}
 \item Riportare in modo chiaro il risultato finale dell'esperienza;
 \item Discutere i risultati, ad es. la bont\`a in termini di errore relativo e accordo con il modello teorico.
Bisogna supportare le conclusioni con stime quantitative, ad
es. valutando la differenza dati-teoria alla luce delle incertezze di misura;
 \item E' anche possibile discutere possibili idee per migliorare le misure o la stima degli errori;
\end{itemize}
\end{enumerate}
\section{Cose da evitare}
\begin{itemize}
 \item Riportare la misura di una quantit\`a senza riportare le unit\`a di misura o la sua incertezza;
 \item Riportare in modo errato le cifre significative;
 \item Non propagare gli errori;
 \item Non includere i grafici o i dati raccolti;
 \item Non riportare nelle tabelle, o sugli assi di un grafico il nome della quantit\`a misurata e/o la sua unit\`a di misura;
 \item Riportare i valori delle misure direttamente sugli assi;
 \item Non evidenziare i risultati finali dell'esperienza;
 \item Discutere in maniera troppo qualitativa i risultati. Ad es.: ''L'accordo con la teoria \`e abbastanza buono..''
 \item Usare vocaboli ed espressioni colorite di dubbia correttezza, come ``graficare'', ``plottare'', ``misure sospette'',
``fare senso'', ecc...
\end{itemize}

\end{article}
\end{document}
