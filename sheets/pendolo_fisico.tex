\documentclass{lab1-article}

\title{Pendolo fisico}

\usepackage{fancyvrb}
\input{python}


\begin{document}


\begin{article}
\selectlanguage{italian}

\maketitle

\secsummary
Lo scopo dell'esperienza \`e quello di misurare il periodo di un pendolo
fisico in funzione della distanza del centro di massa dal punto di
sospensione.


\secmaterials

\begin{itemize}
\item Un'asta metallica forata.
\item Un supporto di sospensione.
\item Cronometro (risoluzione 0.01~s).
\item Metro a nastro (risoluzione 1~mm).
\item Calibro ventesimale (risoluzione 0.05~mm).
\end{itemize}

\secmeasurements

\begin{figure}[htb!]
  \begin{tikzpicture}[scale=1]
    \pgfmathsetmacro{\xc}{2.25}
    \pgfmathsetmacro{\yc}{0}
    \pgfmathsetmacro{\r}{6}
    \pgfmathsetmacro{\w}{0.5}
    \pgfmathsetmacro{\rangle}{1.5}
    \pgfmathsetmacro{\thetazero}{30}
    \node at (0, 0) {};
    \draw[rotate around={\thetazero:(\xc, \yc)}]%
    (\xc - 0.5*\w, \yc + \w) rectangle (\xc + 0.5*\w, \yc - \r - \w);
    \fill (\xc, \yc) circle [radius=0.075];
    \node at (\xc + \w, \yc + 0.5*\w) {$P$};
    \fill (\xc + 0.5*\r*sin{\thetazero}, \yc - 0.5*\r*cos{\thetazero})%
    circle [radius=0.075];
    \node[anchor=west] at%
    (\xc + 0.5*\r*sin{\thetazero} + 0.5*\w,
    \yc - 0.5*\r*cos{\thetazero} + 0.5*\w) {Centro di massa};
    \draw[style=densely dashed] (\xc, \yc + 2*\w)--(\xc, \yc - \r);
    \draw[style=densely dashed] (\xc, \yc)--%
    (\xc + 0.5*\r*sin{\thetazero}, \yc - 0.5*\r*cos{\thetazero});
    \draw (\xc, \yc - \rangle) arc (270:270+\thetazero:\rangle);
    \node[anchor=north] at (\xc + \w, \yc - \rangle) {$\theta$};
    \node[anchor=west] at%
    (\xc + 0.25*\r*sin{\thetazero} + 0.5*\w,
    \yc - 0.25*\r*cos{\thetazero} + 0.25*\w) {$d$};
    \draw[->] (\xc + 0.5*\r*sin{\thetazero}, \yc - 0.5*\r*cos{\thetazero})--%
    (\xc + 0.5*\r*sin{\thetazero}, \yc - 0.5*\r*cos{\thetazero} - 2);
    \node[anchor=north] at%
    (\xc + 0.5*\r*sin{\thetazero}, \yc - 0.5*\r*cos{\thetazero} - 2)%
         {$m\vec{g}$};
  \end{tikzpicture}
  \caption{Schematizzazione dell'apparato sperimentale e definizioni di base.}
  \label{fig:pendolo}
\end{figure}

Un qualunque oggetto fissato ad un punto di sospensione $P$ (che disti $d$
dal centro di massa) e soggetto alla gravit\`a costituisce un pendolo fisico.
Se il pendolo viene spostato di un angolo $\theta$ dalla posizione di
equilibrio, il momento della forza di gravit\`a (rispetto al polo $P$) vale
allora
\begin{align*}
  \tau = -mgd \sin\theta,
\end{align*}
che ad angoli \emph{piccoli} (cosa significa?) diventa
\begin{align}\label{eq:tau_small}
  \tau = - mgd\theta.
\end{align}
D'altra parte, per la seconda equazione cardinale, si ha
\begin{align*}
  \tau = \frac{dL}{dt},
\end{align*}
ed usando le relazioni $L = I\omega$ e $\omega = d\theta/dt$ abbiamo
\begin{align}\label{eq:eq_motion}
  \tau = I \frac{d^2\theta}{dt^2}
\end{align}

La \eqref{eq:tau_small} e la \eqref{eq:eq_motion} permettono di scrivere
\begin{align}
\frac{d^2\theta}{dt^2} + \frac{mgd}{I}\theta = 0.
\end{align}
Si tratta dell'equazione differenziale di un moto armonico di pulsazione
angolare e periodo date rispettivamente da
\begin{align*}
  \omega_0 = \sqrt{\frac{mgd}{I}} \quad
  T_0 = \frac{2\pi}{\omega_0} = 2\pi\sqrt{\frac{I}{mgd}}.
\end{align*}

Sapendo che il momento di inerzia dell'asta (di massa $m$ e lunghezza $l$)
rispetto ad un punto $P$ che dista $d$ dal centro di massa, vale
\begin{align*}
  I = I_{\rm cm} + md^2 = \frac{ml^2}{12} + md^2,
\end{align*}
si ha infine
\begin{align}\label{eq:period}
  T(d) = 2\pi\sqrt{\frac{(l^2/12 + d^2)}{gd}}.
\end{align}


\labsubsection{Dipendenza del periodo da $d$}

Si misuri il periodo di oscillazione $T$ al variare della distanza $d$ del
punto di sospensione dal centro di massa (per i valori $d_i$ corrispondenti ai
fori nella sbarra) e si riporti su di un grafico di dispersione i valori
misurati $T_i$ in funzione delle distanze $d_i$.

Si esegua un fit dei dati con il modello~(\ref{eq:period}), usando $l$ come
parametro libero, e si confronti il valore di \emph{best-fit} $\hat{l}$ con la
lunghezza misurata dell'asta.

Si faccia un grafico dei residui
\begin{align*}
  r_i = T_i - 2\pi\sqrt{\frac{(\hat{l}^2/12 + d_i^2)}{gd_i}}.
\end{align*}
in funzione di $d_i$ per studiare qualitativamente l'accordo del modello con i
dati raccolti.



\secconsiderations

\labsubsection{Misura del periodo}

Anche se la risoluzione del cronometro usato vale 0.01~s, \`e illusorio
pensare che questo sia l'errore di misura da attribuire a misurazioni di tempo
manuali.
Per ridurre l'impatto del tempo di reazione, si consiglia
di misurare il tempo $\tau$ che il sistema impiega a compiere $10$
oscillazioni complete. Per stimare l'errore associato a $\tau$ si ripeta la
misure $n$ volte (con $n \geq 5$) e si prenda il valor medio e la deviazione
standard della media delle misure come miglior stima del misurando ed incertezza
associata, rispettivamente.
(Va da s\'e che si passa da $\tau$ a $T$ dividendo per $10$ sia la misura che l'errore.)

\onecolumn

\begin{Verbatim}[label=\makebox{\href{https://github.com/unipi-physics-labs/lab1-sheets/tree/main/snippy/pendolo_fisico.py}{https://github.com/.../pendolo\_fisico.py}},commandchars=\\\{\}]
\PY{k+kn}{import}\PY{+w}{ }\PY{n+nn}{numpy}\PY{+w}{ }\PY{k}{as}\PY{+w}{ }\PY{n+nn}{np}
\PY{k+kn}{from}\PY{+w}{ }\PY{n+nn}{matplotlib}\PY{+w}{ }\PY{k+kn}{import} \PY{n}{pyplot} \PY{k}{as} \PY{n}{plt}
\PY{k+kn}{from}\PY{+w}{ }\PY{n+nn}{scipy}\PY{n+nn}{.}\PY{n+nn}{optimize}\PY{+w}{ }\PY{k+kn}{import} \PY{n}{curve\PYZus{}fit}

\PY{c+c1}{\PYZsh{} Dati\PYZhy{}\PYZhy{}\PYZhy{}mettete le vostre misure!}
\PY{c+c1}{\PYZsh{} Qui potete anche leggere i dati da file, usando il metodo np.loadtxt(),}
\PY{c+c1}{\PYZsh{} se lo trovate comodo.}
\PY{n}{d} \PY{o}{=} \PY{n}{np}\PY{o}{.}\PY{n}{array}\PY{p}{(}\PY{p}{[}\PY{l+m+mf}{0.100}\PY{p}{,} \PY{l+m+mf}{0.200}\PY{p}{,} \PY{l+m+mf}{0.300}\PY{p}{,} \PY{l+m+mf}{0.400}\PY{p}{]}\PY{p}{)}
\PY{n}{sigma\PYZus{}d} \PY{o}{=} \PY{n}{np}\PY{o}{.}\PY{n}{full}\PY{p}{(}\PY{n}{d}\PY{o}{.}\PY{n}{shape}\PY{p}{,} \PY{l+m+mf}{0.002}\PY{p}{)}
\PY{n}{T} \PY{o}{=} \PY{n}{np}\PY{o}{.}\PY{n}{array}\PY{p}{(}\PY{p}{[}\PY{l+m+mf}{1.943}\PY{p}{,} \PY{l+m+mf}{1.570}\PY{p}{,} \PY{l+m+mf}{1.518}\PY{p}{,} \PY{l+m+mf}{1.572}\PY{p}{]}\PY{p}{)}
\PY{n}{sigma\PYZus{}T} \PY{o}{=} \PY{n}{np}\PY{o}{.}\PY{n}{array}\PY{p}{(}\PY{p}{[}\PY{l+m+mf}{0.005}\PY{p}{,} \PY{l+m+mf}{0.004}\PY{p}{,} \PY{l+m+mf}{0.006}\PY{p}{,} \PY{l+m+mf}{0.005}\PY{p}{]}\PY{p}{)}

\PY{c+c1}{\PYZsh{} Definizione dell\PYZsq{}accelerazione di gravita`.}
\PY{n}{g} \PY{o}{=} \PY{l+m+mf}{9.81}

\PY{k}{def}\PY{+w}{ }\PY{n+nf}{period\PYZus{}model}\PY{p}{(}\PY{n}{d}\PY{p}{,} \PY{n}{l}\PY{p}{)}\PY{p}{:}
\PY{+w}{    }\PY{l+s+sd}{\PYZdq{}\PYZdq{}\PYZdq{}Modello per il periodo del pendolo.}
\PY{l+s+sd}{    \PYZdq{}\PYZdq{}\PYZdq{}}
    \PY{k}{return} \PY{l+m+mf}{2.0} \PY{o}{*} \PY{n}{np}\PY{o}{.}\PY{n}{pi} \PY{o}{*} \PY{n}{np}\PY{o}{.}\PY{n}{sqrt}\PY{p}{(}\PY{p}{(}\PY{n}{l}\PY{o}{*}\PY{o}{*}\PY{l+m+mf}{2.0} \PY{o}{/} \PY{l+m+mf}{12.0} \PY{o}{+} \PY{n}{d}\PY{o}{*}\PY{o}{*}\PY{l+m+mf}{2.0}\PY{p}{)} \PY{o}{/} \PY{p}{(}\PY{n}{g} \PY{o}{*} \PY{n}{d}\PY{p}{)}\PY{p}{)}

\PY{n}{plt}\PY{o}{.}\PY{n}{figure}\PY{p}{(}\PY{l+s+s1}{\PYZsq{}}\PY{l+s+s1}{Periodo}\PY{l+s+s1}{\PYZsq{}}\PY{p}{)}
\PY{c+c1}{\PYZsh{} Scatter plot dei dati.}
\PY{n}{plt}\PY{o}{.}\PY{n}{errorbar}\PY{p}{(}\PY{n}{d}\PY{p}{,} \PY{n}{T}\PY{p}{,} \PY{n}{sigma\PYZus{}T}\PY{p}{,} \PY{n}{sigma\PYZus{}d}\PY{p}{,} \PY{n}{fmt}\PY{o}{=}\PY{l+s+s1}{\PYZsq{}}\PY{l+s+s1}{o}\PY{l+s+s1}{\PYZsq{}}\PY{p}{)}

\PY{c+c1}{\PYZsh{} Fit\PYZhy{}\PYZhy{}\PYZhy{}notate che questo e` un fit ad un solo parametro.}
\PY{n}{popt}\PY{p}{,} \PY{n}{pcov} \PY{o}{=} \PY{n}{curve\PYZus{}fit}\PY{p}{(}\PY{n}{period\PYZus{}model}\PY{p}{,} \PY{n}{d}\PY{p}{,} \PY{n}{T}\PY{p}{,} \PY{n}{sigma}\PY{o}{=}\PY{n}{sigma\PYZus{}T}\PY{p}{)}
\PY{n}{l0} \PY{o}{=} \PY{n}{popt}\PY{p}{[}\PY{l+m+mi}{0}\PY{p}{]}
\PY{n}{sigma\PYZus{}l} \PY{o}{=} \PY{n}{np}\PY{o}{.}\PY{n}{sqrt}\PY{p}{(}\PY{n}{pcov}\PY{p}{[}\PY{l+m+mi}{0}\PY{p}{,} \PY{l+m+mi}{0}\PY{p}{]}\PY{p}{)}
\PY{c+c1}{\PYZsh{} Confrontate i parametri di best fit con la vostra misura diretta!}
\PY{n+nb}{print}\PY{p}{(}\PY{l+s+sa}{f}\PY{l+s+s1}{\PYZsq{}}\PY{l+s+s1}{l0 = }\PY{l+s+si}{\PYZob{}}\PY{n}{l0}\PY{l+s+si}{\PYZcb{}}\PY{l+s+s1}{ +/\PYZhy{} }\PY{l+s+si}{\PYZob{}}\PY{n}{sigma\PYZus{}l}\PY{l+s+si}{\PYZcb{}}\PY{l+s+s1}{\PYZsq{}}\PY{p}{)}

\PY{c+c1}{\PYZsh{} Grafico del modello di best\PYZhy{}fit.}
\PY{n}{x} \PY{o}{=} \PY{n}{np}\PY{o}{.}\PY{n}{linspace}\PY{p}{(}\PY{l+m+mf}{0.05}\PY{p}{,} \PY{l+m+mf}{0.5}\PY{p}{,} \PY{l+m+mi}{100}\PY{p}{)}
\PY{n}{plt}\PY{o}{.}\PY{n}{plot}\PY{p}{(}\PY{n}{x}\PY{p}{,} \PY{n}{period\PYZus{}model}\PY{p}{(}\PY{n}{x}\PY{p}{,} \PY{n}{l0}\PY{p}{)}\PY{p}{)}
\PY{n}{plt}\PY{o}{.}\PY{n}{xlabel}\PY{p}{(}\PY{l+s+s1}{\PYZsq{}}\PY{l+s+s1}{d [m]}\PY{l+s+s1}{\PYZsq{}}\PY{p}{)}
\PY{n}{plt}\PY{o}{.}\PY{n}{ylabel}\PY{p}{(}\PY{l+s+s1}{\PYZsq{}}\PY{l+s+s1}{Periodo [s]}\PY{l+s+s1}{\PYZsq{}}\PY{p}{)}
\PY{n}{plt}\PY{o}{.}\PY{n}{grid}\PY{p}{(}\PY{n}{which}\PY{o}{=}\PY{l+s+s1}{\PYZsq{}}\PY{l+s+s1}{both}\PY{l+s+s1}{\PYZsq{}}\PY{p}{,} \PY{n}{ls}\PY{o}{=}\PY{l+s+s1}{\PYZsq{}}\PY{l+s+s1}{dashed}\PY{l+s+s1}{\PYZsq{}}\PY{p}{,} \PY{n}{color}\PY{o}{=}\PY{l+s+s1}{\PYZsq{}}\PY{l+s+s1}{gray}\PY{l+s+s1}{\PYZsq{}}\PY{p}{)}
\PY{n}{plt}\PY{o}{.}\PY{n}{savefig}\PY{p}{(}\PY{l+s+s1}{\PYZsq{}}\PY{l+s+s1}{massa\PYZus{}raggio.pdf}\PY{l+s+s1}{\PYZsq{}}\PY{p}{)}

\PY{n}{plt}\PY{o}{.}\PY{n}{show}\PY{p}{(}\PY{p}{)}
\end{Verbatim}



\end{article}
\end{document}
